\chaplbl{Glossary}{s:gloss}

\begin{description}

\gitem{g:absolute-beginner}{Absolute beginner} Someone who has never
encountered concepts or material before. The term is used in distinction to
\gref{g:false-beginner}{false beginner}.

\gitem{g:active-learning}{Active learning} An approach to instruction in which
learners engage with material through discussion, problem solving, case studies,
and other activities that require them to reflect on and use new information in
real time.  See also \gref{g:passive-learning}{passive learning}.

\gitem{g:active-teaching}{Active teaching} An approach to instruction in which the
instructor acts on new information acquired from learners while teaching (e.g.,
by dynamically changing an example or rearranging the intended order of
content).  See also \gref{g:passive-teaching}{passive teaching}.

\gitem{g:authentic-task}{Authentic task} A task which contains important
elements of things that learners would do in real (non-classroom situations). To
be authentic, a task should require learners to construct their own answers
rather than choose between provided answers, and to work with the same tools and
data they would use in real life.

\gitem{g:automaticity}{Automaticity} The ability to do a task without
concentrating on its low-level details.

\gitem{g:backward-design}{Backward design} An instructional design method that
works backwards from a summative assessment to formative assessments and thence
to lesson content.

\gitem{g:behaviorism}{Behaviorism} A theory of learning whose central principle
is stimulus and response, and whose goal is to explain behavior without recourse
to internal mental states or other unobservables. See
also \gref{g:cognitivism}{cognitivism}.

\gitem{g:blooms-taxonomy}{Bloom's Taxonomy} A six-part hierarchical
classification of understand whose levels are \emph{knowledge},
\emph{comprehension}, \emph{application}, \emph{analysis}, \emph{synthesis}, and
\emph{evaluation} that has been widely adopted. See also
\gref{g:finks-taxonomy}{Fink's Taxonomy}.

\gitem{g:brand}{Brand} The associations people have with a product's name or
identity.

\gitem{g:calibrated-peer-review}{Calibrated peer review} Having students
compare their reviews of sample work with an instructor's reviews before being
allowed to review their peers' work.

\gitem{g:chunking}{Chunking} The act of grouping related concepts together so
that they can be stored and processed as a single unit.

\gitem{g:co-teaching}{Co-teaching} Teaching with another instructor in the
classroom.

\gitem{g:cognitive-apprenticeship}{Cognitive apprenticeship} A theory of
learning that emphasizes the process of a master passing on skills and insights
situationally to an apprentice.

\gitem{g:cognitive-load-theory}{Cognitive Load Theory} \emph{Cognitive load} is
the amount of mental effort required to solve a problem.  Cognitive load theory
divides this effort into \emph{intrinsic}, \emph{extraneous}, and
\emph{germane}, and holds that people learn faster and better when extraneous
load is reduced.

\gitem{g:cognitivism}{Cognitivism} A theory of learning that holds that mental
states and processes can and must be included in models of learning. See also
\gref{g:behaviorism}{behaviorism}.

\gitem{g:commons}{Commons} something managed jointly by a community according
to rules they themselves have evolved and adopted.

\gitem{g:community-of-practice}{Community of practice} A self-perpetuating
group of people who share and develop a craft such as knitters, musicians, or
programmers. See also
\gref{g:legitimate-peripheral-participation}{legitimate peripheral participation}.

\gitem{g:community-representation}{Community representation} Using cultural
capital to highlight students' social identities, histories, and community
networks in learning activities.

\gitem{g:computational-integration}{Computational integration} Using computing
to re-implement pre-existing cultural artifacts, e.g., creating variants of
traditional designs using computer drawing tools.

\gitem{g:competent-practitioner}{Competent practitioner} Someone who can do
normal tasks with normal effort under normal circumstances.  See also
\gref{g:novice}{novice} and \gref{g:expert}{expert}.

\gitem{g:computational-thinking}{Computational thinking} Thinking about
problem-solving in ways inspired by programming (though the term is used in many
other ways).

\gitem{g:concept-map}{Concept map} A picture of a mental model in which
concepts are nodes in a graph and relationships are (labelled) arcs.

\gitem{g:connectivism}{Connectivism} A theory of learning holds that knowledge
is distributed, that learning is the process of navigating, growing, and pruning
connections, and which emphasizes the social aspects of learning made possible
by the Internet

\gitem{g:constructivism}{Constructivism} A theory of learning that views
learners as actively constructing knowledge.

\gitem{g:content-knowledge}{Content knowledge} A person's understanding of a
subject. See also
\gref{g:general-pedagogical-knowledge}{general pedagogical knowledge}
and \gref{g:pedagogical-content-knowledge}{pedagogical content knowledge}.

\gitem{g:contributing-student}{Contributing student pedagogy} Having students
produce artifacts to contribute to other students' learning.

\gitem{g:conversational-programmer}{Conversational programmer} Someone who
needs to know enough about computing to have a meaningful conversation with a
programmer, but isn't going to program themselves.

\gitem{g:cs-unplugged}{CS Unplugged} A style of teaching that introduces
computing concepts using non-programming examples and artifacts.

\gitem{g:cs0}{CS0} An introductory college-level course on computing aimed at
non-majors with little or no prior experience of programming.

\gitem{g:cs1}{CS1} An introductory college-level computer science course,
typically one semester long, that focuses on variables, loops, functions, and
other basic mechanics.

\gitem{g:cs2}{CS2} A second college-level computer science course that
typically introduces basic data structures such as stacks, queues, and
dictionaries.

\gitem{g:deficit-model}{Deficit model} The idea that some groups are
under-represented in computing (or some other field) because their members lack
some attribute or quality.

\gitem{g:deliberate-practice}{Deliberate practice} The act of observing
performance of a task while doing it in order to improve ability.

\gitem{g:demonstration-lesson}{Demonstration lesson} A staged lesson in which
one teacher presents material to a class of actual students while other teachers
observe in order to learn new teaching techniques.

\gitem{g:diagnostic-power}{Diagnostic power} The degree to which a wrong answer
to a question or exercise tells the instructor what misconceptions a particular
learner has.

\gitem{g:direct-instruction}{Direct instruction} A teaching method centered
around meticulous curriculum design delivered through prescribed script.

\gitem{g:dunning-kruger-effect}{Dunning-Kruger effect} The tendency of people
who only know a little about a subject to incorrectly estimate their
understanding of it.

\gitem{g:educational-psychology}{Educational psychology} The study of how
people learn. See also \gref{g:instructional-design}{instructional design}.

\gitem{g:ego-depletion}{Ego depletion} The impairment of self control that
occurs when it is exercised intensively or for long periods.

\gitem{g:elevator-pitch}{Elevator pitch} A short description of an idea,
project, product, or person that can be delivered and understood in just a few
seconds.

\gitem{g:end-user-programmer}{End-user programmer} Someone who does not
consider themselves a programmer, but who nevertheless writes and debugs
software, such as an artist creating complex macros for a drawing tool.

\gitem{g:end-user-teacher}{End-user teacher} By analogy with
\gref{g:end-user-programmer}{end-user programmer},
someone who is teaching frequently, but whose primary occupation is not
teaching, who has little or no background in pedagogy, and who may work outside
institutional classrooms.

\gitem{g:expert}{Expert} Someone who can diagnose and handle unusual
situations, knows when the usual rules do not apply, and tends to recognize
solutions rather than reasoning to them. See
also \gref{g:competent-practitioner}{competent practitioner}
and \gref{g:novice}{novice}.

\gitem{g:expert-blind-spot}{Expert blind spot} The inability of experts to
empathize with novices who are encountering concepts or practices for the first
time.

\gitem{g:expertise-reversal}{Expertise reversal effect} The way in which
instruction that is effective for novices becomes ineffective for competent
practitioners or experts.

\gitem{g:externalized-cognition}{Externalized cognition} The use of graphical,
physical, or verbal aids to augment thinking.

\gitem{g:extrinsic-motivation}{Extrinsic motivation} Being driven by external
rewards such as payment or fear of punishment. See
also \gref{g:intrinsic-motivation}{intrinsic motivation}.

\gitem{g:faded-example}{Faded example} A series of examples in which a steadily
increasing number of key steps are blanked out. See
also \gref{g:scaffolding}{scaffolding}.

\gitem{g:false-beginner}{False beginner} Someone who has studied a language
before but is learning it again. False beginners start at the same point as true
beginners (i.e., a pre-test will show the same proficiency) but can move much
more quickly.

\gitem{g:far-transfer}{Far transfer} Transfer of learning or proficiency
between widely-separated domains, e.g., improvement in math skills as a result
of playing chess.

\gitem{g:finks-taxonomy}{Fink's Taxonomy} A six-part non-hierarchical
classification of understanding first proposed in~\cite{Fink2013} whose
categories are \emph{foundational knowledge}, \emph{application},
\emph{integration}, \emph{human dimension}, \emph{caring}, and \emph{learning
  how to learn}. See also \gref{g:blooms-taxonomy}{Bloom's Taxonomy}.

\gitem{g:fixed-mindset}{Fixed mindset} The belief that an ability is innate,
and that failure is due to a lack of some necessary attribute. See also
\gref{g:growth-mindset}{growth mindset}.

\gitem{g:flipped-classroom}{Flipped classroom} One in which learners watch
recorded lessons on their own time, while class time is used to work through
problem sets and answer questions.

\gitem{g:flow}{Flow} The feeling of being fully immersed in an activity;
frequently associated with high productivity.

\gitem{g:fluid-representation}{Fluid representation} The ability to move
quickly between different models of a problem.

\gitem{g:formative-assessment}{Formative assessment} Assessment that takes
place during a lesson in order to give both the learner and the instructor
feedback on actual understanding. See
also \gref{g:summative-assessment}{summative assessment}.

\gitem{g:free-range-learner}{Free-range learner} Someone learning outside an
institutional classrooms with required homework and mandated curriculum. (Those
who use the term occasionally refer to students in classrooms as
``battery-farmed learners'', but we don't, because that would be rude.)

\gitem{g:fuzz-testing}{Fuzz testing} A software testing technique based on
generating and submitting random data.

\gitem{g:general-pedagogical-knowledge}{General pedagogical knowledge} A
person's understanding of the general principles of teaching. See also
\gref{g:content-knowledge}{content knowledge}
and \gref{g:pedagogical-content-knowledge}{pedagogical content knowledge}.

\gitem{g:growth-mindset}{Growth mindset} The belief that ability comes with
practice. See also \gref{g:fixed-mindset}{fixed mindset}.

\gitem{g:guided-notes}{Guided notes} Instructor-prepared notes that cue
students to respond to key information in a lecture or discussion.

\gitem{g:hashing}{Hashing} Generating a condensed pseudo-random digital key
from data; any specific input produces the same output, but different inputs are
highly likely to produce different outputs.

\gitem{g:hypercorrection}{Hypercorrection effect} The more strongly someone
believed that their answer on a test was right, the more likely they are not to
repeat the error once they discover that in fact they were wrong.

\gitem{g:implementation-science}{Implementation science} The study of how to
translate research findings to everyday clinical practice.

\gitem{g:impostor-syndrome}{Impostor syndrome} A feeling of insecurity about
one's accomplishments that manifests as a fear of being exposed as a fraud.

\gitem{g:inclusivity}{Inclusivity} Working actively to include people with
diverse backgrounds and needs.

\gitem{g:inquiry-based-learning}{Inquiry-based learning} The practice of
allowing learners to ask their own questions, set their own goals, and find
their own path through a subject.

\gitem{g:instructional-design}{Instructional design} The craft of creating and
evaluating specific lessons for specific audiences. See also
\gref{g:educational-psychology}{educational psychology}.

\gitem{g:intrinsic-motivation}{Intrinsic motivation} Being driven by enjoyment
of a task or the satisfaction of doing it for its own sake.  See also
\gref{g:extrinsic-motivation}{extrinsic motivation}.

\gitem{g:jugyokenkyu}{Jugyokenkyu} Literally ``lesson study'', a set of
practices that includes having teachers routinely observe one another and
discuss lessons to share knowledge and improve skills.

\gitem{g:learned-helplessness}{Learned helplessness} A situation in which
people who are repeatedly subjected to negative feedback that they have no way
to escape learn not to even try to escape when they could.

\gitem{g:learner-persona}{Learner persona} A brief description of a typical
target learner for a lesson that includes their general background, what they
already know, what they want to do, how the lesson will help them, and any
special needs they might have.

\gitem{g:lms}{Learning management system}(LMS): An application for tracking course
enrollment, exercise submissions, grades, and other bureaucratic aspects of
formal classroom learning.

\gitem{g:learning-objective}{Learning objective} What a lesson is trying to
achieve.

\gitem{g:learning-outcome}{Learning outcome} What a lesson actually achieves.

\gitem{g:legitimate-peripheral-participation}{Legitimate peripheral
  participation} Newcomers' participation in simple, low-risk tasks that a
\gref{g:community-of-practice}{community of practice} recognizes as valid contributions.

\gitem{g:live-coding}{Live coding} The act of teaching programming by writing
software in front of learners as the lesson progresses.

\gitem{g:long-term-memory}{Long-term memory} The part of memory that stores
information for long periods of time. Long-term memory is very large, but
slow. See also \gref{g:short-term-memory}{short-term memory}.

\gitem{g:manual}{Manual} Reference material intended to help someone who
already understands a subject fill in (or remind themselves of) details.

\gitem{g:marketing}{Marketing} The craft of seeing things from other people's
perspective, understanding their wants and needs, and finding ways to meet them

\gitem{g:mooc}{Massive Open Online Course}(MOOC): An online course designed
for massive enrollment and asynchronous study, typically using recorded videos
and automated grading.

\gitem{g:mental-model}{Mental model} A simplified representation of the key
elements and relationships of some problem domain that is good enough to support
problem solving.

\gitem{g:metacognition}{Metacognition} Thinking about thinking.

\gitem{g:minute-cards}{Minute cards} A feedback technique in which learners
spend a minute writing one positive thing about a lesson (e.g., one thing
they've learned) and one negative thing (e.g., a question that still hasn't been
answered).

\gitem{g:near-transfer}{Near transfer} Transfer of learning or proficiency
between closely-related domains, e.g., improvement in understanding of decimals
as a result of doing exercises with fractions.

\gitem{g:notional-machine}{Notional machine} A general, simplified model of how
a particular family of programs executes.

\gitem{g:novice}{Novice} Someone who has not yet built a usable mental model of
a domain. See also \gref{g:competent-practitioner}{competent practitioner}
and \gref{g:expert}{expert}.

\gitem{g:pair-programming}{Pair programming} A software development practice in
which two programmers share one computer. One programmer (the driver) does the
typing, while the other (the navigator) offers comments and suggestions in real
time. Pair programming is often used as a teaching practice in programming
classes.

\gitem{g:parsons-problem}{Parsons Problem} An assessment technique developed by
Dale Parsons and others in which learners rearrange given material to construct
a correct answer to a question.

\gitem{g:passive-learning}{Passive learning} An approach to instruction in which
learners read, listen, or watch without immediately using new knowledge.
Passive learning is less effective than \gref{g:active-learning}{active
learning}.

\gitem{g:passive-teaching}{Passive teaching} An approach to instruction in which
the teacher does not adjust pace or examples, or otherwise act on feedback from
learners, during the lesson.  See also \gref{g:active-teaching}{active
teaching}.

\gitem{g:pedagogical-content-knowledge}{Pedagogical content knowledge} (PCK)
The understanding of how to teach a particular subject, i.e., the best order in
which to introduce topics and what examples to use. See also
\gref{g:content-knowledge}{content knowledge}
and \gref{g:general-pedagogical-knowledge}{general pedagogical knowledge}.

\gitem{g:peer-instruction}{Peer instruction} A teaching method in which an
instructor poses a question and then students commit to a first answer, discuss
answers with their peers, and commit to a (revised) answer.

\gitem{g:persistent-memory}{Persistent memory} see \gref{g:long-term-memory}{long-term memory}.

\gitem{g:personalized-learning}{Personalized learning} Automatically tailoring
lessons to meet the needs of individual students.

\gitem{g:plausible-distractor}{Plausible distractor} A wrong or less-than-best
answer to a multiple-choice question that looks like it could be right. See also
\gref{g:diagnostic-power}{diagnostic power}.

\gitem{g:positioning}{Positioning} What sets one brand apart from other,
similar brands.

\gitem{g:preparatory-privilege}{Preparatory privilege} The advantage of coming
from a background that provides more preparation for a particular learning task
than others.

\gitem{g:productive-failure}{Productive failure} A situation in which learners
are deliberately given problems that can't be solved with the knowledge they
have and have to go out and acquire new information in order to make progress.
See also \gref{g:zpd}{Zone of Proximal Development}.

\gitem{g:pull-request}{Pull request} A set of proposed changes to a GitHub
repository that can be reviewed, updated, and eventually merged.

\gitem{g:read-cover-retrieve}{Read-cover-retrieve} A study practice in which
the learner covers up key facts or terms during a first pass through material,
then checks their recall on a second pass.

\gitem{g:reflective-practice}{Reflective practice} see \gref{g:deliberate-practice}{deliberate practice}.

\gitem{g:scaffolding}{Scaffolding} Extra material provided to early-stage
learners to help them solve problems.

\gitem{g:seo}{Search engine optimization} The process of increasing the
quantity and quality of web site traffic.

\gitem{g:short-term-memory}{Short-term memory} The part of memory that briefly
stores information that can be directly accessed by consciousness.

\gitem{g:situated-learning}{Situated learning} A model of learning that focuses
on people's transition from being newcomers to be accepted members of a
\gref{g:community-of-practice}{community of practice}.

\gitem{g:split-attention-effect}{Split-attention effect} The decrease in
learning that occurs when learners must divide their attention between multiple
concurrent presentations of the same information (e.g., captions and a
voiceover).

\gitem{g:stereotype-threat}{Stereotype threat} A situation in which people feel
that they are at risk of being held to stereotypes of their social group.

\gitem{g:subgoal-labelling}{Subgoal labelling} Giving names to the steps in a
step-by-step description of a problem-solving process.

\gitem{g:summative-assessment}{Summative assessment} Assessment that takes
place at the end of a lesson to tell whether the desired learning has taken
place.

\gitem{g:tangible-artifact}{Tangible artifact} Something a learner can work on
whose state gives feedback about the learner's progress and helps the learner
diagnose mistakes.

\gitem{g:test-driven-development}{Test-driven development} A software
development practice in which programmers write tests first in order to give
themselves concrete goals and clarify their understanding of what ``done'' looks
like.

\gitem{g:think-pair-share}{Think-pair-share} A collaboration method in which
each person thinks individually about a question or problem, then pairs with a
partner to pool ideas, and then have one person from each pair present to the
whole group.

\gitem{g:transfer-appropriate-processing}{Transfer-appropriate processing} The
improvement in recall that occurs when practice uses activities similar to those
used in testing.

\gitem{g:tutorial}{Tutorial} A lesson intended to help someone improve their
general understanding of a subject.

\gitem{g:twitch-coding}{Twitch coding} Having a group of people decide moment
by moment or line by line what to add to a program next.

\gitem{g:working-memory}{Working memory} see \gref{g:short-term-memory}{short-term memory}.

\gitem{g:zpd}{Zone of Proximal Development} (ZPD) Includes the set of problems that
people cannot yet solve on their own but are able to solve with assistance from
a more experienced mentor.  See also \gref{g:productive-failure}{productive
failure}.

\end{description}
