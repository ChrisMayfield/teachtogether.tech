\chaplbl{Why I Teach}{s:finale}

When I started volunteering at the University of Toronto,
some of my students asked me why I would teach for free.
This was my answer:

\begin{quote}

When I was your age,
I thought universities existed to teach people how to learn.
Later,
in grad school,
I thought universities were about doing research and creating new knowledge.
Now that I'm in my forties,
though,
I've realized that what we're really teaching you is
how to take over the world,
because you're going to have to whether you want to or not.

My parents are in their seventies.
They don't run the world any more;
it's people my age who pass laws
and make life-and-death decisions in hospitals.
As scary as it is,
\emph{we} are the grownups.

Twenty years from now,
we'll be heading for retirement and \emph{you} will be in charge.
That may sound like a long time when you're nineteen,
but take three breaths and it's gone.
That's why we give you problems whose answers can't be cribbed from last year's notes.
That's why we put you in situations where
you have to figure out what needs to be done right now,
what can be left for later,
and what you can simply ignore.
It's because if you don't learn how to do these things now,
you won't be ready to do them when you have to.

\end{quote}

It was all true,
but it wasn't the whole story.
I don't want people to make the world a better place so that I can retire in comfort.
I want them to do it because it's the greatest adventure of our time.
A hundred and fifty years ago,
most societies practiced slavery.
A hundred years ago,
my grandmother \hreffoot{https://en.wikipedia.org/wiki/The\_Famous\_Five\_(Canada)}{wasn't legally a person} in Canada.
In the year I was born,
most of the world's people suffered under totalitarian rule,
and judges were still ordering electroshock therapy to ``cure'' homosexuals.
There's still a lot wrong with the world,
but look at how many more choices we have than our grandparents did.
Look at how many more things we can know, and be, and enjoy
because we're finally taking the Golden Rule seriously.

I am less optimistic today than I was then.
Climate change,
mass extinction,
surveillance capitalism,
inequality on a scale we haven't seen in a century,
the re-emergence of racist nationalism:
my generation has watched it all happen and shrugged.
The bill for our cowardice, lethargy, and greed won't come due until my daughter is grown,
but it \emph{will} come,
and by the time it does there will be no easy solutions to these problems
(and possibly no solutions at all).

So this is why I teach today:
I'm angry.
I'm angry because your sex and your color and your parents' wealth and connections
shouldn't count for more than how smart or honest or hard-working you are.
I'm angry because we turned the Internet into a cesspool.
I'm angry because billionaires are playing with rocket ships while the planet is melting
and Nazis are on the march once again.
I'm angry,
so I teach,
because the world only gets better when we teach people how to make it better.

In his 1947 essay ``Why I Write'',
\hreffoot{http://www.resort.com/~prime8/Orwell/whywrite.html}{George Orwell} wrote:

\begin{quote}

  In a peaceful age I might have written ornate or merely descriptive books,
  and might have remained almost unaware of my political loyalties.
  As it is I have been forced into becoming a sort of pamphleteer{\ldots}
  Every line of serious work that I have written since 1936 has been written,
  directly or indirectly,
  against totalitarianism{\ldots}
  It seems to me nonsense,
  in a period like our own,
  to think that one can avoid writing of such subjects.
  Everyone writes of them in one guise or another.
  It is simply a question of which side one takes.

\end{quote}

\noindent
Replace ``writing'' with ``teaching'' and you'll have the reason I do what I do.

\noindent
Thank you for reading.
I hope we can learn something together some day.
Until then:

\begin{center}

Start where you are.\\
Use what you have.\\
Help who you can.

\end{center}
