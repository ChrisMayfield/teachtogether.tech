\chapter{Introduction}\label{s:intro}

Hundreds of grassroots groups have sprung up around the world to teach
programming, web design, robotics, and other skills to
\glossref{g:free-range-learner}{free-range learners} outside
traditional classrooms.  These groups exist so that people don't have
to learn these things on their own, but ironically, their founders and
instructors are often teaching themselves how to teach.

There's a better way. Just as knowing a few basic facts about germs
and nutrition can help you stay healthy, knowing a few things about
psychology, instructional design, inclusivity, and community
organization can help you be a more effective teacher. This book
presents evidence-based practices you can use right now, explains why
we believe they are true, and points you at other resources that will
help you go further.  Its four sections cover:

\begin{itemize}

\item
  how people learn;

\item
  how to design lessons that work;

\item
  how to deliver those lessons; and

\item
  how to grow a community of practice around teaching.

\end{itemize}

\noindent
Throughout, we try to follow our own advice: for example, we start
with ideas that are short, engaging, and actionable in order to
motivate you to read further (\chapref{s:motivation}), include lots of
exercises that can be used to reinforce learning (\chapref{s:models}),
and include the original design for this book in \appref{s:v3} so that
you can see what a lesson design looks like.

\begin{callout}{This Book Belongs to Everyone}

  This book is a community resource.  Parts of it were originally
  created for
  \href{http://carpentries.github.io/instructor-training/}{the
    Software Carpentry instructor training program}, which has been
  run over several hundred times over the past six years, and all of
  it can be freely distributed and re-used under the
  \href{https://creativecommons.org/licenses/by/4.0/}{Creative Commons
    - Attribution 4.0 license}.  Please see {\website} to download a
  digital version or to purchase a printed copy at cost.

  Contributions of all kinds are welcome, from errata and minor
  improvements to entirely new sections and chapters.  All proposed
  contributions will be managed in the same way as edits to Wikipedia
  or patches to open source software, and all contributors will be
  credited for their work each time a new version is released.  Please
  see \appref{s:joining} for details and \secref{s:joining-covenant}
  for our code of conduct.

\end{callout}

\section{Who You Are}\label{s:intro-audience}

\secref{s:process-personas} explains how to figure out who your
learners are.  The four I had in mind when writing this book are all
\glossref{g:end-user-teacher}{end-user teachers}: teaching isn't their
primary occupation, they have little or no background in pedagogy, and
they may work outside institutional classrooms.

\begin{description}

  \item[Emily] trained as a librarian, and now works as a web designer
    and project manager in a small consulting company.  In her spare
    time, she helps run web design classes for women entering tech as
    a second career.  She is now recruiting colleagues to run more
    classes in her area using the lessons that she has created, and
    wants to know how to grow a volunteer teaching organization.

  \item[Moshe] is a professional programmer with two teenage children
    whose school doesn't offer programming classes.  He has
    volunteered to run a monthly after-school programming club, and
    while he frequently gives presentations to colleagues, he has no
    experience designing lessons.  He wants to learn how to build
    effective lessons in collaboration with others, and is interested
    in turning his lessons into a self-paced online course.

  \item[Samira] is an undergraduate in robotics who is thinking about
    becoming a full-time teacher after she graduates.  She wants to
    help teach weekend workshops for undergraduate women, but has
    never taught an entire class before, and feels uncomfortable
    teaching things that she's not an expert in.  She wants to learn
    more about education in general in order to decide if it's for
    her.

  \item[Gene] is a professor of computer science whose research area
    is operating systems.  They have been teaching undergraduate
    classes for six years, and increasingly believe that there has to
    be a better way. The only training available through their
    university's teaching and learning center relates to posting
    assignments and grades in the learning management system, so they
    want to find out what else they ought to be asking for.

\end{description}

\noindent
These people have \emph{a variety of technical backgrounds} and
\emph{some previous teaching experience}, but \emph{no formal training
  in teaching, lesson design, or community organization}.  Most work
with \emph{free-range learners} and are \emph{focused on teenagers and
  adults} rather than children; all \emph{have limited time and
  resources}.

\secref{s:joining-using} describes different ways people have used
this material.  (That discussion is delayed to an appendix because it
refers to some of the ideas introduced later in this book.)  We expect
our made-up learners to use this material as follows:

\begin{description}

  \item[Emily] will take part in a weekly online reading group with
    her volunteers.

  \item[Moshe] will cover part of this book in a two-day weekend
    workshop and study the rest on his own.

  \item[Samira] will use this book in a one-semester undergraduate
    course with assignments, a project, and a final exam.

  \item[Gene] will read the book on their own in their office or
    while commuting, wishing all the while that universities did more
    to support high-quality teaching.

\end{description}

\section{What to Read Instead}\label{s:intro-instead}

If you are in a hurry, or want a taste of what this book will cover,
\cite{Brow2018} presents ten evidence-based tips for teaching
computing.  You can download the paper, or read it online, on
\href{https://doi.org/10.1371/journal.pcbi.1006023}{the PLoS website}.

I also recommend:

\begin{itemize}

  \item \href{http://carpentries.github.io/instructor-training/}{The
    Carpentries instructor training}, for which most of the first half
    of this book was originally developed.

  \item \cite{Lang2016} and \cite{Hust2012}, which are short,
    approachable, and connect things you can do right now to the
    research that backs them.

  \item \cite{Majo2015}, \cite{Broo2016} and \cite{Berg2012}.  The
    first catalogs a hundred different kinds of exercises you can do
    with students; the second describes fifty different ways that
    groups can discuss things productively, and the third is a
    collection of patterns for teaching.  These books can be used on
    their own, but I think they make more sense once Huston or Lang
    have given you a framework for understanding them.

  \item \cite{DeBr2015}, which conveys a lot of what \emph{is} true
    about educational by explaining what \emph{isn't}, and
    \cite{Dida2016}, which grounds learning theory in cognitive
    psychology.

  \item \cite{Pape1993}, which remains an inspiring vision of how
    computers could change education.

  \item \cite{Gree2014}, \cite{McMi2017} and \cite{Watt2014}.  These
    three short books explain why so many attempts at educational
    reform have failed over the past forty years, how for-profit
    colleges are exploiting and exacerbating the growing inequality in
    our society, and how technology has repeatedly failed to
    revolutionize education.

  \item \cite{Guzd2015a}, \cite{Hazz2014}, and \cite{Sent2018}, which
    are academically-oriented books I've found about teaching
    computing.

  \item \cite{Brow2007} and \cite{Mann2015}, because you can't teach
    computing well without changing the system in which we teach, and
    you can't do that on your own.

\end{itemize}

Of these, \cite{Pape1993} is the one that shaped my ideas about
teaching the most.  Papert's central argument is that people don't
absorb knowledge; instead, they (re-)construct it for themselves, and
computers are a new and powerful tool for helping them do that.
\href{https://medium.com/bits-and-behavior/mindstorms-what-did-papert-argue-and-what-does-it-mean-for-learning-and-education-c8324b58aca4}{Andy
  Ko's excellent description} does a better job of summarizing
Papert's ideas than I possibly could, and \cite{Craw2010} is a
thought-provoking companion to both.

\section{History}\label{s:intro-history}

\emph{A lot of my stories aren't true, but this is a true
  story{\ldots}}

When I started teaching people how to program in the late 1980s, I
went too fast, used too much jargon, and had no idea how much my
learners actually understood.  I got better over time, but still felt
like I was stumbling around in a darkened room.

In 2010, I rebooted a project called
\href{http://carpentries.org}{Software Carpentry} that teaches basic
computing skills to researchers.  (The name ``carpentry'' was chosen
to distinguish what we taught from software engineering: we were
trying to show people the digital equivalent of painting a bathroom,
not building the Channel Tunnel.)  In the years that followed, I
discovered resources like \href{http://computinged.wordpress.com}{Mark
  Guzdial's blog} and the book \emph{How Learning Works}
\cite{Ambr2010}. These in turn led me to books like
\cite{Hust2012,Lemo2014,Lang2016} that showed me how to build and
deliver better lessons in less time and with less effort.

I started using these ideas in \href{http://carpentries.org}{Software
  Carpentry} in 2012.  The results were everything I'd hoped for, so I
began running training sessions to pass on what I'd learned.  Those
sessions became a
\href{https://carpentries.github.io/instructor-training/}{training
  program} that dozens of trainers have now taught to over a thousand
people on six continents.  Since then, I have run the course for
people who teach programming to children, librarians, and women
re-entering the workforce or changing careers, and all of those
experiences have gone into this book.

\section{Why Learn to Program?}\label{s:intro-why}

Politicians, business leaders, and educators often say that people
should learn to program because the jobs of the future will require
it; for example, \cite{Scaf2017} found that people who aren't software
developers but who still program make higher wages than comparable
workers who do not.

However, as Benjamin Doxtdator has
\href{http://www.longviewoneducation.org/field-guide-jobs-dont-exist-yet/}{pointed
  out}, many of those claims are built on shaky ground.  Even if they
were true, education shouldn't prepare people for the jobs of the
future: it should give them the power to decide what kinds of jobs
there are, and to ensure that those jobs are worth doing.  And as
\href{https://computinged.wordpress.com/2017/10/18/why-should-we-teach-programming-hint-its-not-to-learn-problem-solving/}{Mark
  Guzdial points out}, there are actually many reasons to learn how to
program:

\begin{enumerate}

\item
  To understand our world.

\item
  To study and understand processes.

\item
  To be able to ask questions about the influences on their lives.

\item
  To use an important new form of literacy.

\item
  To have a new way to learn art, music, science, and mathematics.

\item
  As a job skill.

\item
  To use computers better.

\item
  As a medium in which to learn problem-solving.

\end{enumerate}

\noindent
Part of what motivates me to teach is the hope that if enough people
understand how to make technology work for them, we will be able to
build a society in which \emph{all} of the reasons above are valued
and rewarded (\chapref{s:final}).

\section{Have a Code of Conduct}\label{s:intro-code-of-conduct}

The most important thing I've learned about teaching in the last
thirty years is how important it is for everyone to treat everyone
else with respect, both in and out of class.  If you use this material
in any way, please adopt a Code of Conduct like the one in
\appref{s:conduct} and require everyone who takes part in your classes
to abide by it.

A Code of Conduct can't stop people from being offensive, any more
than laws against theft stop people from stealing.  What it \emph{can}
do is make expectations and consequences clear.  More importantly,
having one tells people that there are rules, and that they can expect
a friendly learning experience.

If someone challenges you about having a Code of Conduct, remind them
that it \emph{isn't} an infringement of free speech.  People have a
right to say what they think, but that doesn't mean they have a right
to say it wherever and whenever they want.  If they want to make
someone feel unwelcome, they can go and find their own space in which
to do it.

\section{Acknowledgments}\label{s:intro-acknowledgments}

This book would not exist without the hard work and feedback of Erin
Becker, Azalee Bostroem, Hugo Bowne-Anderson, Neil Brown, Gerard
Capes, Francis Castro, Warren Code, Ben Cotton, Richie Cotton, Karen
Cranston, Katie Cunningham, Natasha Danas, Matt Davis, Neal Davis,
Mark Degani, Michael Deutsch, Brian Dillingham, Kathi Fisler, Auriel
Fournier, Bob Freeman, Nathan Garrett, Mark Guzdial, Rayna Harris,
Ahmed Hasan, Ian Hawke, Felienne Hermans, Kate Hertweck, Toby Hodges,
Dan Katz, Christina Koch, Shriram Krishnamurthi, Colleen Lewis, Sue
McClatchy, Ian Milligan, Lex Nederbragt, Aleksandra Nenadic, Jeramia
Ory, Joel Ostblom, Elizabeth Patitsas, Aleksandra Pawlik, Sorawee
Porncharoenwase, Emily Porta, Alex Pounds, Thomas Price, Danielle
Quinn, Ian Ragsdale, Erin Robinson, Rosario Robinson, Ariel Rokem, Pat
Schloss, Malvika Sharan, Florian Shkurti, Juha Sorva, Tracy Teal,
Tiffany Timbers, Richard Tomsett, Preston Tunnell Wilson, Matt Turk,
Fiona Tweedie, Allegra Via, Anelda van der Walt, St\'{e}fan van der
Walt, Belinda Weaver, Hadley Wickham, Jason Williams, John Wrenn, and
Andromeda Yelton.  I am grateful to them, and to everyone who has used
this material over the years; any mistakes that remain are mine.

\begin{callout}{Breaking the Law}

  Much of the research reported in this book was publicly funded, but
  despite that, a lot of it is locked away behind paywalls.  At a
  guess, I broke the law roughly 250 times to download papers from
  sites like Sci-Hub. I hope the day is coming when no one will need
  to do that; if you are a researcher, please hasten that day by
  publishing your research in open access venues, or by posting copies
  on open preprint servers.

\end{callout}

\section{Exercises}\label{s:intro-exercises}

Each chapter ends with a variety of exercises that include a suggested
format and an indication of how long they usually take in an in-person
setting.  Most can be used in other formats---in particular, if you
are going through this book on your own, you can still do many of the
exercises that are described as being for groups---and you can always
spend more time on them than what's suggested.

The exercises in this chapter can be used as preassessment questions
(\secref{s:classroom-prior}) rather than as in-class exercises.  if
you have learners answer them a few days before a class or workshop
starts, they will give you a much clearer idea of who they are and how
best you can help them.

\exercise{Highs and Lows}{whole class}{5}

Write brief answers to the following questions and share with your
peers.  (If you are taking notes together online as described in
\secref{s:classroom-notetaking}, put your answers there.)

\begin{enumerate}
\item
  What is the best class or workshop you ever took? What made it so good?
\item
  What was the worst one? What made it so bad?
\end{enumerate}

\exercise{Know Thyself}{whole class}{5}

Write brief answers to the following questions and share them as
described above.  Keep your answers somewhere so that you can refer to
them as you go through the rest of this book.

\begin{enumerate}
\item
  What do you most want to teach?
\item
  Who do you most want to teach?
\item
  Why do you want to teach?
\item
  How will you know if you're teaching well?
\end{enumerate}

\exercise{Starting Points}{individual}{5}

Write brief answers to the following questions and share them as
described above.  Keep your answers somewhere so that you can refer to
them as you go through the rest of this book.

\begin{enumerate}
\item
  What do you most want to learn about teaching and learning?
\item
  What is one specific thing you believe is true about teaching and
  learning?
\end{enumerate}

\exercise{Why Learn to Program?}{individual}{20}

Re-read Guzdial's list of reasons to learn to program in
\secref{s:intro-why}, then draw a $3{\times}3$ grid whose X and Y axes
are labelled ``low'', ``medium'', and ``high'' and place each point in
one sector according to how important it is to you (the X axis) and to
the people you plan to teach (the Y axis).

\begin{enumerate}

\item
  Which points are closely aligned in importance (i.e., on the diagonal
  in your grid)?

\item
  Which points are misaligned (i.e., in the off-diagonal corners)?

\item
  How does this change what you teach?

\end{enumerate}
