\chaplbl{Building a Community of Practice}{s:community}

Finland's schools are among the most successful in the world,
but as Anu Partanen \hreffoot{https://www.theatlantic.com/national/archive/2011/12/what-americans-keep-ignoring-about-finlands-school-success/250564/}{pointed out},
they haven't succeeded in isolation.
Other countries' attempts to adopt Finnish teaching methods are doomed to fail
unless those countries also ensure that children (and their parents) are safe,
well nourished,
and treated fairly by the courts \cite{Sahl2015,Wilk2011}.
This isn't surprising given what we know about the importance of motivation for learning (\chapref{s:motivation}):
everyone does less well if they believe the system is unpredictable, unfair, or indifferent.

You don't have to fix all of society's ills in order to teach programming,
but you \emph{do} you have to understand and be involved in
what happens outside of your class if you want people to learn.
This applies to instructors as well as to learners.
Many of the people who teach after-school classes or weekend workshops
start as volunteers or part-timers
and have to juggle many other commitments.
What happens outside the classroom is as important to their success as teachers
as it is to their learners,
and the best way to help them is to foster a teaching community.

A framework for thinking about teaching communities is \gref{g:situated-learning}{situated learning},
which focuses on how \gref{g:legitimate-peripheral-participation}{legitimate peripheral participation}
leads to people becoming members of
a \gref{g:community-of-practice}{community of practice} \cite{Weng2015}.
Unpacking those terms,
a community of practice is a group of people bound together by interest in some activity,
such as knitting or particle physics.
Legitimate peripheral participation means doing simple, low-risk tasks
that community recognizes as valid contributions:
making your first scarf,
stuffing envelopes during an election campaign,
or proof-reading documentation for open source software.

Situated learning focuses on the transition from being a newcomer
to being accepted as a peer by those who are already community members.
This typically means starting with simplified tasks and tools,
then doing similar tasks with more complex tools,
and finally tackling the exercises of advanced practitioners.
For example,
children learning music may start by playing nursery rhymes on a recorder or ukulele,
then play other simple songs on a trumpet or saxophone in a band,
and finally start exploring their own musical tastes.
Healthy communities of practice understand these progressions
and support them by providing ramps rather than cliffs.
Common methods include:

\begin{description}

\item[Problem solving:]
  ``I'm stuck---can we work on designing this lesson together?''

\item[Requests for information:]
  ``What's the password for the mailing list manager?''

\item[Seeking experience:]
  ``Has anyone had a learner with a reading disability?''

\item[Sharing assets:]
  ``I put together a website for a class last year that you can use as a starting point.''

\item[Coordination:]
  ``Can we combine our t-shirt orders to get a discount?''

\item[Building an argument:]
  ``It will be easier to convince my boss to make changes if I know how other bootcamps do this.''

\item[Documenting projects:]
  ``We've had this problem five times now. Let us write it down once and for all.''

\item[Mapping knowledge:]
  ``What other groups are doing this kind of thing in nearby cities?''

\item[Visits:]
  ``Can we come and see your after-school program? We need to establish one in our city.''

\end{description}

Every community of practice is unique,
but \hreffoot{https://www.feverbee.com/types-of-community-and-activity-within-the-community/}{they share some characteristics}:

\begin{description}

\item[Community of action:]
  people focused on a shared goal,
  such as getting someone elected.

\item[Community of concern:]
  members are brought together by a shared issue,
  such as dealing with a long-term illness.

\item[Community of interest:]
  focused on a shared love of something like backgammon or knitting.

\item[Community of place:]
  people who happen to live or work side by side.

\end{description}

Most communities are mixes of these,
such as people in Toronto who like teaching tech,
and a community's focus can change over time.
For example,
a support group for people dealing with depression (community of concern)
can decide to raise funds to keep a help line going (community of action).
Running the help line can then become the group's focus (community of interest).

\begin{aside}{Soup, Then Hymns}
  Manifestos are fun to write,
  but most people join a volunteer community to help and be helped
  rather than to argue over the wording of a grand vision statement.
  (Most people who prefer the latter are \emph{only} interested in arguing{\ldots})
  To be effective you should therefore focus on things that are immediately useful---on
  what people can create that will be used by other community members right away.
  Once your organization shows that it can actually achieve small things,
  people will be more confident that it's worth investing in bigger ones.
  That's the time to worry about manifestos,
  since that's   the point at which it's important to define the values
  that will guide your members.
\end{aside}

\seclbl{Learn, Then Do}{s:community-learn-then-do}

The first step in building a community is to decide if you should,
or whether you would be more effective joining an existing organization.
Thousands of groups are already teaching people tech skills,
from the \hreffoot{http://www.4-h-canada.ca/}{4-H Club}
and \hreffoot{https://www.frontiercollege.ca/}{literacy programs}
to get-into-coding non-profits like
\hreffoot{http://www.blackgirlscode.com/}{Black Girls Code}
and \hreffoot{http://bridgeschool.io/}{Bridge}.
Joining an existing group will give you a head start on teaching,
an immediate set of colleagues,
and a chance to learn more about how to run things;
hopefully,
learning those skills while making an immediate contribution
will be more important than being able to say that
you're the founder or leader of something new.

Whether you join an existing group or set up one of your own,
you will be more effective if you do a bit of background reading.
\cite{Alin1989,Lake2018} is probably the best-known work on grassroots organizing,
while \cite{Brow2007,Midw2010,Lake2018} are practical manuals rooted in decades of practice.
If you want to read more deeply,
\cite{Adam1975} is a history of the Highlander Folk School,
whose approach has been emulated by many successful groups,
while \cite{Spal2014} is a guide to teaching adults
written by someone with deep personal roots in organizing
and \hreffoot{https://www.nonprofitready.org/}{NonprofitReady.org}
offers free professional development training.

\seclbl{Four Steps}{s:community-four-steps}

Everyone who gets involved with your organization
(including you)
goes through four phases:
recruitment, onboarding, retention, and retirement.
You don't need to worry about this cycle when you're getting started,
but it is worth thinking about
as soon as more than a handful of non-founders are involved.

The first step is recruiting volunteers.
Your marketing should help you with this by making your organization findable
and by making its mission and value clear
to people who might want to get involved
(\chapref{s:marketing}).
Share stories that exemplify the kind of help you want
as well as stories about the people you're helping,
and make it clear that there are many ways to get involved.
(We discuss this in more detail in the next section.)

Your best source of new recruits is your own classes:
``see one, do one, teach one'' has worked well for volunteer organizations
for as long as there have \emph{been} volunteer organizations.
Make sure that every class or other encounter
ends by telling people how they could help and that their help would be welcome.
People who come to you this way will know what you do
and have recent experience of being on the receiving end of what you offer,
which helps your organization avoid collective \gref{g:expert-blind-spot}{expert blind spot}.

\begin{aside}{Start Small}
  As \hreffoot{https://en.wikipedia.org/wiki/Ben\_Franklin\_effect}{Ben Franklin} observed,
  a person who has performed a favor for someone
  is more likely to do them another favor
  than someone who had received a favor from that person.
  Asking people to do something small for you
  is therefore a good step toward getting them to do something larger.
  One natural way to do this when teaching
  is to ask people to submit fixes for your lesson materials for typos or unclear wording,
  or to suggest new exercises or examples.
  If your materials are written in a maintainable way (\secref{s:process-maintainability}),
  this gives them a chance to practice some useful skills
  and gives you an opportunity to start a conversation
  that might lead to a new recruit.
\end{aside}

The middle of the volunteer lifecycle is onboarding and retention,
which we will cover below.
The final step is retirement:
everyone moves on eventually,
and healthy organizations plan for this.
A few simple things can make both the person leaving and everyone who is staying
feel positive about the change:

\begin{description}

\item[Ask people to be explicit about their departure]
  so that everyone knows they've actually left.

\item[Make sure they don't feel embarrassed or ashamed about leaving]
  or about anything else.

\item[Give them an opportunity to pass on their knowledge.]
  For example,
  you can ask them to mentor someone for a few weeks as   their last contribution,
  or to be interviewed by someone who is staying with the organization
  to collect any stories that are worth re-telling.

\item[Make sure they hand over the keys.]
  It's awkward to discover six months after someone has left
  that they're the only person who knows how to book a field for the annual picnic.

\item[Follow up 2--3 months after they leave]
  to see if they have any further thoughts about what worked and what didn't while they were with you,
  or any advice to offer that they either didn't think to give
  or were uncomfortable giving on their way out the door.

\item[Thank them,]
  both when they leave and the next time your group gets together.

\end{description}

\begin{aside}{A Missing Manual}
  Thousands of books have been written on how to start a company.
  Only a handful describe how to end one or leave one gracefully,
  even though there is an ending for every beginning.
  If you ever write one,
  please let me know.
\end{aside}

\seclbl{Onboarding}{s:community-onboarding}

After deciding to become part of a group,
people need to get up to speed.
\cite{Shol2019} summarizes some of what we know about helping people do this.
As we said in \secref{s:classroom-coc},
the first rule is to have and enforce a Code of Conduct.
Organizations should designate someone outside the organization
to receive and review reports of inappropriate behavior,
so that complaints about founders or leaders don't go directly to them.
An independent party offers a degree of objectivity
and can help to protect reporters from hesitating to report out of fear of retribution or damage to their reputation.
Project leaders should also publicize the enforcement decisions
so that the community recognizes that the code is meaningful.

The next most important thing is to be welcoming.
As Fogel said~\cite{Foge2005},
``If a project doesn't make a good first impression,
newcomers may wait a long time before giving it a second chance.''
Other authors have empirically confirmed the importance of kind and polite social environments
in open projects \cite{Sing2012,Stei2013,Stei2018}:

\begin{description}

\item[Post a welcome message]
  on the project's social media pages, Slack channels, forums, or email lists.
  Projects might consider maintaining a dedicated ``Welcome'' channel or list,
  where a project lead or community manager writes a short post asking newcomers to introduce themselves.

\item[Help people find ways to make an initial contribution,]
  such as labelling particular lessons or workshops that need work as ``suitable for newcomers''
  and asking established members not to fix them
  in order to ensure there are suitable places for new arrivals to start work.

\item[Direct the newcomer to project members who have a similar background or skill set]
  so as to demonstrate fit to the newcomer.

\item[Point the newcomer to essential project resources]
  such as the contribution guidelines.

\item[Designate one or two members to serve as a point of contact]
  for each newcomer.
  Doing this can make the newcomer less reluctant to ask questions.

\end{description}

A third rule that helps everyone (not just newcomers)
is to make knowledge findable and keep it up to date.
Newcomers are like explorers who must orient themselves within an unfamiliar landscape \cite{Dage2010}.
Information that is spread out usually makes newcomers feel lost and disoriented.
Given the different possibilities of places to maintain information
(e.g., wikis, files in version control, shared documents, old tweets or Slack messages, and email archives)
it is important to keep information about a specific topic consolidated in a single place
so that newcomers do not need to navigate multiple data sources to find what they need.
Organizing the information make newcomers more confident and oriented \cite{Stei2016}.

Finally,
acknowledge newcomers' first contributions
and figure out where and how they might help in the longer term.
Once they have carried their first contribution over the line,
you and they are likely to have a better sense of what they have to offer
and how the project can help them.
Help newcomers find the next problem they might want to work on
or point them at the next thing they might enjoy reading.
In particular,
encouraging them to help the next wave of newcomers
is both a good way to recognize what they have learned,
and an effective way to pass it on.

\seclbl{Retention}{s:community-retention}

\begin{quote}

  If your people aren't having a ball doing it, there is something very wrong.\\
  --- Saul Alinsky

\end{quote}

Community members shouldn't expect to enjoy every moment of their work with your organization,
but if they don't enjoy any of it,
they won't stick around.
Enjoyment doesn't necessarily mean having an annual party:
people may enjoy cooking,
coaching,
or just working quietly beside others.
There are several things every organization should do to ensure
that people are getting something they value out of their work:

\begin{description}

\item[Ask people what they want rather than guessing.]
  Just as you are not your learners (\secref{s:process-personas}),
  you are probably different from other members of your organization.
  Ask people what they want to do,
  what they're comfortable doing (which may not be the same thing),
  and what constraints there are on their time.
  They might say,``Anything,''
  but even a short conversation will probably uncover the fact that
  they like interacting with people but would rather not be managing the group's finances
  or vice versa.

\item[Provide many ways to contribute.]
  The more ways there are for people to help,
  the more people will be able to.
  Someone who doesn't like standing in front of an audience
  may be able to maintain your organization's website,
  handle its accounts,
  or proof-read lessons.

\item[Recognize contributions.]
  Everyone likes to be appreciated,
  so communities should acknowledge
  their members' contributions both publicly and privately
  by mentioning them in presentations,
  putting them on the website,
  and so on.
  Every hour that someone has given your project
  may be an hour taken away from their personal life or their official employment;
  recognize that fact
  and make it clear that while more hours would be welcome,
  you do not expect them to make unsustainable sacrifices.

\item[Make space.]
  You think you're being helpful,
  but intervening in every decision robs people of their autonomy,
  which in return reduces their motivation (\secref{s:motivation}).
  In particular,
  if you're always the first one to reply to email or chat messages,
  people have less opportunity to grow as members
  and to create horizontal collaborations.
  As a result,
  the community will continue to be centered around one or two individuals
  rather than becoming a highly-connected network
  in which others feel comfortable participating.

\end{description}

Another way to reward participation is to offer training.
Organizations need budgets, grant proposals, and dispute resolution.
Most people are never taught how to do this any more than they are taught how to teach,
so the opportunity to gain transferable skills
is a powerful reason for people to get and stay involved.
If you are going to do this,
don't try to provide the training yourself
unless it's what you specialize in.
Many civic and community groups have programs of this kind,
and you can probably make a deal with one of them.

Finally, 
while volunteers can do a lot,
tasks like system administration and accounting eventually need paid staff.
When you reach this point,
either pay people nothing or pay them a proper wage.
If you pay them nothing,
their real reward is the satisfaction of doing good;
if you pay them a token amount,
on the other hand,
you take that away without giving them the satisfaction of earning a living.

\seclbl{Governance}{s:community-governance}

Every organization has a power structure:
the only question is
whether it's formal and accountable or informal and therefore unaccountable \cite{Free1972}.
The latter actually works pretty well for groups of up to half a dozen people
in which everyone knows everyone else.
Beyond that,
you need rules to spell out
who has the authority to make which decisions
and how to achieve consensus (\secref{s:meetings-marthas-rules}).

The governance model I prefer is a \gref{g:commons}{commons},
which is something managed jointly by a community
according to rules they themselves have evolved and adopted.
As \cite{Boll2014} emphasizes,
all three parts of that definition are essential:
a commons isn't just a shared pasture,
but also includes the community that shares it
and the rules they use to do so.

For-profit corporations and incorporated non-profits are more popular models;
the mechanics vary from jurisdiction to jurisdiction,
so you should seek advice before choosing\footnote{This is one of the times when
  having ties with local government or other like-minded organizations pays off.}.
Both kinds of organization vest ultimate authority in their board.
Broadly speaking, this is either a \emph{service board}
whose members also take on other roles in the organization
or a \emph{governance board} whose primary responsibility is to hire, monitor,
and if need be fire the director.
Board members can be elected by the community or appointed;
in either case,
it's important to prioritize competence over passion
(the latter being more important for the rank and file)
and to try to recruit for particular skills such as accounting, marketing, and so on.

\begin{aside}{Choose Democracy}
  When the time comes,
  make your organization a democracy:
  sooner or later (usually sooner),
  every appointed board turns into a mutual agreement society.
  Giving your members power is messy,
  but is the only way invented so far to ensure that
  organizations continue to meet people's actual needs.
\end{aside}

\seclbl{Look After Yourself}{s:community-care}

Burnout is a chronic risk in any community activity \cite{Pign2016},
so learn to say no more often than you say yes.
If you don't take care of yourself,
you won't be able to take care of your community.

One way to make your ``no'' stick
is to write a to-don't list of things that would be worth doing
but which you \emph{aren't} going to do.
At the time of writing,
mine includes four books,
two software projects,
redesign of my personal website,
and learning to play the penny whistle.

Finally,
remind yourself every now and then that
every organization eventually needs fresh ideas and leadership.
When that time comes,
train your successors and move on as gracefully as you can.
They will undoubtedly do things you wouldn't have,
but few things in life are as satisfying as
watching something you helped build take on a life of its own.
Celebrate that---you won't have any trouble finding something else to keep you busy.

\seclbl{Exercises}{s:community-exercises}

Several of these exercises are taken from \cite{Brow2007}.

\exercise{What Kind of Community?}{individual}{15}

Re-read the description of the four types of communities
and decide which one(s) your group is or aspires to be.

\exercise{People You May Meet}{small groups}{30}

As an organizer,
part of your job is sometimes to help people find a way to contribute despite themselves.
In small groups,
pick three of the people below
and discuss how you would help them become a better contributor to your organization.

\begin{description}

\item[Anna]
  knows more about every subject than everyone else put together---at least,
  she thinks she does.
  No matter what you say,
  she'll correct you;
  no matter what you know, she knows better.

\item[Catherine]
  has so little confidence in her own ability
  that she won't make any decision,
  no matter how small,
  until she has checked with someone else.

\item[Frank]
  enjoys knowing things that other people don't.
  He can work miracles,
  but when asked how he did it,
  he'll grin and say,
  ``Oh, I'm sure you can figure it out.''

\item[Hediyeh]
  is quiet.
  She never speaks up in meetings,
  even when she knows that what other people are saying is wrong.
  She might contribute to the mailing list,
  but she's very sensitive to criticism
  and will always back down rather than defending her point of view.

\item[Kenny]
  has discovered that most people would rather shoulder his share of the work
  than complain about him,
  and he takes advantage of it at every turn.
  The frustrating thing is that he's so damn \emph{plausible}
  when someone finally does confront him.
  ``There have been mistakes on all sides,''
  he says,
  or, ``Well, I think you're nit-picking.''

\item[Melissa]
  means well,
  but somehow something always comes up,
  and her tasks are never finished until the last possible moment.
  Of course,
  that means that everyone who is depending on her can't do their work
  until \emph{after} the last possible moment{\ldots}

\item[Raj]
  is rude.
  ``It's just the way I talk,'' he says.
  ``If you can't hack it, maybe you should find another group to work with.''
  His favorite phrase is, ``That's stupid,''
  and he uses obscenity in every second sentence.

\end{description}

\exercise{Values}{small groups}{45}

Answer the following questions on your own
and then compare your answers to those given by other members of your group.

\begin{enumerate}

\item
  What are the values your organization expresses?

\item
  Are these the values you want the organization to express?

\item
  If not, what values would you like it to express?

\item
  What are specific behaviors that demonstrate those values?

\item
  What behaviors would demonstrate the opposite of those values?
\end{enumerate}

\exercise{Meeting Procedures}{small groups}{30}

Answer the following questions on your own
and then compare your answers to those given by other members of your group.

\begin{enumerate}

\item
  How are your meetings run?

\item
  Is this how you want your meetings to be run?

\item
  Are the rules for running meetings explicit or just assumed?

\item
  Are these the rules you want?

\item
  Who is eligible to vote or make decisions?

\item
  Is this who you want to be vested with decision-making authority?

\item
  Do you follow majority rule,
  make decisions by consensus,
  or use some other method?

\item
  Is this the way you want to make decisions?

\item
  How do people in a meeting know when a decision has been made?

\item
  How do people who weren't at a meeting know what decisions were made?

\item
  Is this working for your group?

\end{enumerate}

\exercise{Size}{small groups}{20}

Answer the following questions on your own
and then compare your answers to those given by other members of your group.

\begin{enumerate}

\item
  How big is your group?

\item
  Is this the size you want for your organization?

\item
  If not, what size would you like it to be?

\item
  Do you have any limits on the size of membership?

\item
  Would you benefit from setting such a limit?

\end{enumerate}

\exercise{Staffing}{small groups}{30}

Answer the following questions on your own
and then compare your answers to those given by other members of your group.

\begin{enumerate}

\item
  Do you have paid staff in your organization
  or is everyone a volunteer?

\item
  Should you have paid staff?

\item
  Do you want/need more or less staff?

\item
  What do the staff members do?

\item
  Are these the primary roles and functions that you need staff to fill?

\item
  Who supervises your staff?

\item
  Is this the supervision process that you want for your group?

\item
  What is your staff paid?

\item
  Is this the right salary to get the needed work done?

\end{enumerate}

\exercise{Money}{small groups}{30}

Answer the following questions on your own
and then compare your answers to those given by other members of your group.

\begin{enumerate}

\item
  Who pays for what?

\item
  Is this who you want to be paying?

\item
  Where do you get your money?

\item
  Is this how you want to get your money?

\item
  If not, do you have any plans to get it another way?

\item
  If so, what are they?

\item
  Who is following up to make sure that happens?

\item
  How much money do you have?

\item
  How much do you need?

\item
  What do you spend most of your money on?

\item
  Is this how you want to spend your money?

\end{enumerate}

\exercise{Becoming a Member}{small groups}{45}

Answer the following questions on your own
and then compare your answers to those given by other members of your group.

\begin{enumerate}

\item
  How does someone join your group?

\item
  How well does this process work?

\item
  Are there membership dues?

\item
  Are people required to agree to any rules of behavior upon joining?

\item
  Are these the rules for behavior you want?

\item
  How does a newcomer find out what needs to be done?

\item
  How well does this process work?
  
\end{enumerate}

\exercise{Borrowing Ideas}{whole class}{15}

Many of my ideas about how to build a community
have been shaped by my experience in open source software development.
\cite{Foge2005} (which is \hreffoot{http://producingoss.com/}{available online})
is a good guide to what has and hasn't worked for those communities,
and the \hreffoot{https://opensource.guide/}{Open Source Guides site}
has a wealth of useful information as well.
Choose one section of the latter,
such as ``Finding Users for Your Project''
or ``Leadership and Governance'',
and give a two-minute presentation to the group of one idea from it
that you found useful or that you strongly disagreed with.

\exercise{Who Are You?}{small groups}{20}

The National Oceanic and Atmospheric Administration (NOAA)
has a short, useful, and amusing guide to
\hreffoot{https://coast.noaa.gov/ddb/story\_html5.html}{dealing with disruptive behaviors}.
It categorizes those behaviors under labels like ``talkative'', ``indecisive'', and ``shy'',
and outlines strategies for handling each.
In groups of 3--6,
read the guide and decide which of these descriptions best fits you.
Do you think the strategies described for handling people like you are effective?
Are other strategies equally or more effective?
