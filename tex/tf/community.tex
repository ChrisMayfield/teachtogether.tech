\chapter{Building a Community of Practice}\label{s:community}

Finland's schools are among the most successful in the world,
but as Anu Partanen \hreffoot{https://www.theatlantic.com/national/archive/2011/12/what-americans-keep-ignoring-about-finlands-school-success/250564/}{pointed out},
they haven't succeeded in isolation.
Other countries' attempts to adopt Finnish teaching methods are doomed to fail
unless those countries also ensure that children (and their parents) are safe,
well nourished,
and treated fairly by the courts.
It's a matter of motivation (Chapter~\cite{s:motivation}):
everone does less well
if they believe the system is unpredictable, unfair, or indifferent \cite{Sahl2015,Wilk2011}.
You don't have to fix all of society's ills in order to teach programming,
but you \emph{do} you have to understand and be involved in
what happens outside of your class if you want people to learn.

All of this is true for instructors as well.
Many of the instructors in after-school classes or weekend workshops
start as volunteers or part-timers
and have to juggle many other commitments.
What happens outside the classroom is as important to their success
as it is to their learners,
and the best way to help them is to foster a teaching community.

A framework in which to think about teaching communities is \gref{g:situated-learning}{situated learning},
which focuses on how \gref{g:legitimate-peripheral-participation}{legitimate peripheral participation}
leads to people becoming members of
a \gref{g:community-of-practice}{community of practice} \cite{Weng2015}.
Unpacking those terms,
a community of practice is a group of people bound together by interest in some activity,
such as knitting or particle physics.
Legitimate peripheral participation means doing simple, low-risk tasks
that community recognizes as valid contributions:
making your first scarf,
stuffing envelopes during an election campaign,
or proof-reading documentation for open source software.

Situated learning focuses on the transition from being a newcomer
to being accepted as a peer by those who are already community members.
This typically means starting with simplified tasks and tools,
then doing similar tasks with more complex tools,
and finally tackling the exercises of advanced practitioners.
For example,
children learning music may start by playing nursery rhymes on a recorder or ukulele,
then play other simple songs on a trumpet or saxophone in a band,
and finally start exploring their own musical tastes.
Healthy communities of practice understand these progressions
and support them by providing ramps rather than cliffs.
Some of the ways they do this include:

\begin{description}

\item[Problem solving:]
  ``I'm stuck---can we work on designing this lesson together?''

\item[Requests for information:]
  ``What's the password for the mailing list manager?''

\item[Seeking experience:]
  ``Has anyone had a learner with a reading disability?''

\item[Reusing assets:]
  ``I put together a website for a class last year that you can use as a starting point.''

\item[Coordination and synergy:]
  ``Can we combine our t-shirt orders to get a discount?''

\item[Building an argument:]
  ``It will be easier to convince my boss to make changes if I know how other bootcamps do this.''

\item[Growing confidence:]
  ``That's a really creative exercise you made---I'm going to use it in my next class.''

\item[Discussing developments:]
  ``What do you think of the new registration system?''

\item[Documenting projects:]
  ``We've had this problem five times now. Let us write it down once and for all.''

\item[Mapping knowledge:]
  ``What other groups are doing this kind of thing in nearby cities?''

\item[Visits:]
  ``Can we come and see your after-school program? We need to establish one in our city.''

\end{description}

Situated learning emphasizes that learning is a social activity.
In order to be effective and sustainable,
teaching therefore needs to be a social activity
rooted in \hreffoot{https://www.feverbee.com/types-of-community-and-activity-within-the-community/}{a community of some kind}:

\begin{description}

\item[Community of action:]
  people focused on a shared goal,
  such as getting someone elected.

\item[Community of concern:]
  members are brought together by a shared issue,
  such as dealing with a long-term illness.

\item[Community of interest:]
  focused on a shared love of something like backgammon or knitting.

\item[Community of place:]
  people who happen to live or work side by side.

\end{description}

Most real communities are mixes of these,
such as people in Toronto who like teaching tech,
and a community's focus can change over time.
For example,
a support group for people dealing with depression (community of concern)
can decide to raise funds to keep a help line going (community of action).
Running the help line can then become the group's focus (community of interest).

\section{Learn, Then Do}\label{s:community-learn-then-do}

The first step in building a community is to decide if you really need to,
or whether you would be more effective joining an existing organization.
Thousands of groups are already teaching people tech skills,
from the \hreffoot{http://www.4-h-canada.ca/}{4-H Club}
and \hreffoot{https://www.frontiercollege.ca/}{literacy programs}
to get-into-coding non-profits like
\hreffoot{http://www.blackgirlscode.com/}{Black Girls Code}
and \hreffoot{http://bridgeschool.io/}{Bridge}.
Joining an existing group will give you a head start on teaching,
an immediate set of colleagues,
and a chance to learn more about how to run things;
hopefully,
learning those skills while making an immediate contribution
will be more important than being able to say that
you're the founder or leader of something new.

Whether you join an existing group or set up one of your own,
you owe it to yourself and everyone who's going to work with you
to find out what's been done before.
People have been writing about grassroots organizing for decades;
\cite{Alin1989} is probably the best-known work on the subject,
while \cite{Brow2007,Midw2010} are practical manuals rooted in decades of practice.
If you want to read more deeply,
\cite{Adam1975} is a history of the Highlander Folk School,
whose approach has been emulated by many successful groups,
while \cite{Spal2014} is a guide to teaching adults
written by someone with deep personal roots in organizing
and \hreffoot{https://www.nonprofitready.org/}{NonprofitReady.org}
offers free professional development training.

\section{Three Steps}\label{s:community-three-steps}

Everyone who gets involved with your organization
(including you)
goes through three phases:
recruitment, retention, and retirement.
You don't need to worry about this cycle when you're getting started,
but it is worth thinking about
as soon as more than a handful of non-founders are involved.

The first step is recruiting volunteers.
Your marketing should help you with this by making your organization findable
and by making its mission and value clear
to people who might want to get involved
(Chapter~\ref{s:marketing}).
Share stories that exemplify the kind of help you want
as well as stories about the people you're helping,
and make it clear that there are many ways to get involved.
(We discuss this in more detail in the next section.)

Your best source of new recruits is your own classes:
``see one, do one, teach one'' has worked well for volunteer organizations
for as long as there have \emph{been} volunteer organizations.
Make sure that every class or other encounter
ends by telling people how they could help and that their help would be welcome.
People who come to you this way will know what you do
and have recent experience of being on the receiving end of what you offer,
which helps your organization avoid collective \gref{g:expert-blind-spot}{expert blind spot}.

\begin{aside}{Start Small}
  As \hreffoot{https://en.wikipedia.org/wiki/Ben\_Franklin\_effect}{Ben Franklin} observed,
  a person who has performed a favor for someone
  is more likely to do them another favor
  than someone who had received a favor from that person.
  Asking people to do something small for you
  is therefore a good step toward getting them to do something larger.
  One natural way to do this when teaching
  is to ask people to submit fixes for your lesson materials for typos or unclear wording,
  or to suggest new exercises or examples.
  If your materials are written in a maintainable way (Section~\ref{s:process-maintainability}),
  this gives them a chance to practice some useful skills
  and gives you an opportunity to start a conversation
  that might lead to a new recruit.
\end{aside}

Recruiting doesn't end when someone first shows up: if you don't follow
through, people will come out once or twice, then decide that what
you're doing isn't for them and disappear. One thing you can do to get
newcomers over this initial hump is to have them take part in group
activities before they do anything on their own, both so that they get a
sense of how your organization does things, and so that they build
social ties that will keep them involved.

Another thing you can do is give newcomers a mentor, and make sure the
mentors actually do some proactive mentoring. The most important things
a mentor can do are make introductions and explain the unwritten rules,
so make it clear to mentors that these are their primary
responsibilities, and they are to report back to you every few weeks to
tell you what they've done.

The second part of the volunteer lifecycle is retention, which is a
large enough topic to deserve a long discussion in
Section~\ref{s:community-retention}. The third and final part is
retirement. Sooner or later, everyone moves on (including you). When
this happens:

\begin{description}
\item[Ask people to be explicit about their departure]
so that everyone knows they've actually left.
\item[Make sure they don't feel embarrassed or ashamed about leaving]
or about anything else.
\item[Give them an opportunity to pass on their knowledge.]
For example, you can ask them to mentor someone for a few weeks as
their last contribution, or to be interviewed by someone who's
staying with the organization to collect any stories that are worth
re-telling.
\item[Make sure they hand over the keys.]
It's awkward to discover six months after someone has left that
they're the only person who knows how to book a playing field for
the annual softball game.
\item[Follow up 2--3 months after they leave]
to see if they have any further thoughts about what worked and what
didn't while they were with you, or any advice to offer that they
either didn't think to give or were uncomfortable giving on their
way out the door.
\item[Thank them,]
both when they leave and the next time your group gets together.
\end{description}

\section{Retention}\label{s:community-retention}

Saul Alinsky once said, ``If your people aren't having a ball doing it,
there is something very wrong.'' \cite{Alin1989} Community members
shouldn't expect to enjoy every moment of their work with your
organization, but if they don't enjoy any of it, they won't stay.

Enjoyment doesn't necessarily mean having an annual party: people may
enjoy cooking, coaching, or just working quietly beside others. There
are several things every organization should do to ensure that people
are getting something they value out of their work:

\begin{description}
\item[Ask people what they want rather than guessing.]
Just as you are not your learners (Section~\ref{s:process-personas}),
you are probably different from other members of your organization.
Ask people what they want to do, what they're comfortable doing
(which may not be the same thing), what constraints there are on
their time, and so on. They might start by saying, ``I don't
know---anything!'' but even a short conversation will probably
uncover the fact that they like interacting with people but would
rather not be managing the group's finances, or vice versa.
\item[Provide many ways to contribute.]
The more ways there are for people to help, the more people will be
able to help. Someone who doesn't like standing in front of an
audience may be able to maintain your organization's website or
handle its accounts; someone who doesn't know how to do anything
else may be able to proof-read lessons, and so on. The more kinds of
tasks you do yourself, the fewer opportunities there are for others
to get involved.
\item[Recognize contributions.]
Everyone likes to be appreciated, so communities should acknowledge
their members' contributions both publicly and privately by
mentioning them in presentations, putting them on the website, and
so on.
\item[Make space.]
Micromanaging or trying to control everything centrally means people
won't feel they have the autonomy to act, which will probably cause
them to drift away. In particular, if you're too engaged or too
quick on the reply button, people have less opportunity to grow as
members and to create horizontal collaborations. As a result, the
community will continue to be focused around one or two individuals,
rather than a highly-connected network in which others feel
comfortable participating.
\end{description}

Another way to make participation rewarding is to provide training.
Organizations require committees, meetings, budgets, grant proposals,
and dispute resolution; most people are never taught how to do any of
this, any more than they are taught how to teach, but training people to
do these things helps your organization run more smoothly, and the
opportunity to gain transferable skills is a powerful reason for people
to get and stay involved. If you are going to do this, don't try to
provide the training yourself (unless it's what you specialize in). Many
civic and community groups have programs of this kind, and you can
probably make a deal with one of them.

Other groups may be useful in other ways as well, and you may be useful
to them---if not immediately, then tomorrow or next year. You should
therefore set aside an hour or two every month to find allies and
maintain your relationships with them. One way to do this is to ask them
for advice: how do they think you ought to raise awareness of what
you're doing? Where have they found space to run classes? What needs
do they think aren't being met, and would you be able to meet them
(either on your own, or in partnership with them)? Any group that has
been around for a few years will have useful advice; they will also be
flattered to be asked, and will know who you are the next time you call.

\begin{aside}{Government Matters}
  It's fashionable in tech circles to disparage government institutions
  as slow-moving dinosaurs, but in my experience they are no worse than
  companies of similar size. Your local school board, library, and your
  city councillor's office may be able to offer space, funding,
  publicity, connections with other groups that you may not have met
  yet, help with red tape, and a host of other useful things.
\end{aside}

\begin{aside}{Soup, Then Hymns}
  Manifestos are fun to write, but most people join a volunteer
  community to help and be helped rather than to argue over the wording
  of a grand vision statement. (Most people who prefer the latter are
  \emph{only} interested in arguing{\ldots}) To be effective you should
  therefore focus on things that are immediately useful, e.g., on what
  people can create that will be used by other community members right
  away. Once your organization shows that it can actually achieve small
  things, people will be more confident that it's worth investing in
  bigger ones. That's the time to worry about manifestos, since that's
  the point at which it's important to define values that will guide
  your growth and operations.
\end{aside}

One important special case of making things rewarding is to pay people.
Volunteers can do a lot, but eventually tasks like system administration
and accounting need full-time paid staff. When this time comes, you
should either pay people nothing or pay them a proper wage, but not do
anything in between. If you pay them nothing, their actual reward for
their work is the satisfaction of doing good. If you pay them a token
amount, you take that away without giving them the satisfaction of
earning a living.

\section{Impostor Syndrome Revisited}\label{s:community-impostor-syndrome}

Impostor syndrome thrives in communities with arbitrary, unnecessary
standards, where harsh criticism is the norm, and where secrecy
surrounds the actual process of getting work done, so the Ada
Initiative has \hreffoot{https://www.usenix.org/blog/impostor-syndrome-proof-yourself-and-your-community}{guidelines} for communities to go with
those given in Section~\ref{s:motivation-demotivation} for individuals:

\begin{description}
\item[Encourage people.]
This is as simple as it is effective.
\item[Discourage hostility and bickering.]
Public, hostile, personal arguments are a natural breeding ground
for impostor syndrome.
\item[Eliminate hidden barriers to participation.]
Be explicit about welcoming new students and colleagues, and
thoroughly document how someone can participate in projects and
events in your research group and at your institution.
\item[As a leader, show your own uncertainties]
and demonstrate your own learning process. When people see leaders
whom they respect struggling or admitting they didn't already know
everything when they started, having realistic opinions of their own
work becomes easier.
\item[Reward and encourage people for mentoring newcomers.]
Officially enshrine mentoring as an important criterion in your
career advancement process.
\item[Don't make it personal when someone's work isn't up to snuff.]
When enforcing necessary quality standards, don't make the issue
about the person. They aren't wrong or stupid or a waste of space;
they've simply done one piece of work that didn't meet your
expectations.
\end{description}

\section{Governance}\label{s:community-governance}

As \cite{Free1972} pointed out, every organization has a power
structure: the only question is whether it's formal and accountable, or
informal and unaccountable. Make yours one of the first kind: write and
publish the rules governing everything from who's allowed to use the
name and logo to who gets to decide whether people are allowed to charge
money to teach with whatever materials your group has worked up.

Organizations can govern themselves in many different ways, and a full
discussion of the options is outside the scope of this book. For-profit
corporations and incorporated non-profits are the two most popular models;
the mechanics vary from jurisdiction to jurisdiction, so you should seek
advice locally before doing anything. (This is one of the times when
having ties with local government or other like-minded organizations
pays off.)

The model I prefer is that of a commons, which is ``something managed
jointly by a community according to rules they themselves have evolved
and adopted''. As \cite{Boll2014} emphasizes, all three parts of that
definition are essential: a commons isn't just a shared pasture, but
also includes the community that shares it and the rules they use to do
so.

Most resources, throughout most of human history, have been commons: it
is only in the last few hundred years that impersonal markets have
pushed them to the margins. In order to do so, free-market advocates
have had to convince us we're something we're not (dispassionate
calculators of individual advantage) and erase or devalue local
knowledge and custom with tragic consequences for us individually and
collectively.

Since society has difficulty recognizing commons organizations, and
since most of the people you will want to recruit don't have experience
with them, you will probably wind up having some sort of board, a
director, and other staff. Broadly speaking, your organization can have
either a \emph{service board}, whose members also take on other roles in the
organization, or a \emph{governance board} whose primary responsibility is to
hire, monitor, and if need be fire the director. Board members can be
elected by the community or appointed; in either case, it's important to
prioritize competence over passion (the latter being more important for
the rank and file), and to try to recruit for particular skills such as
accounting, marketing, and so on.

Don't worry about drafting a constitution when you first get started: it
will only result in endless wrangling about what we're going to do
rather than formalization of what you're already doing. When the time
does come to formalize your rules, though, make your organization a
democracy: sooner or later (usually sooner), every appointed board turns
into a mutual agreement society and loses sight of what the community
it's meant to serve actually needs. Giving the community power is
messy, but is the only way invented so far to ensure that an
organization continues to meet people's actual needs.

\section{Final Thoughts}\label{s:community-final}

As \cite{Pign2016} discusses, burnout is a chronic risk in any
community activity. If you don't take care of yourself, you won't be
able to take care of your community.

Every organization eventually needs fresh ideas and fresh leadership.
When that time comes, train your successors and then move on. They will
undoubtedly do things you wouldn't have, but the same is true of every
generation. Few things in life are as satisfying as watching something
you helped build take on a life of its own. Celebrate that---you won't
have any trouble finding something else to keep you busy.

FIXME

Many well-intentioned people want the world to be better
but don't want anything important to change.
A lot of grassroots efforts to teach programming fall into this category:
they want to teach children and adults how to program so that they can get better jobs,
but shy away from trying to change the system
that has shut them out of decent jobs in the past
or 
what kinds of jobs there are going to be and whether they will be worth doing.

\section{Exercises}\label{s:community-exercises}

Several of these exercises are taken from \cite{Brow2007}, which is
an exceptionally useful book on building community organizations.

\subsection*{What Kind of Community? (individual/15)}

Re-read the discussion in the introduction of types of communities and
decide which type or types your group is, or aspires to be.

\subsection*{People You May Meet (small groups/30)}

As an organizer, part of your job is sometimes to help people find a way
to contribute despite themselves. In small groups, pick three of the
people below and discuss how you would help them become a better
contributor to your organization.

\begin{description}
\item[Anna]
knows more about every subject than everyone else put together---at
least, she thinks she does. No matter what you say, she'll correct
you; no matter what you know, she knows better.
\item[Catherine]
has so little confidence in her own ability that she won't make any
decision, no matter how small, until she has checked with someone
else.
\item[Frank]
believes that knowledge is power, and enjoys knowing things that
other people don't. He can make things work, but when asked how he
did it, he'll grin and say, ``Oh, I'm sure you can figure it out.''
\item[Hediyeh]
is quiet. She never speaks up in meetings, even when she knows that
what other people are saying is wrong. She might contribute to the
mailing list, but she's very sensitive to criticism, and will always
back down rather than defending her point of view.
\item[Kenny]
has discovered that most people would rather shoulder his share of
the work than complain about him, and he takes advantage of it at
every turn. The frustrating thing is that he's so damn \emph{plausible}
when someone finally does confront him. ``There have been mistakes on
all sides,'' he says, or, ``Well, I think you're nit-picking.''
\item[Melissa]
means well, but somehow something always comes up, and her tasks are
never finished until the last possible moment. Of course, that means
that everyone who is depending on her can't do their work until
\emph{after} the last possible moment{\ldots}
\item[Raj]
is rude. ``It's just the way I talk,'' he says, ``If you can't hack it,
maybe you should find another team.'' His favorite phrase is, ``That's
stupid,'' and he uses obscenity in every second sentence.
\end{description}

\subsection*{Values (small groups/45)}

Answer the following questions on your own, and then compare your
answers to those given by other members of your group.

\begin{enumerate}
\item
  What are the values your organization expresses?
\item
  Are these the values you want the organization to express?
\item
  If not, what values would you like it to express?
\item
  What are the specific behaviors that demonstrate those values?
\item
  What are some key behaviors that would demonstrate the values you
  would like for your group?
\item
  What are the behaviors that would demonstrate the opposite of those
  values?
\item
  What are some key behaviors that would demonstrate the opposite of
  the values you want to have?
\end{enumerate}

\subsection*{Meeting Procedures (small groups/30)}

Answer the following questions on your own, and then compare your
answers to those given by other members of your group.

\begin{enumerate}
\item
  How are your meetings run?
\item
  Is this how you want your meetings to be run?
\item
  Are the rules for running meetings explicit or just assumed?
\item
  Are these the rules you want?
\item
  Who is eligible to vote/make decisions?
\item
  Is this who you want to be vested with decision-making authority?
\item
  Do you use majority rule, make decisions by consensus, or use some
  other method?
\item
  Is this the way you want to make decisions?
\item
  How do people in a meeting know when a decision has been made?
\item
  How do people who weren't at a meeting know what decisions were
  made?
\item
  Is this working for your group?
\end{enumerate}

\subsection*{Size (small groups/20)}

Answer the following questions on your own, and then compare your
answers to those given by other members of your group.

\begin{enumerate}
\item
  How big is your group?
\item
  Is this the size you want for your organization?
\item
  If not, what size would you like it to be?
\item
  Do you have any limits on the size of membership?
\item
  Would you benefit from setting such a limit?
\end{enumerate}

\subsection*{Staffing (small groups/30)}

Answer the following questions on your own, and then compare your
answers to those given by other members of your group.

\begin{enumerate}
\item
  Do you have paid staff in your organization?
\item
  Or is it all-volunteer?
\item
  Should you have paid staff?
\item
  Do you want/need more or less staff?
\item
  What do you call the staff (e.g., organizer, director, coordinator,
  etc.)?
\item
  What do the staff members do?
\item
  Are these the primary roles and functions that you want the staff to
  be filling?
\item
  Who supervises your staff?
\item
  Is this the supervision process and responsibility chain that you
  want for your group?
\item
  What is your staff paid?
\item
  Is this the right salary to get the needed work done and to fit
  within your resource constraints?
\item
  What benefits does your group provide to its staff (health, dental,
  pension, short and long-term disability, vacation, comp time, etc.)?
\item
  Are these the benefits that you want to give?
\end{enumerate}

\subsection*{Money (small groups/30)}

Answer the following questions on your own, and then compare your
answers to those given by other members of your group.

\begin{enumerate}
\item
  Who pays for what?
\item
  Is this who you want to be paying?
\item
  Where do you get your money?
\item
  Is this how you want to get your money?
\item
  If not, do you have any plans to get it another way?
\item
  If so, what are they?
\item
  Who is following up to make sure that happens?
\item
  How much money do you have?
\item
  How much do you need?
\item
  What do you spend most of your money on?
\item
  Is this how you want to spend your money?
\end{enumerate}

\subsection*{Becoming a Member (small groups/45)}

Answer the following questions on your own, and then compare your
answers to those given by other members of your group.

\begin{enumerate}
\item
  How does someone join?
\item
  Does this process work for your organization?
\item
  What are the membership criteria?
\item
  Are these the membership criteria you want?
\item
  Are people required to agree to any rules of behavior upon joining?
\item
  Are these the rules for behavior you want?
\item
  Are there membership dues?
\end{enumerate}

\subsection*{Borrowing Ideas (whole class/15)}

Many of our ideas about how to build a community have been shaped by
our experience of working in open source software development.
\cite{Foge2005} (which is \hreffoot{http://producingoss.com/}{available online}) is a
good guide to what has and hasn't worked for those communities, and
the \hreffoot{https://opensource.guide/}{Open Source Guides site} has a wealth of
useful information as well. Choose one section of the latter, such as
``Finding Users for Your Project'' or ``Leadership and Governance'', read
it through, and give a two-minute presentation to the group of one
idea from it that you found useful or that you strongly disagreed
with.

\subsection*{Who Are You? (small groups/20)}

The National Oceanic and Atmospheric Administration (NOAA) has
published a short, amusing, and above all useful guide to \hreffoot{https://coast.noaa.gov/ddb/story\_html5.html}{dealing
with disruptive behaviors}. It categorizes those
behaviors under labels like ``talkative'', ``indecisive'', and ``shy'', and
outlines strategies for handling each. In groups of 3--6, read the
guide and decide which of these descriptions best fits you. Do you
think the strategies described for handling people like you are
effective? Are other strategies equally or more effective?
