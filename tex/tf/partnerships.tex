\chaplbl{Partnerships}{s:partner}

It's fashionable in tech circles
to disparage universities and government institutions as slow-moving dinosaurs,
but in my experience they are no worse than companies of similar size.
Your local school board, library, and your city councillor's office
may be able to offer space, funding, publicity,
connections with other groups that you may not have met yet,
and a host of other useful things;
getting to know them can help you solve or avoid problems in the short term
and pay dividends down the road.

The hardest part about building these relationships is getting started.
Set aside an hour or two every month
to find allies and maintain your relationships with them.
One way to do this is to ask them for advice:
how do they think you ought to raise awareness of what you're doing?
Where have they found space to run classes?
What needs do they think aren't being met
and would you be able to meet them?
Any group that has been around for a few years will have useful advice;
they will also be flattered to be asked,
and will know who you are the next time you call.

\begin{aside}{Reader Beware}
  Unlike most of the rest of this book,
  this chapter is drawn more from things I have seen than from things I have done.
  Most of my attempts to get large institutions to change have been unproductive
  (which is part of why I left academia in 2010
  to restart \hreffoot{http://software-carpentry.org}{Software Carpentry}).
  While contributions to any part of this book are welcome,
  I would especially like to hear about your experiences with this.
\end{aside}

\seclbl{Working With Schools}{s:partner-schools}

Everyone is afraid of the unknown and of embarrassing themselves.
As a result,
most people would rather fail than change.
For example, Lauren Herckis looked at
\hreffoot{https://www.insidehighered.com/news/2017/07/06/anthropologist-studies-why-professors-dont-adopt-innovative-teaching-methods}{why university faculty don't adopt better teaching methods}.
She found that the main reason is a fear of looking stupid in front of students;
secondary reasons were
concern that the inevitable bumps in changing teaching methods would affect course evaluations
and people's desire to continue emulating the teachers who had inspired them.
It's pointless to argue about whether these issues are ``real'' or not:
faculty believe they are,
so any plan to work with faculty needs to address them.

\cite{Bark2015} did a two-part study of how computer science educators adopt new teaching practices
as individuals, organizationally, and in society as a whole.
They asked and answered three key questions:

\begin{description}

\item[How do faculty hear about new teaching practices?]
  They intentionally seek out new practices
  because they're motivated to solve a problem (particularly student engagement),
  are made aware through deliberate initiatives by their institutions,
  pick them up from colleagues,
  or get them from expected \emph{and unexpected} interactions at conferences
  (teaching-related or otherwise).

\item[Why do they try them out?]
  Sometimes because of institutional incentives
  (e.g., they innovate to improve their chances of promotion),
  but there is often tension at research institutions
  where rhetoric about the importance of teaching is largely disbelieved.
  Another important reason is their own cost/benefit analysis:
  will the innovation save them time?
  A third is that they are inspired by role models---again,
  this largely affects innovations aimed to improve engagement and motivation
  rather than learning outcomes---and a fourth is trusted sources,
  e.g.,
  people they meet at conferences who are in the same situation they are
  and reported successful adoption.

  But faculty had concerns that were often not addressed by people advocating changes.
  The first was Glass's Law:
  any new tool or practice initially slows you down,
  so while new practices might make teaching more effective in the long run,
  they can't be afforded in the short run.
  Another is that the physical layout of classrooms makes many new practices hard:
  for example,
  discussion groups don't work well in theater-style seating.

  But the most telling result was this:
  ``Despite being researchers themselves,
  the CS faculty we spoke to for the most part did not believe that
  results from educational studies were credible reasons to try out teaching practices.''
  This is consistent with other findings:
  even people whose entire careers are devoted to research often disregard educational research.

\item[Why do they keep using them?]
  As \cite{Bark2015} says, ``Student feedback is critical,''
  and is often the strongest reason to continue using a practice,
  even though we know that students' self-reports don't correlate strongly with learning outcomes \cite{Star2014,Uttl2017}
  (though student attendance in lectures is a good indicator of engagement).
  Another reason to retaining a practice is institutional requirements,
  although if this is the only motivation,
  people will often drop the practice
  when the explicit incentive or monitoring is removed.

\end{description}

The good news is that you can tackle these problems systematically.
\cite{Baue2015} looked at adoption of new medical techniques within the US Veterans Administration.
They found that evidence-based practices in medicine
take an average of 17 years to be incorporated into routine general practice,
and that only about half of such practices are ever widely adopted.
This depressing finding and others like it spurred the growth of
\gref{g:implementation-science}{implementation science},
which is the study of how to get people to adopt better practices.

As Chapter~\ref{s:community} said,
the starting point is to find out what the people you're trying to help believe they need.
For example,
\cite{Yada2016} summarizes feedback from K-12 teachers on the preparation and support they want.
While it may not all be applicable to all settings,
having a cup of tea with a few people and listening before you speak
makes a world of difference to their willingness to try something new.

Once you know what people need,
the next step is to make changes incrementally,
within institutions' own frameworks.
\cite{Nara2018} describes an intensive three-year bachelor's program
based on tight-knit cohorts and administrative support
that tripled graduation rates,
while \cite{Hu2017} describes impact of introducing a six-month certification program
for existing high school teachers who want to teach computing.
The number of computing teachers had been stable from 2007 to 2013,
but quadrupled after introduction of the new certification program
without diluting quality:
new-to-computing teachers seemed to be as effective as teachers with more computing training
at teaching the introductory course.

More broadly,
\cite{Borr2014} categorizes ways to make change happen in higher education.
The categories are defined by whether the change is individual or systemic
and whether it is prescribed (top-down) or emergent (bottom-up).
The person trying to make the changes (and make them stick)
has a different role in each situation,
and should pursue different strategies accordingly.
The paper goes on to explain each of the methods in detail,
while \cite{Hend2015a,Hend2015b} present the same ideas in more actionable form.

Coming from outside,
you will probably fall into the Individual/Emergent category to start with,
since you will be approaching teachers one by one
and trying to make change happen bottom-up.
If this is the case,
the strategies Borrego and Henderson recommend center around
having teachers reflect on their teaching individually or in groups.
Live coding to show them what you do or the examples you use,
then having them live code in turn
to show how they would use those ideas and techniques in their setting,
gives everyone a chance to pick up things that will be useful to them in their context.

\seclbl{Working Outside Schools}{s:partner-outside}

Schools and universities aren't the only places people go to learn programming;
over the past few years, a growing number have turned to intensive bootcamp programs.
These are typically one to six months long,
run by volunteer groups or by for-profit companies,
and target people who are retraining to get into tech.
Some are very high quality,
but others exist primarily to separate people from their money \cite{McMi2017}.

\cite{Thay2017} interviewed 26 alumni of such bootcamps
that provide a second chance for those who missed computing education opportunities earlier
(though phrasing it this way makes some pretty big assumptions
when it comes to people from underrepresented groups).
Bootcamp students face great personal costs and risks:
they must spend significant time, money, and effort before, during, and after bootcamps,
and changing careers can take a year or more.
Several interviewees felt that their certificates were looked down on by employers;
as some said,
getting a job means passing an interview,
but since interviewers often won't share their reasons for rejection,
it's hard to know what to fix or what else to learn.
Many resorted to internships (paid or otherwise)
and spent a lot of time building their portfolios and networking.
The three informal barriers they most clearly identified were jargon,
impostor syndrome,
and a sense of not fitting in.

\cite{Burk2018} dug into this a bit deeper
by comparing the skills and credentials that tech industry recruiters are looking for
to those provided by four-year degrees and bootcamps.
Based on interviews with 15 hiring managers from firms of various sizes and some focus groups,
they found that recruiters uniformly emphasized soft skills
(especially teamwork, communication, and the ability to continue learning).
Many companies required a four-year degree
(though not necessarily in computer science),
but many also praised bootcamp graduates for being older or more mature
and having more up-to-date knowledge.

If you are approaching an existing bootcamp,
your best strategy could be to emphasize what you know about teaching
rather than what you know about tech,
since many of their founders and staff have programming backgrounds
but little or no training in education.
The first few chapters of this book have played well with this audience in the past,
and \cite{Lang2016} describes
evidence-based teaching practices that can be put in place
with minimal effort and at low cost.
These may not have the most impact,
but scoring a few early wins helps build support for larger efforts.

\seclbl{Final Thoughts}{s:partner-final}

It is impossible to change large institutions on your own:
you need allies,
and to get allies,
you need tactics.
The most useful guide I have found is \cite{Mann2015},
which catalogs more than four dozen of these
and organizes them according to whether they're best deployed early,
later,
throughout the change cycle,
or when you encounter resistance.
A handful of their patterns include:

\begin{description}

\item[In Your Space:]
  Keep the new idea visible
  by placing reminders throughout the organization.

\item[Token:]
  To keep a new idea alive in a person's memory,
  hand out tokens that can be identified with the topic being introduced.

\item[Champion Skeptic:]
  Ask strong opinion leaders who are skeptical of your new idea
  to play the role of ``official skeptic''.
  Use their comments to improve your effort,
  even if you don't change their minds.

\item[Future Commitment:]
  If you are able to anticipate some of your needs,
  you can ask for a future commitment from busy people.
  If given some lead time,
  they may be more willing to help.
  
\end{description}

Conversely,
\cite{Farm2006} has ten tongue-in-cheek rules for ensuring that new tools \emph{aren't} adopted,
all of which apply to new teaching practices as well:

\begin{enumerate}

\item
  Make it optional.

\item
  Economize on training.

\item
  Don't use it in a real project.

\item
  Never integrate it.

\item
  Use it sporadically.

\item
  Make it part of a quality initiative.

\item
  Marginalize the champion.

\item
  Capitalize on early missteps.

\item
  Make a small investment.

\item
  Exploit fear, uncertainty, doubt, laziness, and inertia.

\end{enumerate}

The most important strategy is
to be willing to change your goals
based on what you learn from the people you are trying to help.
Tutorials showing them how to use a spreadsheet
might help them more quickly and more reliably than
an introduction to JavaScript.
I have often made the mistake of confusing things I was passionate about
with things that other people ought to know;
if you truly want to be a partner,
always remember that learning and change have to go both ways.

\seclbl{Exercises}{s:partner-exercises}

\subsection*{Collaborations (small groups/30)}

Answer the following questions on your own,
then compare your answers to those given by other members of your group.

\begin{enumerate}

\item
  Do you have any agreements or relationships with other groups?

\item
  Do you want to have relationships with any other groups?

\item
  How would having (or not having) collaborations
  help you to achieve your goals?

\item
  What are your key collaborative relationships?

\item
  Are these the right collaborators for achieving your goals?

\item
  What groups or entities would you like your organization
  to have agreements or relationships with?

\end{enumerate}

\subsection*{Educationalization (whole class/10)}

\cite{Laba2008} explores why the United States and other countries
keep pushing the solution of social problems onto educational institutions
and why that continues not to work.
As he points out,
``[Education] has done very little to promote equality of race, class, and gender;
to enhance public health, economic productivity, and good citizenship;
or to reduce teenage sex, traffic deaths, obesity, and environmental destruction.
In fact,
in many ways it has had a negative effect on these problems
by draining money and energy away from social reforms that might have had a more substantial impact.''
He goes on to write:

\begin{quote}

  So how are we to understand the success of this institution
  in light of its failure to do what we asked of it?
  One way of thinking about this is that
  education may not be doing what we ask,
  but it is doing what we want.
  We want an institution that will pursue our social goals
  in a way that is in line with the individualism at the heart of the liberal ideal,
  aiming to solve social problems
  by seeking to change the hearts, minds, and capacities of individual students.
  Another way of putting this is that
  we want an institution through which we can express our social goals
  without violating the principle of individual choice
  that lies at the center of the social structure,
  even if this comes at the cost of failing to achieve these goals.
  So education can serve as a point of civic pride,
  a showplace for our ideals,
  and a medium for engaging in uplifting but ultimately inconsequential disputes
  about alternative visions of the good life.
  At the same time,
  it can also serve as a convenient whipping boy
  that we can blame for its failure to achieve our highest aspirations for ourselves as a society.

\end{quote}

How do efforts to teach computational thinking and digital citizenship in schools
fit into this framework?
Do bootcamps avoid these traps or simply deliver them in a new guise?

\subsection*{Institutional Adoption (whole class/15)}

Re-read the list of motivations to adopt new practices
given in Section~\ref{s:partner-schools}.
Which of these apply to you and your colleagues?
Which are irrelevant to your context?
Which do you emphasize
if and when you interact with people working in formal educational institutions?

\subsection*{Making It Fail (small groups/15)}

Working in small groups,
re-read the list in Section~\ref{s:partner-final} of ways to ensure new tools aren't adopted.
Which of these have you seen done recently?
Which have you done yourself?
What form did they take?

\subsection*{Mentoring (whole class/15)}

The Institute for African-American Mentoring in Computer Science
has published \hreffoot{http://iaamcs.org/guidelines}{a brief set of guidelines for mentoring doctoral students}.
Read the guidelines individually,
then go through them as a class
and rate your efforts for your own group as +1 (definitely doing),
0 (not sure or not applicable),
or -1 (definitely not doing).
