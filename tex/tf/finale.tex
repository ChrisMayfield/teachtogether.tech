\chapter{Why I Teach}\label{s:finale}

When I first started teaching at the University of Toronto, some of my
students asked me why I was doing it. This was my answer:

\begin{quote}

When I was your age, I thought universities existed to teach people
how to learn. Later, in grad school, I thought universities were about
doing research and creating new knowledge. Now that I'm in my forties,
though, I've realized that what we're really teaching you is how to
take over the world, because you're going to have to whether you want
to or not.

My parents are in their seventies. They don't run the world any more;
it's people my age who pass laws, set interest rates, and make
life-and-death decisions in hospitals. As scary as it is, \emph{we} are the
grownups.

Twenty years from now, though, we'll be heading for retirement and
\emph{you} will be in charge. That may sound like a long time when you're
nineteen, but take three breaths and it's gone. That's why we give you
problems whose answers can't be cribbed from last year's notes. That's
why we put you in situations where you have to figure out what needs
to be done right now, what can be left for later, and what you can
simply ignore. It's because if you don't learn how to do these things
now, you won't be ready to do them when you have to.

\end{quote}

It was all true, but it wasn't the whole story. I don't want people to
make the world a better place so that I can retire in comfort. I want
them to do it because it's the greatest adventure of our time. A
hundred and fifty years ago, most societies practiced slavery. A
hundred years ago, my grandmother \hreffoot{https://en.wikipedia.org/wiki/The\_Famous\_Five\_(Canada)}{wasn't legally a
person} in Canada. Fifty years ago, most of the world's
people suffered under totalitarian rule; in the year I was born,
judges were still ordering electroshock therapy to ``cure''
homosexuals. There's still a lot wrong with the world, but look at how
many more choices we have than our grandparents did. Look at how many
more things we can know, and be, and enjoy.

This didn't happen by chance. It happened because millions of people
made millions of little decisions, the sum of which was a better world.
We don't think of these day-to-day decisions as political, but every
time we buy one brand of running shoe instead of another or shout an
anatomical insult instead of a racial one at a cab driver, we're
choosing one vision of the world instead of another.

In his 1947 essay ``\hreffoot{http://www.resort.com/~prime8/Orwell/whywrite.html}{Why I Write}'', George Orwell wrote:

\begin{quote}

In a peaceful age I might have written ornate or merely descriptive
books, and might have remained almost unaware of my political
loyalties. As it is I have been forced into becoming a sort of
pamphleteer{\ldots} Every line of serious work that I have
written since 1936 has been written, directly or indirectly, against
totalitarianism{\ldots} It seems to me nonsense, in a period
like our own, to think that one can avoid writing of such subjects.
Everyone writes of them in one guise or another. It is simply a
question of which side one takes{\ldots}

\end{quote}

Replace ``writing'' with ``teaching'' and you'll have the reason I do what I
do. The world doesn't get better on its own. It gets better because
people make it better: penny by penny, vote by vote, and one lesson at a
time. So:

\begin{quote}

Start where you are.\\
Use what you have.\\
Help who you can.

\end{quote}

Thank you for reading. I hope we can learn something together some day.
