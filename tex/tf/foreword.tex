\chapter*{Foreword}
\markboth{Foreword}{Foreword}

Hundreds of grassroots groups have sprung up around the world to teach
programming, web design, robotics, and other skills to \protect\hyperlink{g:free-range-learner}{free-range
learners} outside traditional classrooms. These
groups exist so that people don't have to learn these things on their
own, but ironically, their founders and instructors are often teaching
themselves how to teach.

There's a better way. Just as knowing a few basic facts about germs
and nutrition can help you stay healthy, knowing a few things about
psychology, instructional design, inclusivity, and community
organization can help you be a more effective teacher. This book
presents evidence-based practices you can use right now, explains why
we believe they are true, and points you at other resources that will
help you go further. Its four sections cover:

\begin{itemize}
\tightlist
\item
  how people learn;
\item
  how to design lessons that work;
\item
  how to deliver those lessons; and
\item
  how to grow a community of practice around teaching.
\end{itemize}

\section{This Book Belongs to Everyone}\label{s:index-everyone}

This book is a community resource. Parts of it were originally
created for \href{http://carpentries.github.io/instructor-training/}{the Software Carpentry instructor training
program}, which has been run over several
hundred times over the past six years, and all of it can be freely
distributed and re-used under the \href{https://creativecommons.org/licenses/by/4.0/}{Creative Commons - Attribution 4.0
license} (Appendix~\ref{s:license}). Please see
\url{http://teachtogether.tech/} to download a digital version
or \href{http://www.lulu.com/commerce/index.php?fBuyContent=23123539}{purchase a printed copy} at cost.

Contributions of all kinds are welcome, from errata and minor
improvements to entirely new sections and chapters. All proposed
contributions will be managed in the same way as edits to Wikipedia
or patches to open source software, and all contributors will be
credited for their work each time a new version is released. Please
see Appendix~\ref{s:joining} for details and our code of conduct.
