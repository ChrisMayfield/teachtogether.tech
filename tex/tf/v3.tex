\chapter{Design Notes}\label{s:v3}

This design follows the backward design process described in
Chapter~\ref{s:process}.

\section{Brainstorming}\label{s:v3-brainstorming}

These questions and answers provide a rough scope for the material.

\subsection*{What problems will learners learn how to solve?}

  \begin{enumerate}
  \item
    How people learn and what that tells us about how best to teach
    them (educational psychology, cognitive load, study skills).
  \item
    How to design and deliver instruction in computing skills
    (backward curriculum design, some pedagogical content knowledge
    for computing).
  \item
    How to deliver lessons (teaching as a performance art, live
    coding, motivation and demotivation, and automation).
  \item
    How to grow a teaching community (community organization and
    marketing).
  \end{enumerate}

\subsection*{What is out of scope?}

  \begin{enumerate}
  \item
    How to teach children or people with special learning needs.
    Much of what's in this material applies to those learners, but
    they have different or extra needs.
  \item
    How to rigorously assess the impact of training. Informal
    self-assessment will be included, but we will not try to explain
    how to do publishable scientific research in education.
  \item
    How to design and deliver entire degree programs and other
    extended curriculum. Again, much of what's in this material
    applies, but the extra needs of large-scale curriculum design is
    out of scope.
  \end{enumerate}

\subsection*{What concepts and techniques will learners encounter?}

  \begin{enumerate}
  \item
    7±2 and chunking.
  \item
    Authentic tasks with tangible artifacts.
  \item
    Bloom's Taxonomy, Fink's Taxonomy, and Piaget's development
    stage theory.
  \item
    Branding.
  \item
    Cognitive development from novice to competent to expert.
  \item
    Cognitive load.
  \item
    Collaborative lesson development.
  \item
    Concept mapping.
  \item
    Designing assessments with diagnostic power.
  \item
    Dunning-Kruger effect.
  \item
    Expert blind spot.
  \item
    Externalized cognition.
  \item
    Fixed vs. growth mindset (and critiques of it).
  \item
    Formative vs. summative assessment.
  \item
    Governance models of community organizations.
  \item
    Inquiry-based learning (and critiques of it).
  \item
    Intrinsic vs. extrinsic motivation.
  \item
    \emph{Jugyokenkyu} (lesson study).
  \item
    Learner personas.
  \item
    Legitimate peripheral participation in a community of practice.
  \item
    Live coding (teaching as a performance art).
  \item
    Pedagogical content knowledge (PCK) and technological
    pedagogical and content knowledge (TPACK).
  \item
    Peer instruction.
  \item
    Reflective (deliberate) practice.
  \item
    Backward design.
  \item
    Stereotype threat (and critiques of it).
  \item
    Working memory vs. persistent memory.
  \end{enumerate}

\subsection*{What mistakes or misconceptions will they have?}

  \begin{enumerate}
  \item
    Children and adults learn the same way.
  \item
    Computing education should be for and about computer science.
  \item
    Programming skill is innate.
  \item
    Student evaluations of courses are indicative of learning
    outcomes.
  \item
    Teaching ability is innate.
  \item
    The best way to teach is to throw people in at the deep end.
  \item
    The best way to teach is to use ``real'' tools right from the
    start.
  \item
    Visual-auditory-kinesthetic (VAK) learning styles are real.
  \item
    Women just don't like programming or innately have less
    aptitude.
  \item
    Getting a (better) job is the main reason someone should learn
    how to program.
  \end{enumerate}

\subsection*{In what contexts will this material be used?}

  \begin{enumerate}
  \item
    Primary: an intensive weekend workshop for people in tech who
    want to volunteer with grassroots get-into-coding initiatives.
  \item
    Secondary: self-study or guided study by such people.
  \item
    Secondary: a one-semester undergraduate course for computer
    science majors interested in education.
  \end{enumerate}

\section{Intended Audience}\label{s:v3-intended-audience}

\emph{These learner personas clarify what readers are interested in and what
can be assumed about their prior knowledge.}

\begin{description}
\item[Emily]
trained as a librarian, and now works as a web designer and project
manager in a small consulting company. In her spare time, she helps
run web design classes for women entering tech as a second career.
She is now recruiting colleagues to run more classes in her area
using the lessons that she has created, and wants to know how to
grow a volunteer teaching organization.
\item[Moshe]
is a professional programmer with two teenage children whose school
doesn't offer programming classes. He has volunteered to run a
monthly after-school programming club, and while he frequently gives
presentations to colleagues, he has no experience designing lessons.
He wants to learn how to build effective lessons in collaboration
with others, and is interested in turning his lessons into a
self-paced online course.
\item[Samira]
is an undergraduate in robotics who is thinking about becoming a
full-time teacher after she graduates. She wants to help teach
weekend workshops for undergraduate women, but has never taught an
entire class before, and feels uncomfortable teaching things that
she's not an expert in. She wants to learn more about education in
general in order to decide if it's for her.
\item[Gene]
is a professor of computer science whose research area is operating
systems. They have been teaching undergraduate classes for six
years, and increasingly believe that there has to be a better way.
The only training available through their university's teaching and
learning center relates to posting assignments and grades in the
learning management system, so they want to find out what else they
ought to be asking for.
\end{description}

Common elements:

\begin{itemize}
\item
  A variety of technical backgrounds and skills.
\item
  May or may not have some teaching experience.
\item
  No formal training in teaching, lesson design, or community
  organization.
\item
  More likely to teach in free-range settings often than in
  institutional classrooms with required homework, final exams, and
  externally-mandated curriculum.
\item
  Focused on teenagers and adults rather children.
\item
  Limited time and resources (either because they are volunteers, or
  because their institution considers teaching a secondary
  responsibility).
\end{itemize}

Learning contexts:

\begin{description}
\item[Emily]
will take part in a weekly online reading group with her volunteers.
\item[Moshe]
will cover part of this book in a two-day weekend workshop and study
the rest on his own.
\item[Samira]
will use this book in a one-semester undergraduate course with
assignments, a project, and a final exam.
\item[Gene]
will read the book on their own in their office or while commuting,
wishing all the while that universities did more to support
high-quality teaching.
\end{description}

\section{Exercises}\label{s:v3-exercises}

\emph{These formative exercises summarize what learners will be able to do
with their new knowledge. The finished book will include others as
well.}

\begin{enumerate}
\item
  Create multiple choice questions whose incorrect answers have
  diagnostic power.
\item
  Give feedback on a recorded teaching episode and compare points with
  expert feedback.
\item
  Create a Parsons Problem.
\item
  Create a short debugging exercise.
\item
  Create a short execution tracing exercise.
\item
  Explain their personal motivation for teaching.
\item
  Explain their community of practice's aims and conventions.
\item
  Explain the difference between an oversight board and a governance
  board.
\item
  Design an hour-long lesson using backward design.
\item
  Describe the pros and cons of standardized testing.
\item
  Create learner personas for their intended students.
\item
  Write and critique learning objectives for an hour-long lesson.
\item
  Write and critique a short value proposition for a class they intend
  to offer.
\item
  Teach a short lesson using live coding and critique a recording of
  it.
\item
  Create a short video lesson and critique it.
\item
  Construct a short series of faded examples that illustrate a
  problem-solving pattern in programming.
\item
  Describe ways in which they differ from their intended learners.
\item
  Create and critique an elevator pitch for a course they intend to
  teach.
\item
  Write a ``cold call'' email to solicit support for what they intend to
  teach.
\item
  Create a concept map for a topic they intend to teach.
\item
  Explain six strategies students can use to learn more effectively.
\item
  Analyze and critique the accessibility of a short online lesson.
\item
  Analyze and critique the inclusivity of a short lesson.
\item
  Create and critique a non-programming exercise to use in a
  programming class.
\item
  Describe the relative merits of block-based and text-based
  environments for introductory programming classes for adults.
\item
  Create a one-to-one matching exercise for use in a class they intend
  to teach.
\item
  Create a diagram labelling exercise for use in a class they intend
  to teach.
\item
  Write and submit an improvement or extension to an existing lesson
  and review a peer's submission.
\item
  Describe the pros and cons of collaborative note-taking.
\item
  Describe the pros and cons of gamification in online learning.
\item
  Create and critique a short questionnaire for assessing learners'
  prior knowledge.
\item
  Conduct a demonstration lesson using peer instruction.
\item
  Describe ways in which computing is unwelcoming to or unaccepting of
  people from diverse backgrounds.
\item
  Demonstrate several ways to ensure that an instructor's attention is
  fairly distributed through a class.
\item
  Describe the pros and cons of in-person, automated, and hybrid
  teaching strategies.
\item
  Write and critique automated tests for a short programming exercise.
\end{enumerate}

\section{Outline}\label{s:v3-outline}

\emph{Each major section can be covered in detail in 2--3 weeks in a
conventional classroom format, or in less detail in one full day in an
intensive workshop format.}

\begin{enumerate}
\item
  Introduction
\item
  Learning

  \begin{enumerate}
  \item
    Building Mental Models
  \item
    Expertise and Memory
  \item
    Cognitive Load
  \item
    Effective Learning
  \end{enumerate}
\item
  Designing

  \begin{enumerate}
  \item
    A Lesson Design Process
  \item
    Pedagogical Content Knowledge
  \end{enumerate}
\item
  Delivering

  \begin{enumerate}
  \item
    Teaching as a Performance Art
  \item
    Live Coding
  \item
    Motivation and Demotivation
  \item
    Automation
  \item
    Hybrid Models
  \end{enumerate}
\item
  Organizing

  \begin{enumerate}
  \item
    Awareness
  \item
    Operations
  \item
    Building Community
  \end{enumerate}
\end{enumerate}

\section{Course Overview}\label{s:v3-course-overview}

\subsection*{Brief Description}

Teaching isn't magic: good teachers are simply people who have learned
how to design lessons to achieve concrete goals, how to get and use
feedback from learners, and how to work well with other teachers. This
book will show you how to do these things and more, and will introduce
you to some of the research that explains why some things work and some
things don't. It is primarily intended for people in tech with no formal
training in teaching who want to help adults learn how to create web
sites, write programs, and analyze data, but the ideas apply equally
well to other groups in other settings.

\subsection*{Learning Objectives}

Learners will be able to\ldots{}

\begin{enumerate}
\item
  Explain the cognitive changes that occur as people go from novice to
  competent to expert and how best to teach each group.
\item
  Explain how to design, construct, and maintain lessons in a
  systematic, collaborative way.
\item
  Design exercises to help correct key misconceptions that learners
  have about computing.
\item
  Summarize key elements of pedagogical content knowledge related to
  computing and other technical skills.
\item
  Compare and contrast teaching with other performance arts and
  participate in structured critiques of live teaching.
\item
  Compare and contrast interactive teaching, automated teaching, and
  hybrid models.
\item
  Describe factors that motivate or demotivate adult learners and how
  to take those into account when teaching.
\item
  Describe ways in which members of different groups are made to feel
  unwelcome or excluded in computing and what can be done to make
  computing more inclusive.
\item
  Explain the purpose and value of their teaching and of their
  community of practice.
\item
  Be a productive member of a community of teaching practice.
\end{enumerate}

\subsection*{Prerequisites}

Some exercises will assume a small amount of programming knowledge:
readers should know how to loop over the elements of a list, how to take
action using an if-else statement, and how to write and call a simple
function.
