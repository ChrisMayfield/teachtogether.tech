\chapter{Glossary}\label{s:gloss}

\textbf{\hypertarget{g:absolute-beginner}{Absolute beginner}\label{g:absolute-beginner}}: Someone who has
never encountered concepts or material before. The term is used in
distinction to \emph{false beginner}.

\textbf{\hypertarget{g:authentic-task}{Authentic task}\label{g:authentic-task}}: A task which contains
important elements of things that learners would do in real
(non-classroom situations). To be authentic, a task should require
learners to construct their own answers rather than choose between
provided answers, and to work with the same tools and data they would
use in real life.

\textbf{\hypertarget{g:automaticity}{Automaticity}\label{g:automaticity}}: The ability to do a task
without concentrating on its low-level details.

\textbf{\hypertarget{g:backward-design}{Backward design}\label{g:backward-design}}: An instructional design
method that works backwards from a summative assessment to formative
assessments and thence to lesson content.

\textbf{\hypertarget{g:behaviorism}{Behaviorism}\label{g:behaviorism}}: A theory of learning whose
central principle is stimulus and response, and whose goal is to explain
behavior without recourse to internal mental states or other
unobservables. See also \emph{cognitivism}.

\textbf{\hypertarget{g:blooms-taxonomy}{Bloom's Taxonomy}\label{g:blooms-taxonomy}}: A six-part
hierarchical classification of understand whose levels are \emph{knowledge},
\emph{comprehension}, \emph{application}, \emph{analysis}, \emph{synthesis}, and
\emph{evaluation} that has been widely adopted. See also \emph{Fink's Taxonomy}.

\textbf{\hypertarget{g:brand}{Brand}\label{g:brand}}: The associations people have with a
product's name or identity.

\textbf{\hypertarget{g:calibrated-peer-review}{Calibrated peer review}\label{g:calibrated-peer-review}}: Having
students compare their reviews of sample work with an instructor's
reviews before being allowed to review their peers' work.

\textbf{\hypertarget{g:chunking}{Chunking}\label{g:chunking}}: The act of grouping related concepts
together so that they can be stored and processed as a single unit.

\textbf{\hypertarget{g:co-teaching}{Co-teaching}\label{g:co-teaching}}: Teaching with another
instructor in the classroom.

\textbf{\hypertarget{g:cognitive-apprenticeship}{Cognitive apprenticeship}\label{g:cognitive-apprenticeship}}: A
theory of learning that emphasizes the process of a master passing on
skills and insights situationally to an apprentice.

\textbf{\hypertarget{g:cognitive-load-theory}{Cognitive Load Theory}\label{g:cognitive-load-theory}}: \emph{Cognitive
load} is the amount of mental effort required to solve a problem.
Cognitive load theory divides this effort into \emph{intrinsic},
\emph{extraneous}, and \emph{germane}, and holds that people learn faster and
better when extraneous load is reduced.

\textbf{\hypertarget{g:cognitivism}{Cognitivism}\label{g:cognitivism}}: A theory of learning that holds
that mental states and processes can and must be included in models of
learning. See also \emph{behaviorism}.

\textbf{\hypertarget{g:community-of-practice}{Community of practice}\label{g:community-of-practice}}: A
self-perpetuating group of people who share and develop a craft such as
knitters, musicians, or programmers. See also \emph{legitimate peripheral
participation}.

\textbf{\hypertarget{g:community-representation}{Community representation}\label{g:community-representation}}: Using
cultural capital to highlight students' social identities, histories,
and community networks in learning activities.

\textbf{\hypertarget{g:computational-integration}{Computational integration}\label{g:computational-integration}}:
Using computing to re-implement pre-existing cultural artifacts, e.g.,
creating variants of traditional designs using computer drawing tools.

\textbf{\hypertarget{g:competent-practitioner}{Competent practitioner}\label{g:competent-practitioner}}: Someone
who can do normal tasks with normal effort under normal circumstances.
See also \emph{novice} and \emph{expert}.

\textbf{\hypertarget{g:computational-thinking}{Computational thinking}\label{g:computational-thinking}}: Thinking
about problem-solving in ways inspired by programming (though the term
is used in many other ways).

\textbf{\hypertarget{g:concept-map}{Concept map}\label{g:concept-map}}: A picture of a mental model in
which concepts are nodes in a graph and relationships are (labelled)
arcs.

\textbf{\hypertarget{g:connectivism}{Connectivism}\label{g:connectivism}}: A theory of learning holds
that knowledge is distributed, that learning is the process of
navigating, growing, and pruning connections, and which emphasizes the
social aspects of learning made possible by the Internet

\textbf{\hypertarget{g:constructivism}{Constructivism}\label{g:constructivism}}: A theory of learning that
views learners as actively constructing knowledge.

\textbf{\hypertarget{g:content-knowledge}{Content knowledge}\label{g:content-knowledge}}: A person's
understanding of a subject. See also \emph{general pedagogical knowledge} and
\emph{pedagogical content knowledge}.

\textbf{\hypertarget{g:contributing-student}{Contributing student pedagogy}\label{g:contributing-student}}:
Having students produce artifacts to contribute to other students'
learning.

\textbf{\hypertarget{g:conversational-programmer}{Conversational programmer}\label{g:conversational-programmer}}:
Someone who needs to know enough about computing to have a meaningful
conversation with a programmer, but isn't going to program themselves.

\textbf{\hypertarget{g:cs0}{CS0}\label{g:cs0}}: An introductory college-level course on
computing aimed at non-majors with little or no prior experience of
programming.

\textbf{\hypertarget{g:cs1}{CS1}\label{g:cs1}}: An introductory college-level computer science
course, typically one semester long, that focuses on variables, loops,
functions, and other basic mechanics.

\textbf{\hypertarget{g:cs2}{CS2}\label{g:cs2}}: A second college-level computer science course
that typically introduces basic data structures such as stacks, queues,
and dictionaries.

\textbf{\hypertarget{g:deficit-model}{Deficit model}\label{g:deficit-model}}: The idea that some groups
are under-represented in computing (or some other field) because their
members lack some attribute or quality.

\textbf{\hypertarget{g:deliberate-practice}{Deliberate practice}\label{g:deliberate-practice}}: The act of
observing performance of a task while doing it in order to improve
ability.

\textbf{\hypertarget{g:demonstration-lesson}{Demonstration lesson}\label{g:demonstration-lesson}}: A staged
lesson in which one teacher presents material to a class of actual
students while other teachers observe in order to learn new teaching
techniques.

\textbf{\hypertarget{g:diagnostic-power}{Diagnostic power}\label{g:diagnostic-power}}: The degree to which a
wrong answer to a question or exercise tells the instructor what
misconceptions a particular learner has.

\textbf{\hypertarget{g:direct-instruction}{Direct instruction}\label{g:direct-instruction}}: A teaching method
centered around meticulous curriculum design delivered through
prescribed script.

\textbf{\hypertarget{g:educational-psychology}{Educational psychology}\label{g:educational-psychology}}: The study
of how people learn. See also \emph{instructional design}.

\textbf{\hypertarget{g:ego-depletion}{Ego depletion}\label{g:ego-depletion}}: The impairment of self
control that occurs when it is exercised intensively or for long
periods.

\textbf{\hypertarget{g:elevator-pitch}{Elevator pitch}\label{g:elevator-pitch}}: A short description of an
idea, project, product, or person that can be delivered and understood
in just a few seconds.

\textbf{\hypertarget{g:end-user-programmer}{End-user programmer}\label{g:end-user-programmer}}: Someone who
does not consider themselves a programmer, but who nevertheless writes
and debugs software, such as an artist creating complex macros for a
drawing tool.

\textbf{\hypertarget{g:end-user-teacher}{End-user teacher}\label{g:end-user-teacher}}: By analogy with
\emph{end-user programmer}, someone who is teaching frequently, but whose
primary occupation is not teaching, who has little or no background in
pedagogy, and who may work outside institutional classrooms.

\textbf{\hypertarget{g:expert}{Expert}\label{g:expert}}: Someone who can diagnose and handle
unusual situations, knows when the usual rules do not apply, and tends
to recognize solutions rather than reasoning to them. See also
\emph{competent practitioner} and \emph{novice}.

\textbf{\hypertarget{g:expert-blind-spot}{Expert blind spot}\label{g:expert-blind-spot}}: The inability of
experts to empathize with novices who are encountering concepts or
practices for the first time.

\textbf{\hypertarget{g:expertise-reversal}{Expertise reversal effect}\label{g:expertise-reversal}}: The way in
which instruction that is effective for novices becomes ineffective for
competent practitioners or experts.

\textbf{\hypertarget{g:externalized-cognition}{Externalized cognition}\label{g:externalized-cognition}}: The use
of graphical, physical, or verbal aids to augment thinking.

\textbf{\hypertarget{g:extrinsic-motivation}{Extrinsic motivation}\label{g:extrinsic-motivation}}: Being driven
by external rewards such as payment or fear of punishment. See also
\emph{intrinsic motivation}.

\textbf{\hypertarget{g:faded-example}{Faded example}\label{g:faded-example}}: A series of examples in
which a steadily increasing number of key steps are blanked out. See
also \emph{scaffolding}.

\textbf{\hypertarget{g:false-beginner}{False beginner}\label{g:false-beginner}}: Someone who has studied a
language before but is learning it again. False beginners start at the
same point as true beginners (i.e., a pre-test will show the same
proficiency) but can move much more quickly.

\textbf{\hypertarget{g:far-transfer}{Far transfer}\label{g:far-transfer}}: Transfer of learning or
proficiency between widely-separated domains, e.g., improvement in math
skills as a result of playing chess.

\textbf{\hypertarget{g:finks-taxonomy}{Fink's Taxonomy}\label{g:finks-taxonomy}}: A six-part
non-hierarchical classification of understanding first proposed in
\cite{Fink2013} whose categories are \emph{foundational knowledge},
\emph{application}, \emph{integration}, \emph{human dimension}, \emph{caring}, and \emph{learning
how to learn}. See also \emph{Bloom's Taxonomy}.

\textbf{\hypertarget{g:fixed-mindset}{Fixed mindset}\label{g:fixed-mindset}}: The belief that an ability
is innate, and that failure is due to a lack of some necessary
attribute. See also \emph{growth mindset}.

\textbf{\hypertarget{g:flipped-classroom}{Flipped classroom}\label{g:flipped-classroom}}: One in which
learners watch recorded lessons on their own time, while class time is
used to work through problem sets and answer questions.

\textbf{\hypertarget{g:flow}{Flow}\label{g:flow}}: The feeling of being fully immersed in an
activity; frequently associated with high productivity.

\textbf{\hypertarget{g:fluid-representation}{Fluid representation}\label{g:fluid-representation}}: The ability
to move quickly between different models of a problem.

\textbf{\hypertarget{g:formative-assessment}{Formative assessment}\label{g:formative-assessment}}: Assessment
that takes place during a lesson in order to give both the learner and
the instructor feedback on actual understanding. See also \emph{summative
assessment}.

\textbf{\hypertarget{g:free-range-learner}{Free-range learner}\label{g:free-range-learner}}: Someone learning
outside an institutional classrooms with required homework and mandated
curriculum. (Those who use the term occasionally refer to students in
classrooms as ``battery-farmed learners'', but we don't, because that
would be rude.)

\textbf{\hypertarget{g:functional-programming}{Functional programming}\label{g:functional-programming}}: A style
of programming in which data structures cannot be modified once they
have been created, and in which functions that operate on other
functions are widely used for abstraction.

\textbf{\hypertarget{g:fuzz-testing}{Fuzz testing}\label{g:fuzz-testing}}: A software testing technique
based on generating and submitting random data.

\textbf{\hypertarget{g:general-pedagogical-knowledge}{General pedagogical knowledge}\label{g:general-pedagogical-knowledge}}:
A person's understanding of the general principles of teaching. See also
\emph{content knowledge} and \emph{pedagogical content knowledge}.

\textbf{\hypertarget{g:growth-mindset}{Growth mindset}\label{g:growth-mindset}}: The belief that ability
comes with practice. See also \emph{fixed mindset}.

\textbf{\hypertarget{g:guided-notes}{Guided notes}\label{g:guided-notes}}: Instructor-prepared notes
that cue students to respond to key information in a lecture or
discussion.

\textbf{\hypertarget{g:hashing}{Hashing}\label{g:hashing}}: Generating a condensed pseudo-random
digital key from data; any specific input produces the same output, but
different inputs are highly likely to produce different outputs.

\textbf{\hypertarget{g:hypercorrection}{Hypercorrection effect}\label{g:hypercorrection}}: The more
strongly someone believed that their answer on a test was right, the
more likely they are not to repeat the error once they discover that in
fact they were wrong.

\textbf{\hypertarget{g:implementation-science}{Implementation science}\label{g:implementation-science}}: The study
of how to translate research findings to everyday clinical practice.

\textbf{\hypertarget{g:impostor-syndrome}{Impostor syndrome}\label{g:impostor-syndrome}}: A feeling of
insecurity about one's accomplishments that manifests as a fear of being
exposed as a fraud.

\textbf{\hypertarget{g:inclusivity}{Inclusivity}\label{g:inclusivity}}: Working actively to include
people with diverse backgrounds and needs.

\textbf{\hypertarget{g:inquiry-based-learning}{Inquiry-based learning}\label{g:inquiry-based-learning}}: The
practice of allowing learners to ask their own questions, set their own
goals, and find their own path through a subject.

\textbf{\hypertarget{g:instructional-design}{Instructional design}\label{g:instructional-design}}: The craft of
creating and evaluating specific lessons for specific audiences. See
also \emph{educational psychology}.

\textbf{\hypertarget{g:intrinsic-motivation}{Intrinsic motivation}\label{g:intrinsic-motivation}}: Being driven
by enjoyment of a task or the satisfaction of doing it for its own sake.
See also \emph{extrinsic motivation}.

\textbf{\hypertarget{g:jugyokenkyu}{Jugyokenkyu}\label{g:jugyokenkyu}}: Literally ``lesson study'', a set
of practices that includes having teachers routinely observe one another
and discuss lessons to share knowledge and improve skills.

\textbf{\hypertarget{g:lateral-knowledge-transfer}{Lateral knowledge transfer}\label{g:lateral-knowledge-transfer}}:
The ``accidental'' transfer of knowledge that occurs when an instructor is
teaching one thing, and the learner picks up another.

\textbf{\hypertarget{g:learned-helplessness}{Learned helplessness}\label{g:learned-helplessness}}: A situation
in which people who are repeatedly subjected to negative feedback that
they have no way to escape learn not to even try to escape when they
could.

\textbf{\hypertarget{g:learner-persona}{Learner persona}\label{g:learner-persona}}: A brief description of
a typical target learner for a lesson that includes their general
background, what they already know, what they want to do, how the lesson
will help them, and any special needs they might have.

\textbf{\hypertarget{g:learning-objective}{Learning objective}\label{g:learning-objective}}: What a lesson is
trying to achieve.

\textbf{\hypertarget{g:learning-outcome}{Learning outcome}\label{g:learning-outcome}}: What a lesson
actually achieves.

\textbf{\hypertarget{g:legitimate-peripheral-participation}{Legitimate peripheral participation}\label{g:legitimate-peripheral-participation}}:
Newcomers' participation in simple, low-risk tasks that a \emph{community
of practice} recognizes as valid contributions.

\textbf{\hypertarget{g:live-coding}{Live coding}\label{g:live-coding}}: The act of teaching programming
by writing software in front of learners as the lesson progresses.

\textbf{\hypertarget{g:long-term-memory}{Long-term memory}\label{g:long-term-memory}}: The part of memory
that stores information for long periods of time. Long-term memory is
very large, but slow. See also \emph{short-term memory}.

\textbf{\hypertarget{g:marketing}{Marketing}\label{g:marketing}}: The craft of seeing things from
other people's perspective, understanding their wants and needs, and
finding ways to meet them

\textbf{\hypertarget{g:mental-model}{Mental model}\label{g:mental-model}}: A simplified representation
of the key elements and relationships of some problem domain that is
good enough to support problem solving.

\textbf{\hypertarget{g:metacognition}{Metacognition}\label{g:metacognition}}: Thinking about thinking.

\textbf{\hypertarget{g:minute-cards}{Minute cards}\label{g:minute-cards}}: A feedback technique in which
learners spend a minute writing one positive thing about a lesson (e.g.,
one thing they've learned) and one negative thing (e.g., a question that
still hasn't been answered).

\textbf{\hypertarget{g:near-transfer}{Near transfer}\label{g:near-transfer}}: Transfer of learning or
proficiency between closely-related domains, e.g., improvement in
understanding of decimals as a result of doing exercises with fractions.

\textbf{\hypertarget{g:notional-machine}{Notional machine}\label{g:notional-machine}}: A general, simplified
model of how a particular family of programs executes.

\textbf{\hypertarget{g:novice}{Novice}\label{g:novice}}: Someone who has not yet built a usable
mental model of a domain. See also \emph{competent practitioner} and
\emph{expert}.

\textbf{\hypertarget{g:pair-programming}{Pair programming}\label{g:pair-programming}}: A software
development practice in which two programmers share one computer. One
programmer (the driver) does the typing, while the other (the navigator)
offers comments and suggestions in real time. Pair programming is often
used as a teaching practice in programming classes.

\textbf{\hypertarget{g:parsons-problem}{Parsons Problem}\label{g:parsons-problem}}: An assessment technique
developed by Dale Parsons and others in which learners rearrange given
material to construct a correct answer to a question.

\textbf{\hypertarget{g:pedagogical-content-knowledge}{Pedagogical content knowledge}\label{g:pedagogical-content-knowledge}}:
(PCK) The understanding of how to teach a particular subject, i.e., the
best order in which to introduce topics and what examples to use. See
also \emph{content knowledge} and \emph{general pedagogical knowledge}.

\textbf{\hypertarget{g:peer-instruction}{Peer instruction}\label{g:peer-instruction}}: A teaching method in
which an instructor poses a question and then students commit to a first
answer, discuss answers with their peers, and commit to a (revised)
answer.

\textbf{\hypertarget{g:persistent-memory}{Persistent memory}\label{g:persistent-memory}}: see \emph{long-term
memory}.

\textbf{\hypertarget{g:personalized-learning}{Personalized learning}\label{g:personalized-learning}}:
Automatically tailoring lessons to meet the needs of individual
students.

\textbf{\hypertarget{g:plausible-distractor}{Plausible distractor}\label{g:plausible-distractor}}: A wrong or
less-than-best answer to a multiple-choice question that looks like it
could be right. See also \emph{diagnostic power}.

\textbf{\hypertarget{g:positioning}{Positioning}\label{g:positioning}}: What sets one brand apart from
other, similar brands.

\textbf{\hypertarget{g:preparatory-privilege}{Preparatory privilege}\label{g:preparatory-privilege}}: The
advantage of coming from a background that provides more preparation for
a particular learning task than others.

\textbf{\hypertarget{g:pull-request}{Pull request}\label{g:pull-request}}: A set of proposed changes to
a GitHub repository that can be reviewed, updated, and eventually
merged.

\textbf{\hypertarget{g:read-cover-retrieve}{Read-cover-retrieve}\label{g:read-cover-retrieve}}: A study
practice in which the learner covers up key facts or terms during a
first pass through material, then checks their recall on a second pass.

\textbf{\hypertarget{g:reflective-practice}{Reflective practice}\label{g:reflective-practice}}: see \emph{deliberate
practice}.

\textbf{\hypertarget{g:scaffolding}{Scaffolding}\label{g:scaffolding}}: Extra material provided to
early-stage learners to help them solve problems.

\textbf{\hypertarget{g:short-term-memory}{Short-term memory}\label{g:short-term-memory}}: The part of memory
that briefly stores information that can be directly accessed by
consciousness.

\textbf{\hypertarget{g:situated-learning}{Situated learning}\label{g:situated-learning}}: A model of learning
that focuses on people's transition from being newcomers to be accepted
members of a \emph{community of practice}.

\textbf{\hypertarget{g:split-attention-effect}{Split-attention effect}\label{g:split-attention-effect}}: The
decrease in learning that occurs when learners must divide their
attention between multiple concurrent presentations of the same
information (e.g., captions and a voiceover).

\textbf{\hypertarget{g:stereotype-threat}{Stereotype threat}\label{g:stereotype-threat}}: A situation in
which people feel that they are at risk of being held to stereotypes of
their social group.

\textbf{\hypertarget{g:subgoal-labelling}{Subgoal labelling}\label{g:subgoal-labelling}}: Giving names to the
steps in a step-by-step description of a problem-solving process.

\textbf{\hypertarget{g:summative-assessment}{Summative assessment}\label{g:summative-assessment}}: Assessment
that takes place at the end of a lesson to tell whether the desired
learning has taken place.

\textbf{\hypertarget{g:tangible-artifact}{Tangible artifact}\label{g:tangible-artifact}}: Something a learner
can work on whose state gives feedback about the learner's progress and
helps the learner diagnose mistakes.

\textbf{\hypertarget{g:test-driven-development}{Test-driven development}\label{g:test-driven-development}}: A
software development practice in which programmers write tests first in
order to give themselves concrete goals and clarify their understanding
of what ``done'' looks like.

\textbf{\hypertarget{g:think-pair-share}{Think-pair-share}\label{g:think-pair-share}}: A collaboration
method in which each person thinks individually about a question or
problem, then pairs with a partner to pool ideas, and then have one
person from each pair present to the whole group.

\textbf{\hypertarget{g:transfer-appropriate-processing}{Transfer-appropriate processing}\label{g:transfer-appropriate-processing}}:
The improvement in recall that occurs when practice uses activities
similar to those used in testing.

\textbf{\hypertarget{g:twitch-coding}{Twitch coding}\label{g:twitch-coding}}: Having a group of people
decide moment by moment or line by line what to add to a program next.

\textbf{\hypertarget{g:working-memory}{Working memory}\label{g:working-memory}}: see \emph{short-term memory}.
