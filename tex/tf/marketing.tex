\chapter{Marketing}\label{s:marketing}

People with academic or technical backgrounds often think that
\gref{g:marketing}{marketing} is about spin and misdirection.
In reality,
it's about seeing things from other people's perspective,
understanding their wants and needs,
and explaining how you can help them---in short,
how to teach them.
This chapter will look at how to use ideas from the previous chapters
to get people to understand and support what you're doing.

\section{What Are You Offering to Whom?}\label{s:marketing-what-whom}

The first step is to figure out what you are offering to whom,
i.e.,
what actually brings in the volunteers,
funding,
and other support you need to keep going.
The answer is often counter-intuitive.
For example,
most scientists think their papers are their product,
but it's actually their grant proposals,
because those are what brings in grant money \cite{Kuch2011}.
Their papers are the advertising that persuades people to fund those proposals,
just as albums are now what persuades people to buy musicians' concert tickets and t-shirts.

Suppose that your group offers weekend programming workshops
to people who are re-entering the workforce after being away for several years.
If workshop participants can pay enough to cover your costs,
then they are your customers and the workshops are the product.
If,
on the other hand,
the workshops are free or the learners are only paying a token amount to cut the no-show rate,
then your actual product may be some mix of:

\begin{itemize}

\item
  your grant proposals;

\item
  the alumni of your workshops
  that the companies sponsoring you would like to hire;

\item
  the half-page summary of your workshops in the mayor's annual report to city council
  that shows how she's supporting the local tech sector;
  or

\item
  the personal satisfaction that your volunteer instructors get from teaching.

\end{itemize}

As with lesson design (Chapter~\ref{s:process}),
the first steps in marketing are
to create personas that describe people who might be interested in what you're doing
and to figure out which of their needs you can meet.
One way to summarize the latter is to write a set of \gref{g:elevator-pitch}{elevator pitches}
aimed at different personas.
A widely-used template for these is:

\begin{quote}

  For \emph{target audience} \\
  who \emph{dissatisfaction with what's currently available} \\
  our \emph{category} \\
  provide \emph{key benefit}. \\
  Unlike \emph{alternatives} \\
  our program \emph{key distinguishing feature.}

\end{quote}

Continuing with the weekend workshop example,
we could use this pitch for participants:

\begin{quote}

  For \emph{people re-entering the workforce after being away for several years}
  who \emph{still have family responsibilities},
  our \emph{introductory programming workshops}
  provide \emph{weekend classes with on-site childcare}.
  Unlike \emph{online classes},
  our program \emph{gives participants a chance to meet people who are at the same stage of life}.

\end{quote}

\noindent
This one would be more appropriate for
decision makers at companies that might sponsor workshops:

\begin{quote}

  For \emph{a company that wants to recruit entry-level software developers}
  that \emph{is struggling to find mature, diverse candidates}
  our \emph{introductory programming workshops}
  provide \emph{a diverse pool of potential recruits}.
  Unlike \emph{college recruiting fairs},
  our program \emph{connects companies with mature candidates for entry-level positions}.

\end{quote}

If you don't know why different potential stakeholders might be interested in what you're doing,
ask them.
If you do know,
ask them anyway:
answers can change over time,
and you may discover things you previously overlooked.

Once you have these pitches,
they should drive what you put on your web site and in publicity material
to help people figure out as quickly as possible
if you and they have something to talk about.
(You probably \emph{shouldn't} copy them verbatim,
though:
many people in tech have seen this template so often that
their eyes will glaze over if they encounter it again.)

As you are writing these pitches,
remember that there are many reasons to learn how to program
(Section~\ref{s:intro-exercises}).
A sense of accomplishment,
control over their own lives,
and being part of a community may motivate people more than money
(Chapter~\ref{s:motivation}).
They might volunteer to teach with you because their friends are doing it;
similarly,
a company may say that they're sponsoring classes for economically disadvantaged high school students
because they want a larger pool of potential employees further down the road,
but the CEO might actually be doing it simply because it's the right thing to do.

\section{Branding and Positioning}\label{s:marketing-branding}

A \gref{g:brand}{brand} is someone's first reaction to a mention of a product;
if the reaction is ``what's that?'',
you don't have a brand (yet).
Branding is important because
people aren't going to help something they don't know about or don't care about.

Most discussion of branding today focuses on
how to build awareness online.
Mailing lists,
blogs,
and Twitter all give you ways to reach people,
but as the volume of misinformation increases,
people pay less attention to each individual interruption.
This makes \gref{g:positioning}{positioning} ever more important.
Sometimes called ``differentiation'',
it is what sets your offering apart from others---the ``unlike'' section of your elevator pitches.
When you reach out to people who are already familiar with your field,
you should emphasize your positioning,
since it's what will catch their attention.

There are other things you can do to help build your brand.
One is to use props
like a robot that one of your students made from scraps she found around the house
or the website another student made for his parents' retirement home.
Another is to make a short video---no more than a few minutes long---that showcases
the backgrounds and accomplishments of your students.
The aim of both is to tell a story:
while people always ask for data,
they believe and remember stories.

\begin{aside}{Foundational Myths}
  One of the most compelling stories a person or group can tell is
  why and how they got started.
  Are you teaching what you wish someone had taught you but didn't?
  Was there one particular person you wanted to help,
  and that opened the floodgates?
  If there isn't a section on your website starting, ``Once upon a time,''
  think about adding one.
\end{aside}

One crucial step is to make your organization findable in online searches.
\cite{DiSa2014b} discovered that
the search terms that parents used for out-of-school computing classes
didn't actually find those classes,
and many other groups face similar challenges.
There's a lot of folklore about how to make things findable
(otherwise known as \gref{g:seo}{search engine optimization} or SEO);
given Google's near-monopoly powers and lack of transparency,
most of it boils down to trying to stay one step ahead of
algorithms designed to prevent people from gaming rankings.

Unless you're very well funded,
the best you can do is to search for yourself and your organization on a regular basis
and see what comes up,
then read \hreffoot{https://moz.com/learn/seo/on-page-factors}{these guidelines}
and do what you can to improve your site.
Keep \hreffoot{https://xkcd.com/773/}{this XKCD cartoon} in mind:
people don't want to know about your org chart or get a virtual tour of your site---they want your address,
parking information,
and some idea of what you teach,
when you teach it,
and how it's going to change their lives.

\begin{aside}{Not Everyone Lives Online}
  These examples assume people have access to the internet
  and that groups have money, materials, free time, and/or technical skills.
  Many don't---in fact,
  those serving economically disadvantaged groups almost certainly don't.
  (As Rosario Robinson says, ``Free works for those that can afford free.'')
  Stories are more important than course outlines in those situations
  because they are easier to retell.
  Similarly,
  if the people you hope to reach are not online as often as you,
  then notice boards in schools,
  local libraries,
  drop-in centers,
  and grocery stores may be the most effective way to reach them.
\end{aside}

\begin{aside}{Referrals}
  As discussed in Chapter~\ref{s:community},
  building alliances with other groups that are doing things related to your work
  pays off in many ways.
  One of those is referrals:
  if someone who approaches you for help would be better served by some other organization,
  take a moment to make an introduction.
  If you've done this several times,
  add something to your website to help the next person find what they need.
  The organizations you are helping will soon start to help you in return.
\end{aside}

\section{The Art of the Cold Call}\label{s:marketing-cold-call}

Building a web site and hoping that people find it is easy;
calling people up or knocking on their door without any sort of prior introduction
is much harder.
As with standing up and teaching,
though,
it's a craft that can be learned.
Here are ten simple rules for talking people into things:

\begin{description}

  \item[1: Don't.]
    If you have to talk someone into something,
    odds are that they don't really want to do it.
    Respect that:
    it's almost always better in the long run to leave some particular thing undone
    than to use guilt or any underhanded psychological tricks that will only engender resentment.

  \item[2: Be kind.]
    I don't know if there actually is a book called
    ``Secret Tricks of the Ninja Sales Masters'',
    but if there is,
    it probably tells readers that doing something for a potential customer
    that creates a sense of obligation,
    which in turn increases the odds of a sale.
    That may work, but it only works once and it's a skeezy thing to do.
    On the other hand,
    if you are genuinely kind
    and help other people because it's what good people do,
    you just might inspire them to be good people too.

  \item[3: Appeal to the greater good.]
    If you open by talking about what's in it for them,
    you are signalling that they should think of their interaction with you
    as a commercial exchange of value to be bargained over.
    Instead,
    start by explaining how whatever you want them to help with is going to make the world a better place,
    and \emph{mean it}.
    If what you're proposing isn't going to make the world a better place,
    propose something better.

  \item[4: Start small.]
    Most people are understandably reluctant to dive into things head-first,
    so give them a chance to test the waters
    and to get to know you and everyone else involved in
    whatever it is you want help with.
    Don't be surprised or disappointed if that's where things end:
    everyone is busy or tired or has projects of their own,
    or maybe just has a different mental model of how collaboration is supposed to work.
    Remember the 90-9-1 rule---90\% of people will watch,
    9\% will speak up,
    and 1\% will actually do things---and set your expectations accordingly.

  \item[5: Don't build a project: build a community.]
    I used to belong to a baseball team that never actually played baseball:
    our ``games'' were just an excuse for us to hang out and enjoy each other's company.
    You probably don't want to go quite that far,
    but sharing a cup of tea with someone or celebrating the birth of their first grandchild
    can get you things that no reasonable amount of money can.

  \item[6: Establish a point of connection.]
    ``I was speaking to X'' or ``we met at Y'' gives them context,
    which in turn makes them more comfortable.
    This must be specific:
    spammers and cold-callers have trained us all to ignore anything that starts,
    ``I recently came across your website{\ldots}''

  \item[7: Be specific about what you are asking for.]
    People need to know this
    so that they can figure out whether the time and skills they have
    are a match for what you need.
    Being realistic up front is also a sign of respect:
    if you tell people you need a hand moving a few boxes
    when you're actually packing up an entire house,
    they're probably not going to help you a second time.

  \item[8: Establish your credibility.]
    Mention your backers,
    your size,
    how long your group has been around, or something that you've accomplished in the past
    so that they'll believe you're worth taking seriously.

\item[9: Create a slight sense of urgency.]
  ``We're hoping to launch this in the spring'' is more likely to get a positive response
  than ``We'd eventually like to launch this.''
  However, the word ``slight'' is important:
  if your request is urgent,
  most people will assume you're disorganized or that something has gone wrong
  and may then err on the side of prudence.

\item[10: Take a hint.]
  If the first person you ask for help says no,
  ask someone else.
  If the fifth or the tenth person says no,
  ask yourself if what you're trying to do makes sense and is worth doing.

\end{description}

The email template follows all of these rules.
It has worked pretty well:
we found that about half of emails were answered,
about half of those wanted to talk more,
and about half of those led to workshops,
which means that 10--15\% of targeted emails turned into workshops.
That can still be pretty demoralizing if you're not used to it,
but is much better than the 2--3\% response rate most organizations expect with cold calls.

\begin{quote}

  \noindent
  Hi NAME,

  I hope you don't mind mail out of the blue,
  but I wanted to follow up on our conversation at VENUE
  to see if you would be interested having us run an instructor training workshop---we're
  scheduling the next batch over the next couple of weeks.

  This one-day workshop will teach your volunteers
  a handful of practical, evidence-based teaching practices.
  It has been run over a hundred times in various forms on six continents
  for non-profit organizations, libraries, and companies,
  and all of the material is freely available online at http://teachtogether.tech.
  Topics will include:

  \begin{itemize}
  \item learner personas
  \item differences between different kinds of learners
  \item using formative assessment to diagnose misunderstandings
  \item teaching as a performance art
  \item what motivates and demotivates adult learners
  \item the importance of inclusivity and how to be a good ally
  \end{itemize}
  
  If this sounds interesting,
  please give me a shout---I'd welcome a chance to talk ways and means.

  Thanks,

  NAME

\end{quote}

\section{A Final Thought}\label{s:marketing-final}

As \cite{Kuch2011} says,
if you can't be first in a category,
try to create a new category that you can be first in.
If you can't do that,
join an existing group or think about doing something else entirely.
This isn't defeatist:
if someone else is already doing what you have in mind,
you should either chip in or tackle one of the other equally useful things
you could be doing instead.

\section{Exercises}\label{s:marketing-exercises}

\subsection*{Write an Elevator Pitch for a City Councillor (individual/10)}

This chapter described an organization
that offers weekend programming workshops for people re-entering the workforce.
Write an elevator pitch for that organization
aimed at a city councillor whose support the organization needs.

\subsection*{Write Elevator Pitches for Your Organization (individual/30)}

Identify two groups of people your organization needs support from
and write an elevator pitch aimed at each one.

\subsection*{Email Subjects (pairs/10)}

Write the subject lines (and only the subject lines) for three email messages:
one announcing a new course,
one announcing a new sponsor,
and one announcing a change in project leadership.
Compare your subject lines to a partner's
and see if you can merge the best features of each while also shortening them.

\subsection*{Handling Passive Resistance (small groups/30)}

People who don't want change will sometimes say so out loud,
but may also often use various forms of passive resistance,
such as just not getting around to it over and over again,
or raising one possible problem after another
to make the change seem riskier and more expensive than it's actually likely to be
\cite{Scot1987}.
Working in small groups,
list three or four reasons why people might not want your teaching initiative to go ahead,
and explain what you can do with the time and resources you have to counteract each.

\subsection*{Why Learn to Program? (individual/15)}

Revisit the ``Why Learn to Program?'' exercise in Section~\ref{s:intro-exercises}.
Where do your reasons for teaching and your learners' reasons for learning align?
Where do they not?
How does that affect your marketing?

\subsection*{Conversational Programmers (think-pair-share/15)}

A \gref{g:conversational-programmer}{conversational programmer}
is someone who needs to know enough about computing
to have a meaningful conversation with a programmer,
but isn't going to program themselves.
\cite{Wang2018} found that most learning resources don't address this group's needs.
Working in pairs,
write a pitch for a half-day workshop intended to help people that fit this description
and then share your pair's pitch with the rest of the class.
