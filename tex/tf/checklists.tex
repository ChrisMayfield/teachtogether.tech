\chapter{Checklists and Templates}\label{s:checklist}

\cite{Gawa2007} popularized the idea that using checklists can save
lives (and make many other things better too). The results of recent
studies have been more nuanced \cite{Avel2013,Urba2014}, but we still
find them useful, particularly when bringing new instructors onto a
team.

\section{Lesson Design}\label{s:checklists-design}

Designing a good course is as hard as designing good software. To help
you, this appendix summarizes a process based on evidence-based teaching
practices:

\begin{itemize}
\item
  It lays out a step-by-step progression to help you figure out what
  to think about in what order.
\item
  It provides spaced deliverables so you can re-scope or redirect
  effort without too many unpleasant surprises.
\item
  Everything from Step 2 onward goes into your final course, so there
  is no wasted effort.
\item
  Writing sample exercises early lets you check that everything you
  want your students to do actually works.
\end{itemize}

This backward design process was developed independently by
\cite{Wigg2005,Bigg2011,Fink2013}. We have
slimmed it down by removing steps related to meeting curriculum
guidelines and other institutional requirements.

Note that the steps are described in order of increasing detail, but the
process itself is always iterative. You will frequently go back to
revise earlier work as you learn something from your answer to a later
question or realize that your initial plan isn't going to play out the
way you first thought.

\subsection{Brainstorming}

The first step is to throw together some rough ideas so that you and
your colleagues can make sure your thoughts about the course are
aligned. To do this, write some point-form answers to three or four of
the questions listed below. You aren't expected to answer all of them,
and you may pose and answer others if you think it's helpful, but you
should always include a couple of answers to the first.

\begin{enumerate}
\item
  What problem(s) will student learn how to solve?
\item
  What concepts and techniques will students learn?
\item
  What technologies, packages, or functions will students use?
\item
  What terms or jargon will you define?
\item
  What analogies will you use to explain concepts?
\item
  What heuristics will help students understand things?
\item
  What mistakes or misconceptions do you expect?
\item
  What datasets will you use?
\end{enumerate}

You may not need to answer every question for every course, and you will
often have questions or issues we haven't suggested, but couple of hours
of thinking at this stage can save days of rework later on.

\textbf{Deliverable:} a rough scope for the course that you have agreed with
your colleagues.

\subsection{Who Is This Course For?}

``Beginner'' and ``expert'' mean different things to different people, and
many factors besides pre-existing knowledge influence who a course is
suitable for. The second step in designing a course is therefore to
figure out who your audience is. To do this, you should either create
some learner personas (Section~\ref{s:process-personas}), or (preferably)
reference ones that you and your colleagues have drawn up together.

After you are done brainstorming, you should go through these personas
and decide which of them your course is intended for, and how it will
help them. While doing this, you should make some notes about what
specific prerequisite skills or knowledge you expect students to have
above and beyond what's in the persona.

\textbf{Deliverable:} brief summaries of who your course will help and how.

\subsection{What Will Learners Do Along the Way?}

The best way to make the goals in Step 1 firmer is to write full
descriptions of a couple of exercises that students will be able to do
toward the end of the course. Writing exercises early is directly
analogous to \href{https://en.wikipedia.org/wiki/Test-driven\_development}{test-driven development}: rather than working
forward from a (probably ambiguous) set of learning objectives,
designers work backward from concrete examples of where their students
are going. Doing this also helps uncover technical requirements that
might otherwise not be found until uncomfortably late in the lesson
development process.

To complement the full exercise descriptions, you should also write
brief point-form descriptions of one or two exercises per lecture hour
to show how quickly you expect learners to progress. (Again, these serve
as a good reality check on how much you're assuming, and help uncover
technical requirements.) One way to create these ``extra'' exercises is to
make a point-form list of the skills needed to solve the major exercises
and create an exercise that targets each.

\textbf{Deliverable:} 1--2 fully explained exercises that use the skills the
student is to learn, plus half a dozen point-form exercise outlines.

\textbf{Note:} be sure to include solutions with example code so that you can
check that your software can do everything you need.

\subsection{How Are Concepts Connected?}

In this stage, you put the exercises in a logical order then derive a
point-form course outline for the entire course from them. This is also
when you will consolidate the datasets your formative assessments have
used.

\textbf{Deliverable:} a course outline.

\textbf{Notes:}

\begin{itemize}
\item
  The final outline should be at the lecture and formative assessment
  level, e.g., one major bullet point for each hour of work with 3--4
  minor bullet points for the episodes in that hour.
\item
  It's common to change assessments in this stage so that they can
  build on each other.
\item
  You are likely to discover things you forgot to list earlier during
  this stage, so don't be surprised if you have to double back a few
  times.
\end{itemize}

\subsection{Course Overview}

You can now write a course overview consisting of:

\begin{itemize}
\item
  a one-paragraph description (i.e., a sales pitch to students)
\item
  half a dozen learning objectives
\item
  a summary of prerequisites
\end{itemize}

Doing this earlier often wastes effort, since material is usually added,
cut, or moved around in earlier steps.

\textbf{Deliverable:} course description, learning objectives, and
prerequisites.

\section{Teaching Evaluation}\label{s:checklists-teacheval}

This rubric is designed to assess 5--10 minute recordings of people
teaching with slides, live coding, or a mix of both. You can use it as
a starting point for creating a rubric of your own. Rate each item as
``Yes'', ``Iffy'', ``No'', or ``Not Applicable''.

\begin{itemize}
\item
  Opening

  \begin{itemize}
  \item
    Exists (use N/A for other responses if not)
  \item
    Good length (10--30 seconds)
  \item
    Introduces self
  \item
    Introduces topics to be covered
  \item
    Describes prerequisites
  \end{itemize}
\item
  Content

  \begin{itemize}
  \item
    Clear goal/narrative arc
  \item
    Inclusive language
  \item
    Authentic tasks/examples
  \item
    Teaches best practices/uses idiomatic code
  \item
    Steers a path between the Scylla of jargon and the Charybdis of over-simplification
  \end{itemize}
\item
  Delivery

  \begin{itemize}
  \item
    Clear, intelligible voice (use ``Iffy'' or ``No'' for strong accent)
  \item
    Rhythm: not too fast or too slow, no long pauses or self-interruption, not obviously reading from a script
  \item
    Self-assured: does not stray into the icky tarpit of uncertainty or the dungheap of condescension
  \end{itemize}
\item
  Slides

  \begin{itemize}
  \item
    Exist (use N/A for other responses if not)
  \item
    Slides and speech complement one another (dual coding)
  \item
    Readable fonts and colors/no overwhelming slabs of text
  \item
    Frequent change (something happens on screen at least every 30 seconds)
  \item
    Good use of graphics
  \end{itemize}
\item
  Live Coding

  \begin{itemize}
  \item
    Used (use N/A for other responses if not)
  \item
    Code and speech complement one another (i.e., instructor doesn't just read code aloud)
  \item
    Readable fonts and colors/right amount of code on the screen at a time
  \item
    Proficient use of tools
  \item
    Highlights key features of code
  \item
    Dissects errors
  \end{itemize}
\item
  Closing

  \begin{itemize}
  \item
    Exists (use N/A for other responses if it doesn't)
  \item
    Good length (10--30 seconds)
  \item
    Summarizes key points
  \item
    Outlines next steps
  \end{itemize}
\item
  Overall

  \begin{itemize}
  \item
    Points clearly connected/logical flow
  \item
    Make the topic interesting (i.e., not boring)
  \item
    Knowledgeable
  \end{itemize}
\end{itemize}

\section{Teamwork Evaluation}\label{s:checklists-teameval}

This rubric is designed to assess individual performance within a
team. You can use it as a starting point for creating a rubric of
your own. Rate each item as ``Yes'', ``Iffy'', ``No'', or ``Not Applicable''.

\begin{itemize}
\item
  Communication

  \begin{itemize}
  \item
    Listens attentively to others without interrupting.
  \item
    Clarifies with others have said to ensure understanding.
  \item
    Articulates ideas clearly and concisely.
  \item
    Gives good reasons for ideas.
  \item
    Wins support from others.
  \end{itemize}
\item
  Decision Making

  \begin{itemize}
  \item
    Analyzes problems from different points of view.
  \item
    Applies logic in solving problems.
  \item
    Offers solutions based on facts rather than ``gut feel'' or intuition.
  \item
    Solicits new ideas from others.
  \item
    Generates new ideas.
  \item
    Accepts change.
  \end{itemize}
\item
  Collaboration

  \begin{itemize}
  \item
    Acknowledges issues that the team needs to confront and resolve.
  \item
    Encourages ideas and opinions even when they differ from his/her own.
  \item
    Works toward solutions and compromises that are acceptable to all involved.
  \item
    Shares credit for success with others.
  \item
    Encourages participation among all participants.
  \item
    Accepts criticism openly and non-defensively.
  \item
    Cooperates with others.
  \end{itemize}
\item
  Self-Management

  \begin{itemize}
  \item
    Monitors progress to ensure that goals are met.
  \item
    Puts top priority on getting results.
  \item
    Defines task priorities for work sessions.
  \item
    Encourages others to express their views even when they are contrary.
  \item
    Stays focused on the task during meetings.
  \item
    Uses meeting time efficiently.
  \item
    Suggests ways to proceed during work sessions.
  \end{itemize}
\end{itemize}

\section{Event Setup}\label{s:checklists-events}

The checklists below are used before, during, and after instructor
training events, and can easily be adapted for end-learner workshops as
well. We recommend that every group build and maintain its own
checklists customized for its instructors' and learners' needs.

\subsection{Scheduling the Event}

\begin{enumerate}
\item
  Decide if it will be in person, online for one site, or online for
  several sites.
\item
  Talk through expectations with the host(s) and make sure that
  everyone agrees on who is covering travel costs.
\item
  Determine who is allowed to take part: is the event open to all
  comers, restricted to members of one organization, or something in
  between?
\item
  Arrange instructors.
\item
  Arrange space, including breakout rooms if needed.
\item
  Choose dates. If it is in person, book travel.
\item
  Get names and email addresses of attendees from host(s).
\item
  Make sure they are added to the registration system.
\end{enumerate}

\subsection{Setting Up}

\begin{enumerate}
\item
  Set up a web page with details on the workshop, including date,
  location, and a list of what participants need to bring.
\item
  Check whether any attendees have special needs.
\item
  If the workshop is online, test the video conferencing link.
\item
  Make sure attendees will all have network access.
\item
  Create an Etherpad or Google Doc for shared notes.
\item
  Email attendees a welcome message that includes a link to the
  workshop home page, background readings, and a description of any
  prerequisite tasks.
\end{enumerate}

\subsection{At the Start of the Event}

\begin{enumerate}
\item
  Remind everyone of the code of conduct.
\item
  Collect attendance.
\item
  Distribute sticky notes.
\item
  Collect any relevant online account IDs.
\end{enumerate}

\subsection{At the End of the Event}

\begin{enumerate}
\item
  Update attendance records. Be sure to also record who participated
  as an instructor or helper.
\item
  Administer a post-workshop survey.
\item
  Update the course notes and/or checklists.
\end{enumerate}

\subsection{Travel Kit}

Here are a few things instructors take with them when they travel to
teach:

\begin{itemize}
\item
  sticky notes
\item
  cough drops
\item
  comfortable shoes
\item
  a small notepad
\item
  a spare power adapter
\item
  a spare shirt
\item
  deodorant
\item
  a variety of video adapters
\item
  laptop stickers
\item
  a toothbrush or some mouthwash
\item
  a granola bar or some other emergency snack
\item
  Eno or some other antacid (because road food)
\item
  business cards
\item
  a printed copy of the notes, or a tablet or other device
\item
  an insulated cup for tea/coffee
\item
  spare glasses/contacts
\item
  a notebook and pen
\item
  a portable WiFi hub (in case the room's network isn't working)
\item
  extra whiteboard markers
\item
  a laser pointer
\item
  a packet of wet wipes (because spills happen)
\item
  USB drives with installers for various operating systems
\item
  running shoes, a bathing suit, a yoga mat, or whatever else you
  exercise in or with
\end{itemize}
