\chapter{Marketing}\label{s:marketing}

\begin{objectives}

\item
  Explain what marketing actually is.

\item
  Explain the value of what they are offering to different potential
  stakeholders.

\item
  State what a brand is and what their organization's is.

\end{objectives}

It's hard to get people with academic or technical backgrounds to take
\glossref{g:marketing}{marketing} seriously, not least because it's
perceived as being about spin and misdirection. In reality, it is the
craft of seeing things from other people's perspective, understanding
their wants and needs, and finding ways to meet them. This should
sound familiar: many of the techniques introduced in
\chapref{s:process} do exactly this for lessons. This chapter will
look at how to apply similar ideas to the larger problem of getting
people to understand and support what you're doing.

\section{What Are You Offering to Whom?}\label{s:marketing-what-whom}

The first step is to figure out what you are offering to whom, i.e.,
what actually brings in the volunteers, funding, and other support you
need to keep going.  As \cite{Kuch2011} points out, the answer is
often counter-intuitive. For example, most scientists think their
papers are their product, but it's actually their grant proposals,
because those are what brings in money. Their papers are the
advertising that persuades people to fund those proposals, just as
albums are now what persuades people to buy musicians' concert tickets
and t-shirts.

You may not be a scientist, so suppose instead that your group is
offering weekend programming workshops to people who are re-entering
the workforce after taking several years out to look after young
children.  If your learners are paying enough for your workshops to
cover your costs, then the learners are your customers and the
workshops are the product. If, on the other hand, the workshops are
free, or the learners are only paying a token amount (to cut the
no-show rate), then your actual product may be some mix of:

\begin{itemize}

\item
  your grant proposals,

\item
  the alumni of your workshops that the companies sponsoring you would
  like to hire,

\item
  the half page summary of your work in the mayor's annual report to
  city council that shows how she's supporting the local tech sector, or

\item
  the personal satisfaction that your volunteer instructors get from
  teaching.

\end{itemize}

As with the lesson design process in \chapref{s:process}, you should
try to create personas to describe people who might be interested in
what you're doing and figure out which of their needs your program
will meet. You should also write a set of
\glossref{g:elevator-pitch}{elevator pitches}, each aimed at a
different potential stakeholder. A widely-used template for these
pitches looks like this:

\begin{enumerate}
\item
  For \emph{target audience}
\item
  who \emph{dissatisfaction with what's currently available}
\item
  our \emph{category}
\item
  provide \emph{key benefit}.
\item
  Unlike \emph{alternatives}
\item
  our program \emph{key distinguishing feature.}
\end{enumerate}

\noindent
Continuing with the weekend workshop example, we could use this
pitch for participants:

\begin{quote}

  For \emph{people re-entering the workforce after taking time out to
    raise children} who \emph{still have regular childcare
    responsibilities}, our \emph{introductory programming workshops}
  provide \emph{weekend classes with on-site childcare}. Unlike
  \emph{online classes}, our program \emph{gives participants a chance
    to meet people who are at the same stage of life}.

\end{quote}

\noindent
but this one for companies that we want to donate staff time for
teaching:

\begin{quote}

  For \emph{a company that wants to recruit entry-level software
    developers} that \emph{is struggling to find mature, diverse
    candidates} our \emph{introductory programming workshops} provide
  \emph{a pool of potential recruits in their thirties that includes
    large numbers of people from underrepresented groups}. Unlike
  \emph{college recruiting fairs}, our program \emph{connects
    companies directly with a diverse audience}.

\end{quote}

If you don't know why different potential stakeholders might be
interested in what you're doing, ask them. If you do know, ask them
anyway: answers can change over time, and it's a good way to discover
things that you might have missed.

Once you have written these pitches, you should use them to drive what
you put on your organization's web site and in other publicity
material, since it will help people figure out as quickly as possible
whether you and they have something to talk about.  (You probably
\emph{shouldn't} copy them verbatim, since many people in tech have
seen this template so often that their eyes will glaze over if they
encounter it again.)

As you are writing these pitches, remember that people are not just
economic animals. A sense of accomplishment, control over their own
lives, and being part of a community motivates them just as much as
money. People may volunteer to teach with you because their friends
are doing it; similarly, a company may say that they're sponsoring
classes for economically disadvantaged high school students because
they want a larger pool of potential employees further down the road,
but the CEO might actually be doing it simply because it's the right
thing to do.

\section{Branding and Positioning}\label{s:marketing-branding}

A \glossref{g:brand}{brand} is someone's first reaction to a mention
of a product; if the reaction is ``what's that?'', you don't have a
brand yet.  Branding is important because people aren't going to help
with something they don't know about or don't care about.

Most discussion of branding today focuses on ways to build awareness
online. Mailing lists, blogs, and Twitter all give you ways to reach
people, but as the volume of (mis)information steadily increases, the
attention people pay to each interruption decreases. As this happens,
\glossref{g:positioning}{positioning} becomes more important.
Sometimes called ``differentiation'', it is what sets your offering
apart from others, i.e., it's the ``unlike'' section of your elevator
pitches.  When you reach out to people who are already familiar with
your field, you should emphasize your positioning, since it's what
will catch their attention.

There are other things you can do to help build your brand.  One is to
use props: a robot car that one of your students made from scraps she
found around the house, the website another student made for his
parents' retirement home, or anything else that makes what you're
doing seem real. Another is to make a short video---no more than a few
minutes long---showcasing the backgrounds and accomplishments of your
students.  The aim of both is to tell a story: while people always ask
for data, stories are what they believe.

Notice, though that these examples assume people have access to the
money, materials, and/or technology needed to create these products.
Many don't---in fact, those serving economically disadvantaged groups
almost certainly don't.  As Rosario Robinson says, ``Free works for
those that can afford free.''  In those situations, stories become
even more important, because they can be shared and re-shared without
limit.

\begin{callout}{Foundational Myths}

  One of the most compelling stories a person or organization can tell
  is why and how they got started. Are you teaching what you wish
  someone had taught you but didn't? Was there one particular person
  you wanted to help, and that opened the floodgates?  If there isn't
  a section on your website starting, ``Once upon a time,'' think
  about adding one.

\end{callout}

Whatever else you do, make your organization findable in online
searches: \cite{DiSa2014b} discovered that the search terms parents
were likely to use for out-of-school computing classes didn't actually
find those classes.  There's a lot of folklore about how to make
things findable under the label ``SEO'' (for ``search engine
optimization''); given Google's near-monopoly powers and lack of
transparency, most of it boils down to trying to stay one step ahead
of algorithms designed to prevent people from gaming rankings.

Unless you're very well funded, the best you can do is to search for
yourself and your organization on a regular basis and see what comes
up, then read \href{https://moz.com/learn/seo/on-page-factors}{these
  guidelines from Moz} and do what you can to improve your site. Keep
\href{https://xkcd.com/773/}{this cartoon} in mind: people don't
(initially) want to know about your org chart or get a virtual tour of
your site; they want your address, parking information, and above all,
some idea of what you teach, when you teach it, how to get in touch,
and how it's going to change their life.

Offline findability is equally important for new organizations. Many
of the people you hope to reach might not be online as often as you,
and some won't be online at all.  Notice boards in schools, local
libraries, drop-in centers, and grocery stores are still an effective
way to reach them.

\begin{callout}{Build Alliances}

  As discussed in \chapref{s:community}, building alliances with other
  groups that are doing things related to what you're doing pays off
  in many ways. One of those is referrals: if someone approaches you
  for help, but would be better served by some other organization,
  take a moment to make an introduction. If you've done this several
  times, add something to your website to help the next person find
  what they need. The organizations you are helping will soon start to
  help you in return.

\end{callout}

\section{The Art of the Cold Call}\label{s:marketing-cold-call}

Building a web site and hoping that people find it is one thing;
calling people up or knocking on their door without any sort of prior
introduction is another. As with standing up and teaching, though,
it's a craft that can be learned like any other, and there are a few
simple rules you can follow:

\begin{description}

\item[Establish a point of connection] such as ``I was speaking to X''
  or ``You attended bootcamp Y''. This must be specific: spammers and
  headhunters have trained us all to ignore anything that starts, ``I
  recently read your website''.

\item[Create a slight sense of urgency] by saying something like,
  ``We're booking workshops right now.''  Be cautious with this,
  though; as with the previous recommendation, the web's race to the
  bottom has conditioned people to discount anything that sounds like
  a hustle.

\item[Explain how you are going to help make their lives better.]  A
  pitch like ``Your students will be able to do their math homework
  much faster if you let us tutor them'' is a good attention-getter.

\item[Be specific about what you are offering.] ``Our usual two-day
  curriculum includes{\ldots}'' helpers listeners figure out right
  away whether a conversation is worth pursuing.

\item[Make yourself credible] by mentioning your backers, your size,
  how long you've been around, or your instructors's backgrounds.

\item[Tell them what your terms are.]  Do you charge money? Do they
  need to cover instructors' travel costs?  Can they reserve seats for
  their own staff?

\item[Write a good subject line.]  Keep it short, avoid ALL CAPS,
  words like ``sale'' or ``free'' (which increase the odds that your
  message will be treated as spam), and never! use! exclamation!
  marks!

\item[Keep it short,] since the purest form of respect is to treat
  other people as if their time was as valuable as your own.

\end{description}

The email template below puts all of these points in action.  It has
worked pretty well, but ``pretty well'' is relative: we found that
about half of emails were answered, about half of those answers were,
``Sure, let's talk more,'' and about half of those led to workshops,
which means that 10--15\% of targeted emails to people we had some
sort of connection with turned into workshops.  That's much better
than the 2--3\% response rate most organizations expect with cold
calls, but can still be pretty demoralizing if you're not used to it.

\begin{callout}{Mail Out of the Blue}

  \noindent
  Hi NAME,

  I hope you don't mind mail out of the blue, but I wanted to follow
  up on our conversation at the tech showcase last week to see if you
  would be interested having us run an instructor training workshop -
  we're scheduling the next batch over the next couple of weeks.

  This one-day class will introduce your volunteer teachers to a
  handful of key practices that are grounded in education research and
  proven useful in practice. The class has been delivered dozens of
  times on four continents, and will be hands-on: short lessons will
  alternate with individual and group practical exercises, including
  practice teaching sessions.

  If this sounds interesting, please give me a shout - I'd welcome a
  chance to talk ways and means.

  Thanks,\\
  NAME

\end{callout}

\section{A Final Thought}\label{s:marketing-final}

As \cite{Kuch2011} says, if you can't be first in a category, create a
new category that you can be first in; if you can't do that, join an
existing group or think about doing something else entirely. This
isn't defeatist: if someone else is already doing what you're doing
better than you, there are probably lots of other equally useful
things you could be doing instead.

\section{Exercises}\label{s:marketing-exercises}

\exercise{Write an Elevator Pitch for a City Councillor}{individual}{10}

This chapter described an organization that offers weekend programming
workshops for people re-entering the workforce after taking a break to
raise children. Write an elevator pitch for that organization aimed at
a city councillor whose support the organization needs.

\exercise{Write Elevator Pitches for Your Organization}{individual}{30}

Identify two groups of people your organization needs support from,
and write an elevator pitch aimed at each one.

\exercise{Identify Causes of Passive Resistance}{small groups}{30}

People who don't want change will sometimes say so out loud, but will
also often use various forms of passive resistance, such as just not
getting around to it over and over again, or raising one possible
problem after another to make the change seem riskier and more
expensive than it's actually likely to be. Working in small groups,
list three or four reasons why people might not want your teaching
initiative to go ahead, and explain what you can do with the time and
resources you have to counteract each.

\exercise{Why Learn to Program?}{individual}{15}

Revisit the ``Why Learn to Program?''  exercise in
\secref{s:intro-exercises}.  Where do your reasons for teaching and
your learners' reasons for learning align?  Where are they not
aligned?  How does that affect your marketing?

\exercise{Appealing to Your Learners}{think-pair-share}{15}

Adult learners are different from children and teens: in general, they
are better at managing their time, they're learning because they want
to or need to, and they bring a lot of previous experience of learning
into the room, so they tend to be better at knowing when they're
struggling productively and when they're just struggling.

Working in pairs, write a one-paragraph pitch for a class on web
design that touches on these points, and then compare your pair's
pitch with those of other pairs.

\exercise{Email Subjects}{pairs}{10}

Write the subject lines (and only the subject lines) for three email
messages: one announcing a new course, one announcing a new sponsor,
and one announcing a change in project leadership.  Compare your
subject lines to a partner's and see if you can merge the best
features of each while also shortening them.

\exercise{Conversational Programmers}{think-pair-share}{15}

A \glossref{g:conversational-programmer}{conversational programmer} is
someone who needs to know enough about computing to have a meaningful
conversation with a programmer, but isn't going to program themselves.
\cite{Wang2018} found that most learning resources don't address this
group's needs.  Working in pairs, write a pitch for a half-day
workshop intended to help people that fit this description, and then
share your pair's pitch with the rest of the class.
