\chapter{Joining Our Community}\label{s:joining}

This appendix describes how you can become part of our community by
using, sharing, and improving this material.

\section{Contributor Covenant}\label{s:joining-covenant}

The contributor covenant laid out below governs contributions to this
book.  It is adapted from the
\href{https://www.contributor-covenant.org}{Contributor Covenant}
version 1.4; please see \appref{s:conduct} for a sample code of
conduct for use in classes and other learning situations.

\subsection*{Our Pledge}

In the interest of fostering an open and welcoming environment, we as
contributors and maintainers pledge to making participation in our
project and our community a harassment-free experience for everyone,
regardless of age, body size, disability, ethnicity, gender identity
and expression, level of experience, education, socioeconomic status,
nationality, personal appearance, race, religion, or sexual identity
and orientation.

\subsection*{Our Standards}

Examples of behavior that contributes to creating a positive
environment include:

\begin{itemize}
\item
  Using welcoming and inclusive language
\item
  Being respectful of differing viewpoints and experiences
\item
  Gracefully accepting constructive criticism
\item
  Focusing on what is best for the community
\item
  Showing empathy towards other community members
\end{itemize}

Examples of unacceptable behavior by participants include:

\begin{itemize}
\item
  The use of sexualized language or imagery and unwelcome sexual
  attention or advances
\item
  Trolling, insulting/derogatory comments, and personal or political
  attacks
\item
  Public or private harassment
\item
  Publishing others' private information, such as a physical or
  electronic address, without explicit permission
\item
  Other conduct which could reasonably be considered inappropriate in
  a professional setting
\end{itemize}

\subsection*{Our Responsibilities}

Project maintainers are responsible for clarifying the standards of
acceptable behavior and are expected to take appropriate and fair
corrective action in response to any instances of unacceptable
behavior.

Project maintainers have the right and responsibility to remove, edit,
or reject comments, commits, code, wiki edits, issues, and other
contributions that are not aligned to this covenant, or to ban
temporarily or permanently any contributor for other behaviors that
they deem inappropriate, threatening, offensive, or harmful.

\subsection*{Scope}

This covenant applies both within project spaces and in public spaces
when an individual is representing the project or its
community. Examples of representing a project or community include
using an official project e-mail address, posting via an official
social media account, or acting as an appointed representative at an
online or offline event. Representation of a project may be further
defined and clarified by project maintainers.

\subsection*{Enforcement}

Instances of abusive, harassing, or otherwise unacceptable behavior
may be reported by contacting the project team at
[gvwilson@third-bit.com](mailto:gvwilson@third-bit.com).  All
complaints will be reviewed and investigated and will result in a
response that is deemed necessary and appropriate to the
circumstances. The project team is obligated to maintain
confidentiality with regard to the reporter of an incident.  Further
details of specific enforcement policies may be posted separately.

Project maintainers who do not follow or enforce the covenant in good
faith may face temporary or permanent repercussions as determined by
other members of the project's leadership.

\section{Using This Material}\label{s:joining-using}

This material has been used in many ways, from a multi-week online
class to an intensive in-person workshop.  It's usually possible to
cover large parts of Chapters~\ref{s:models}--\ref{s:process},
\chapref{s:performance}, and \chapref{s:motivation} in two long days.

\subsection*{In Person}

This is the most effective way to deliver this training, but also the
most demanding.  Participants are physically together. When they need
to practice teaching in small groups, some or all of them go to nearby
breakout spaces.  Participants use their own tablets or laptops to
view online material during the class and for shared note-taking
(\secref{s:classroom-notetaking}), and use pen and paper or
whiteboards for other exercises.  Questions and discussion are done
aloud.

If you are teaching in this format, you should use sticky notes as
status flags so that you can see who needs help, who has questions,
and who's ready to move on (\secref{s:classroom-status-flags}).  You
should also use them to distribute attention so that everyone gets a
fair share of the instructor's time (\secref{s:classroom-attention}),
and as minute cards to encourage learners to reflect on what they've
just learned and to give you actionable feedback while you still have
time to act on it (\secref{s:classroom-minute-cards}).

\subsection*{Online in Groups}

In this format, 10--40 learners are together in 2--6 groups of 4--12,
but those groups are geographically distributed.  Each group uses one
camera and microphone to connect to the video call, rather than each
person being on the call separately. We have found that having good
audio matters more than having good video, and that the better the
audio, the more learners can communicate with the instructor and other
rooms by voice rather than via text online.

The entire class does shared note-taking together, and also uses the
shared notes for asking and answering questions.  (Having several
dozen people try to talk on a call works poorly, so in most sessions,
the instructor does the talking and learners respond through the
note-taking tool's chat.)

\subsection*{Online as Individuals}

The natural extension of being online in groups is to be online as
individuals.  As with online groups, the instructor will do most of
the talking and learners will mostly participate via text chat.  Good
audio is once again more important than good video, and participants
should use text chat to signal that they want to speak next
(\appref{s:meetings}).

Having participants online individually makes it more difficult to
draw and share concept maps (\secref{s:memory-exercises}) or give
feedback on teaching (\secref{s:performance-exercises}).  Instructors
should therefore rely more on exercises with written results that can
be put in the shared notes, such as giving feedback on stock videos of
people teaching.

\subsection*{Multi-Week Online}

This was the first format used, and I no longer recommend it: while
spreading the class out gives people time to reflect and tackle larger
exercises, it also greatly increases the odds that they'll have to
drop out because of other demands on their time.

The class meets every week for an hour via video conferencing. Each
meeting may be held twice to accommodate learners' time zones and
schedules.  Participants use shared note-taking as described above for
online group classes, post homework online between classes, and
comment on each other's work. (In practice, comments are relatively
rare: people strongly prefer to discuss material in the weekly
meetings.)

\section{Contributing and Maintaining}\label{s:joining-contributing}

This book is a community resource: contributions of all kinds are
welcome, from suggestions for improvements to errata and new material.
All contributors must abide by the contributor covenant presented
above; by submitting your work, you are agreeing that it may
incorporated in either original or edited form and release it under
the same license as the rest of this material (\appref{s:license}).
If your material is incorporated, we will add you to the
acknowledgments (\secref{s:intro-acknowledgments}) unless you request
otherwise.

\begin{itemize}

\item
  The source for this book is stored on GitHub at {\repository}.  If
  you know how to use Git and GitHub and would like to change, fix, or
  add something, please submit a \glossref{g:pull-request}{pull
    request} that modifies the LaTeX source in the \texttt{tex}
  directory.  If you would like to preview your changes, please read
  the instructions in the \texttt{BUILD.md} file in the root directory
  of the project.

\item
  If you simply want to report an error, ask a question, or make a
  suggestion, please file an issue at {\repository}. You need to have
  a GitHub account in order to do this, but do not need to know how to
  use Git.

\item
  If you do not wish to create a GitHub account, please email your
  contribution to {\email} with either ``T3'' or ``Teaching Tech
  Together'' somewhere in the subject line.  We will try to respond
  within a week.

\end{itemize}

Please note that we also welcome improvements to our build process,
tooling, and typography, and are always grateful for more diagrams;
please see the file \texttt{BUILD.md} in the root directory of the
book's GitHub repository at {\repository} for more information.
Finally, we always enjoy hearing how people have used this material:
please let us know if you have a story you would like to share.

\subsection*{A Teaching Commons}

\secref{s:community-governance} defined a commons as something managed
jointly by a community according to rules they themselves have evolved
and adopted.  Open source software and Wikipedia are both successful
examples; the question is, why don't teachers build lessons
collaboratively in the same way?  People have proposed
\href{http://blog.mrmeyer.com/2016/why-secondary-teachers-dont-want-a-github-for-lesson-plans/}{a
  variety of reasons}, but I don't think any of them
\href{http://third-bit.com/2016/04/29/why-teachers-dont-collaborate.html}{hold
  up to close scrutiny}.

\href{http://software-carpentry.org}{Software Carpentry} is proof by
implementation that a teaching commons can produce and maintain
high-quality lessons that hundreds of people can use \cite{Wils2016}.
I hope you will choose to help us do the same for this book.  If you
are new to working this way:

\begin{description}

\item[Start small.] Fix a typo, clarify the wording of an exercise,
  correct or update a citation, or suggest a better example or
  analogy to illustrate some point.

\item[Join the conversation.] Have a look at the issues and proposed
  changes that other people have already filed and add your comments
  to them.  It's often possible to improve improvements, and it's a
  good way to introduce yourself to the community and make new
  friends.  (To make this as easy as possible, we tag some issues and
  proposed changes as ``Suitable for Newcomers'' or ``Help Wanted''.)

\item[Discuss, then edit.] If you want to propose a large change, such
  as reorganizing or splitting an entire chapter, please file an issue
  that outlines your proposal and your reasoning and tag it with
  ``Proposal''.  We encourage everyone to add comments to these issues
  so that the whole discussion of what and why is in the open and can
  be archived.  If the proposal is accepted, the actual work may then
  be broken down into several smaller issues or changes that can be
  tackled independently.

\end{description}
