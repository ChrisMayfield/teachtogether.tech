\chapter{A Lesson Design Process}\label{s:process}

\begin{objectives}

\item Describe the steps in backward lesson design and explain why it
  generally produces better lessons than the more common forward
  development process.

\item Define ``teaching to the test'' and explain why backward lesson
  design is \emph{not} the same thing.

\item Construct and critique five-part learner personas.

\item Construct good learning objectives and critique learning
  objectives with reference to Bloom's Taxonomy and/or Fink's
  Taxonomy.

\end{objectives}

Most people design lessons like this:

\begin{enumerate}

\item
  Someone asks you to teach something you haven't thought about in
  years.

\item
  You start writing slides to explain what you know about the subject.

\item
  After two or three weeks, you make up an assignment based more or
  less on what you've taught so far.

\item
  You repeat step 3 several times.

\item
  You stay awake into the wee hours of the morning to create a final
  exam and promise yourself that you'll be more organized next time.

\end{enumerate}

There's a better way, but to explain it, we first need to explain how
\glossref{g:test-driven-development}{test-driven development} (TDD) is
used in software development. Programmers who are using TDD don't
write software and then write tests to check that the software is
doing the write thing. Instead, they write the tests first, then write
just enough new software to make those tests pass, and then clean up a
bit.

TDD works because writing tests forces programmers to specify exactly
what they're trying to accomplish and what ``done'' looks like. It's
easy to be vague when using a human language like English or Korean;
it's much harder to be vague in Python or R.  TDD also reduces the
risk of endless polishing, and the risk of confirmation bias: someone
who hasn't written a program is much more likely to be objective when
testing it than its original author, and someone who hasn't written a
program \emph{yet} is more likely to test it objectively than someone
who has just put in several hours of hard work and really, really
wants to be done.

A similar backward method works very well for lesson design. This
method is something called \glossref{g:backward-design}{backward
  design}; developed independently in
\cite{Wigg2005,Bigg2011,Fink2013}, it is summarized in
\cite{McTi2013}, and in simplified form, its steps are:

\begin{enumerate}

\item
  Brainstorm to get a rough idea of what you want to cover, how you're
  going to do it, what problems or misconceptions you expect to
  encounter, what's \emph{not} going to be included, and so on. You
  may also want to draw some concept maps at this stage.

\item
  Create or recycle learner personas (discussed in the next section)
  to figure out who you are trying to teach and what will appeal to
  them.  (This step can also be done first, before the brainstorming.)

\item
  Create formative assessments that will give the learners a chance to
  practice the things they're trying to learn and tell you and them
  whether they're making progress and where they need to focus their
  work.

\item
  Put the formative assessments in order based on their complexity and
  dependencies to create a course outline.

\item
  Write just enough to get learners from one formative assessment to
  the next. Each hour in the classroom will then consist of three or
  four such episodes.

\end{enumerate}

This method helps to keep teaching focused on its objectives. It also
ensures that learners don't face anything on the final exam that the
course hasn't prepared them for.  It is \emph{not} the same thing as
``teaching to the test''. When using backward design, teachers set
goals to aid in lesson design, and may never actually give the final
exam that they wrote. In many school systems, on the other hand, an
external authority defines assessment criteria for all learners,
regardless of their individual situations.  The outcomes of those
summative assessments directly affect the teachers' pay and promotion,
which means teachers have an incentive to focus on having learners
pass test rather than on helping them learn.

\begin{callout}{Measure{\ldots}And Then?}

  \cite{Gree2014} argues that this focus on measurement is appealing
  to those with the power to set the tests, but unlikely to improve
  outcomes unless it is coupled with support for teachers to make
  improvements based on test outcomes.  The latter is often missing
  because large organizations usually value uniformity over
  productivity \cite{Scot1998}; we will return to this topic in
  \chapref{s:performance}.

\end{callout}

It's important to note that while lesson design is \emph{described} as
a sequence, it's almost never \emph{done} that way: we may, for
example, change our mind about what we want to teach based on
something that occurs to us while we're writing an MCQ, or re-assess
who we're trying to help once we have a lesson outline.  However, it's
important that the notes we leave behind to present things in the
order described above, because that's the easier way for whoever has
to use or maintain the lesson to retrace our thinking.  The same
rewriting of history is useful for the same reasons in software design
and many other fields \cite{Parn1986}.

\appref{s:v3} presents the design notes for this version of this book.
A few things have been added, dropped, or rearranged, but what you are
reading now matches the plan pretty closely.

\section{Learner Personas}\label{s:process-personas}

A key step in the lesson design process described above is figuring
out who your audience is.  One way to do this is to write two or three
\glossref{g:learner-persona}{learner personas}. This technique is
borrowed from user interface designers, who create short profiles of
typical users to help them think about their audience.

Learner personas have five parts: the person's general background,
what they already know, what \emph{they} think they want to do (as
opposed to what someone who already understands the subject thinks),
how the course will help them, and any special needs they might
have. The personas in \secref{s:intro-audience} have the five points
listed above, rearranged to flow more readably; a learner persona for
a weekend workshop aimed at college students might be:

\begin{enumerate}

\item
  Jorge has just moved from Costa Rica to Canada to study agricultural
  engineering. He has joined the college soccer team, and is looking
  forward to learning how to play ice hockey.

\item
  Other than using Excel, Word, and the Internet, Jorge's most
  significant previous experience with computers is helping his sister
  build a WordPress site for the family business back home in Costa
  Rica.

\item
  Jorge needs to measure properties of soil from nearby farms using a
  handheld device that sends logs in a text format to his computer.
  Right now, Jorge has to open each file in Excel, crop the first and
  last points, and calculate an average.

\item
  This workshop will show Jorge how to write a little Python program to
  read the data, select the right values from each file, and calculate
  the required statistics.

\item
  Jorge can read English well, but still struggles sometimes to keep
  up with spoken conversation (especially if it involves a lot of new
  jargon).

\end{enumerate}

\begin{callout}{A Gentle Reminder}

  When designing lessons, you must always remember that \emph{you are
    not your learners}. You may be younger (if you're teaching
  seniors) or wealthier (and therefore able to afford to download
  videos without foregoing a meal to pay for the bandwidth), but you
  are almost certainly more knowledgeable about technology. Don't
  assume that you know what they need or will understand: ask them,
  and pay attention to their answer. After all, it's only fair that
  learning should go both ways.

\end{callout}

Rather than writing new personas for every lesson or course, it's
common for teachers to create and share a handful that cover everyone
they are likely to teach, then pick a few from that set to describe
who particular material is intended for.  When personas are used this
way, they become a convenient shorthand for design issues: when
speaking with each other, teachers can say, ``Would Jorge understand
why we're doing this?'' or, ``What installation problems would Jorge
face?''

Brainstorming the broad outlines of what you're going to teach and
then deciding who you're trying to help is one approach; it's equally
valid to pick an audience and then brainstorm their needs.  Either
way, \cite{Guzd2016} offers the following guidance:

\begin{enumerate}

\item
  Connect to what learners know.

\item
  Keep cognitive load low.

\item
  Use authentic tasks (see \secref{s:motivation-authentic}).

\item
  Be generative and productive.

\item
  Test your ideas rather than trusting your instincts.

\end{enumerate}

Of course, one size won't fit all.  \cite{Alha2018} reported
improvement in learning outcomes and student satisfaction in a
course for students from a variety of academic backgrounds which
allowed them to choose between different domain-related assignments.
It's extra work to set up and grade, but that's manageable if the
projects are open-ended (so that they can be used repeatedly) and if
the load is shared with other teachers
(\secref{s:process-maintainability}).  Other work has shown that
building courses for science students around topics as diverse as
music, data science, and cell biology will also improve outcomes
\cite{Pete2017,Dahl2018,Ritz2018}.

\section{Learning Objectives}\label{s:process-objectives}

Formative and summative assessments help teachers figure out what
they're going to teach, but in order to communicate that to learners
and other teachers, a course description should also have
\glossref{g:learning-objective}{learning objectives}.  These help
ensure that everyone has the same understanding of what a lesson is
supposed to accomplish.  For example, a statement like ``understand
Git'' could mean any of the following, each of which would be backed
by a very different lesson:

\begin{itemize}

\item
  Learners can describe three scenarios in which version control systems
  like Git are better than file-sharing tools like Dropbox, and two in
  which they are worse.

\item
  Learners can commit a changed file to a Git repository using a desktop
  GUI tool.

\item
  Learners can explain what a detached HEAD is and recover from it using
  command-line operations.

\end{itemize}

\begin{callout}{Objectives vs.\ Outcomes}

  A learning objective is what a lesson strives to achieve. A
  \glossref{g:learning-outcome}{learning outcome} is what it actually
  achieves, i.e., what learners actually take away. The role of
  summative assessment is therefore to compare learning outcomes with
  learning objectives.

\end{callout}

A learning objective is a single sentence describing how a learner
will demonstrate what they have learned once they have successfully
completed a lesson.  More specifically, it has a \emph{measurable or
  verifiable verb} that states what the learner will do, and specifies
the \emph{criteria for acceptable performance}. Writing these kinds of
learning objectives may initially seem restrictive or limiting, but
will make you, your fellow teachers, and your learners happier in the
long run. You will end up with clear guidelines for both your teaching
and assessment, and your learners will appreciate the clear
expectations.

One way to understand what makes for a good learning objective is to
see how a poor one can be improved:

\begin{itemize}

\item
  ``The learner will be given opportunities to learn good programming
  practices.''  This describes the lesson's content, not the
  attributes of successful students.

\item
  ``The learner will have a better appreciation for good programming
  practices.''  This doesn't start with an active verb or define the
  level of learning, and the subject of learning has no context and is
  not specific.

\item
  ``The learner will understand how to program in R.''  While this
  starts with an active verb, it doesn't define the level of learning,
  and the subject of learning is still too vague for assessment.

\item
  ``The learner will write one-page data analysis scripts to read,
  filter, summarize, and print results for tabular data using R and R
  Studio.'' This starts with an active verb, defines the level of
  learning, and provides context to ensure that outcomes can be
  assessed.

\end{itemize}

When it comes to choosing verbs, many teachers use
\glossref{g:blooms-taxonomy}{Bloom's taxonomy}.  First published in
1956, it was updated at the turn of the century \cite{Ande2001}, and
is the most widely used framework for discussing levels of
understanding.  Its most recent form has six categories; the list
below defines each, and gives a few of the verbs typically used in
learning objectives written for each:

\begin{description}

\item[Remembering:] Exhibit memory of previously learned material by
  recalling facts, terms, basic concepts, and answers.
  \emph{(recognize, list, describe, name, find)}

\item[Understanding:] Demonstrate understanding of facts and ideas by
  organizing, comparing, translating, interpreting, giving
  descriptions, and stating main ideas.
  \emph{(interpret, summarize, paraphrase, classify, explain)}

\item[Applying:] Solve new problems by applying acquired knowledge,
  facts, techniques and rules in a different way.  \emph{(build,
    identify, use, plan, select)}

\item[Analyzing:] Examine and break information into parts by
  identifying motives or causes.  Make inferences and find evidence to
  support generalizations.
  \emph{(compare, contrast, simplify)}

\item[Evaluating:] Present and defend opinions by making judgments
  about information, validity of ideas, or quality of work based on a
  set of criteria.
  \emph{(check, choose, critique, prove, rate)}

\item[Creating:] Compile information together in a different way by
  combining elements in a new pattern or proposing alternative
  solutions.
  \emph{(design, construct, improve, adapt, maximize, solve)}

\end{description}

\cite{Masa2018} found that even experienced educators have trouble
agreeing on how to classify a question or idea according to Bloom's
Taxonomy, but the material in most introductory programming courses
fits into the first four of these levels; only once that material has
been mastered can learners start to think about evaluating and
creating.  (As Daniel Willingham has said, people can't think without
something to think about \cite{Will2010}.)

Another way to think about learning objectives comes from
\cite{Fink2013}, which defines learning in terms of the change it is
meant to produce in the learner.  \glossref{g:finks-taxonomy}{Fink's
  Taxonomy} also has six categories, but unlike Bloom's, they are
complementary rather than hierarchical:

\begin{description}

  \item[Foundational Knowledge:] understanding and remembering
    information and ideas.
    \emph{(remember, understand, identify)}

  \item[Application:] skills, critical thinking, managing projects.
    \emph{(use, solve, calculate, create)}

  \item[Integration:] connecting ideas, learning experiences, and real
    life.
    \emph{(connect, relate, compare)}

  \item[Human Dimension:] learning about oneself and others.
    \emph{(come to see themselves as, understand others in terms of,
      decide to become)}

  \item[Caring:] developing new feelings, interests, and values.
    \emph{(get excited about, be ready to, value)}

  \item[Learning How to Learn:] becoming a better student.
    \emph{(identify source of information for, frame useful questions
      about)}

\end{description}

\noindent
A set of learning objectives based on this taxonomy for an
introductory course on HTML and CSS might be:

\newpage % PDF

\begin{quote}
  By the end of this course, learners will be able to:

  \begin{itemize}

  \item
    Explain the difference between markup and presentation, what CSS
    properties are, and how CSS selectors work.

  \item
    Write and style a web page using common tags and CSS properties.

  \item
    Compare and contrast authoring with HTML and CSS to authoring with
    desktop publishing tools.

  \item
    Identify issues in sample web pages that would make them difficult
    for the visually impaired to interact with and provide appropriate
    corrections.

  \item
    Explain the role that JavaScript plays in styling web pages and
    want to learn more about how to use it.

  \end{itemize}
\end{quote}

\section{Maintainability}\label{s:process-maintainability}

It takes a lot of effort to create a good lesson, but once it has been
built, someone needs to maintain it, and doing that is a lot easier if
it has been built in a maintainable way.  But what exactly does
``maintainable'' mean? The short answer is that a lesson is
maintainable if it's cheaper to update it than to replace it.  This
equation depends on three factors.  The first is \emph{how well
  documented the course's design is.} If the person doing maintenance
doesn't know (or doesn't remember) what the lesson is supposed to
accomplish or why topics are introduced in a particular order, it will
take her more time to update it. One of the reasons to use the design
process described earlier in this chapter is to capture decisions
about why each course is the way it is.

The second factor is \emph{how easy it is for collaborators to
  collaborate technically.}  Teachers usually share material by
mailing PowerPoint files to each other or putting them in a shared
drive. Collaborative writing tools like
\href{http://docs.google.com}{Google Docs} and wikis are a big
improvement, as they allow many people to update the same document and
comment on other people's updates. The version control systems used by
programmers, such as \href{http://github.com}{GitHub}, are another big
advance, since they let any number of people work independently and
then merge their changes back together in a controlled, reviewable
way. Unfortunately, version control systems have a long, steep
learning curve, and (still) don't handle common office document
formats.

The third factor, which is the most important in practice, is
\emph{how willing people are to collaborate.}  The tools needed to
build a ``Wikipedia for lessons'' have been around for twenty years,
but most teachers still don't write and share lessons the way that
they write and share encyclopedia entries, even though commons-based
lesson development and maintenance actually works very well
(\secref{s:community-governance} and \secref{s:joining-contributing}).

\cite{Leak2017} interviewed 17 computer science teachers to find out
why they don't use resource sharing sites.  They found that most of
the reasons were operational.  For example, respondents said that
sites need good landing pages that ask ``what is your current role?''
and ``what course and grade level are you interested in?'', and should
display all their resources in context, since visitors may be new
teachers who are struggling to connect the dots themselves.  They also
said that sites should allow anonymous posts on discussion forums to
reduce fear of looking foolish in front of peers.

One interesting observation is that while teachers don't collaborate
at scale, they \emph{do} remix by finding other people's materials
online or in textbooks and reworking them. That suggests that the root
problem may be a flawed analogy: rather than lesson development being
like writing a Wikipedia article or some open source software, perhaps
it's more like sampling in music.

If this is true, then lessons may be the wrong granularity for
sharing, and collaboration might be more likely to take hold if the
thing being collaborated on was smaller. This fits well with
Caulfield's theory of
\href{https://hapgood.us/2016/05/13/choral-explanations/}{choral
  explanations}.  He argues that sites like
\href{https://stackoverflow.com/}{Stack Overflow} succeed because they
provide a chorus of answers for every question, each of which is most
suitable for a slightly different questioner. If Caulfield is right,
the lessons of tomorrow may include guided tours of community-curated
Q\&A repositories designed to accommodate learners at widely different
levels.

\section{Exercises}\label{s:process-exercises}

\exercise{Create Learner Personas}{small groups}{30}

Working in small groups, create a five-point persona that describes
one of your typical learners.

\exercise{Classify Learning Objectives}{pairs}{10}

Look at the example learning objectives given for an introductory
course on HTML and CSS in \secref{s:process-objectives} and classify
each according to Bloom's Taxonomy.  Compare your answers with those
of your partner: where did you agree and disagree, and why?

\exercise{Write Learning Objectives}{pairs}{20}

Write one or more learning objectives for something you currently
teach or plan to teach using Bloom's Taxonomy. Working with a partner,
critique and improve the objectives.

\exercise{Write More Learning Objectives}{pairs}{20}

Write one or more learning objectives for something you currently
teach or plan to teach using Fink's Taxonomy. Working with a partner,
critique and improve the objectives.

\exercise{Building Lessons by Subtracting Complexity}{individual}{20}

One way to build a programming lesson is to write the program you want
learners to finish with, then remove the most complex part that you
want them to write and make it the last exercise. You can then remove
the next most complex part you want them to write and make it the
penultimate exercise, and so on. Anything that's left---i.e., anything
you don't want them to write as an exercise---becomes the starter code
that you give them. This typically includes things like importing
libraries and loading data.

Take a program or web page that you want your learners to be able to
create on their own at the end of a lesson and work backward to break
it into digestible parts.  How many are there?  What key idea is
introduced by each one?

\exercise{Inessential Weirdness}{individual}{15}

Betsy Leondar-Wright coined the phrase
``\href{http://www.classmatters.org/2006_07/its-not-them.php}{inessential
  weirdness}'' to describe things groups do that aren't really
necessary, but which alienate people who aren't members of that group.
Sumana Harihareswara later used this notion as the basis for a talk on
\href{https://www.harihareswara.net/sumana/2016/05/21/0}{inessential
  weirdnesses in open source software}, which includes things like
making disparaging comments about Microsoft Windows, command-line
tools with cryptic names, and the command line itself.  Take a few
minutes to read these articles, then make a list of inessential
weirdnesses you think your learners might encounter when you first
teach them. How many of these can you avoid with a little effort?

\exercise{PRIMM}{individual}{15}

One approach to introducing new ideas in computing is
\href{http://blogs.kcl.ac.uk/cser/2017/09/01/primm-a-structured-approach-to-teaching-programming/}{PRIMM}:
\textbf{P}redict a program's behavior or output, \textbf{R}un it to
see what it actually does, \textbf{I}nvestigate why it does that
(e.g., by stepping through it in a debugger or drawing the flow of
control), \textbf{M}odify it (or its inputs), and then \textbf{M}ake
something similar from scratch.  Pick something you have recently
taught or been taught and outline a short lesson that follows these
five steps.

\exercise{Evaluating Lessons}{pairs}{20}

\cite{Mart2017} specifies eight dimensions along which lessons can be
evaluated:

\begin{description}

  \item[Closed vs.\ open:] is there a well-defined path and endpoint,
    or are learners exploring?

  \item[Cultural relevance:] how well is the task connected to things
    they do outside class?

  \item[Recognition:] how easily can the learner share the product of
    their work?

  \item[Space to play:] seems to overlap closed vs. open

  \item[Driver shift:] how often are learners in control of the
    learning experience (tight cycles of ``see then do'' score highly)

  \item[Risk reward:] to what extent is taking risks rewarded or
    recognized?

  \item[Grouping:] is learning individual, in pairs, or in larger
    groups?

  \item[Session shape:] theater-style classroom, dinner seating, free
    space, public space, etc.

\end{description}

Working with a partner, go through a set of lessons you have recently
taught, or have recently been taught, and rate them as ``low'',
``medium'', ``high'', or ``not applicable'' on each of these criteria.
Which two criteria are most important to you personally as a teacher?
As a learner?

\exercise{Concrete-Representational-Abstract}{pairs}{15}

\href{https://makingeducationfun.wordpress.com/2012/04/29/concrete-representational-abstract-cra/}{Concrete-Representational-Abstract}
(CRA) is another approach to introducing new ideas that is used
primarily with younger learners.  The first step is the concrete
stage, and involves physically manipulating objects to solve a problem
(e.g., piling blocks to do addition).  In the the representational
stage, images are used to represent those objects, and in the final
abstract stage, the learner uses numbers or symbols.

\begin{enumerate}

\item
  Write each of the numbers 2, 7, 5, 10, 6 on a sticky note.

\item
  Simulate a loop that finds the largest value by looking at each in
  turn (concrete).

\item
  Sketch a diagram of the process you used, labelling each step
  (representational).

\item
  Write instructions that someone else could follow to go through the
  same steps you used (abstract).

\item
  Compare your representational and abstract materials with those of
  your partner.
  
\end{enumerate}
