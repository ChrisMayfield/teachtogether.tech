\chapter{Lesson Design Template}\label{s:template}

Designing a good course is as hard as designing good software. To help
you, this appendix summarizes a process based on evidence-based teaching
practices:

\begin{itemize}

\item
  It lays out a step-by-step progression to help you figure out what to
  think about in what order.

\item
  It provides spaced deliverables so you can re-scope or redirect
  effort without too many unpleasant surprises.

\item
  Everything from Step 2 onward goes into your final course, so there is
  no wasted effort.

\item
  Writing sample exercises early lets you check that everything you want
  your students to do actually works.

\end{itemize}

This backward design process was developed independently by
\cite{Wigg2005,Bigg2011,Fink2013}. We have slimmed it down by
removing steps related to meeting curriculum guidelines and other
institutional requirements.

Note that the steps are described in order of increasing detail, but
the process itself is always iterative. You will frequently go back to
revise earlier work as you learn something from your answer to a later
question or realize that your initial plan isn't going to play out the
way you first thought.

\section*{Step 1: Brainstorming}

The first step is to throw together some rough ideas so that you and
your colleagues can make sure your thoughts about the course are
aligned. To do this, write some point-form answers to three or four of
the questions listed below. You aren't expected to answer all of them,
and you may pose and answer others if you think it's helpful, but you
should always include a couple of answers to the first.

\begin{enumerate}
\item
  What problem(s) will student learn how to solve?
\item
  What concepts and techniques will students learn?
\item
  What technologies, packages, or functions will students use?
\item
  What terms or jargon will you define?
\item
  What analogies will you use to explain concepts?
\item
  What heuristics will help students understand things?
\item
  What mistakes or misconceptions do you expect?
\item
  What datasets will you use?
\end{enumerate}

You may not need to answer every question for every course, and you will
often have questions or issues we haven't suggested, but couple of hours
of thinking at this stage can save days of rework later on.

Deliverable: a rough scope for the course that you have agreed with your
colleagues.

\section*{Step 2: Who Is This Course For?}

``Beginner'' and ``expert'' mean different things to different people,
and many factors besides pre-existing knowledge influence who a course
is suitable for. The second step in designing a course is therefore to
figure out who your audience is. To do this, you should either create
some learner personas (\secref{s:process-personas}), or (preferably)
reference ones that you and your colleagues have drawn up together.

After you are done brainstorming, you should go through these personas
and decide which of them your course is intended for, and how it will
help them. While doing this, you should make some notes about what
specific prerequisite skills or knowledge you expect students to have
above and beyond what's in the persona.

Deliverable: brief summaries of who your course will help and how.

\section*{Step 3: What Will Learners Do Along the Way?}

The best way to make the goals in Step 1 firmer is to write full
descriptions of a couple of exercises that students will be able to do
toward the end of the course. Writing exercises early is directly
analogous to
\href{https://en.wikipedia.org/wiki/Test-driven_development}{test-driven
development}: rather than working forward from a (probably ambiguous)
set of learning objectives, designers work backward from concrete
examples of where their students are going. Doing this also helps
uncover technical requirements that might otherwise not be found until
uncomfortably late in the lesson development process.

To complement the full exercise descriptions, you should also write
brief point-form descriptions of one or two exercises per lecture hour
to show how quickly you expect learners to progress. (Again, these serve
as a good reality check on how much you're assuming, and help uncover
technical requirements.) One way to create these ``extra'' exercises is
to make a point-form list of the skills needed to solve the major
exercises and create an exercise that targets each.

Deliverable: 1--2 fully explained exercises that use the skills the
student is to learn, plus half a dozen point-form exercise outlines.

Note: be sure to include solutions with example code so that you can
check that your software can do everything you need.

\section*{Step 4: How Are Concepts Connected?}

In this stage, you put the exercises in a logical order then derive a
point-form course outline for the entire course from them. This is also
when you will consolidate the datasets your formative assessments have
used.

Deliverable: a course outline.

Note:

\begin{itemize}
\item
  The final outline should be at the lecture and formative assessment
  level, e.g., one major bullet point for each hour of work with 3--4
  minor bullet points for the episodes in that hour.
\item
  It's common to change assessments in this stage so that they can build
  on each other.
\item
  You are likely to discover things you forgot to list earlier during
  this stage, so don't be surprised if you have to double back a few
  times.
\end{itemize}

\section*{Step 5: Course Overview}

You can now write a course overview consisting of:

\begin{itemize}
\item
  a one-paragraph description (i.e., a sales pitch to students)
\item
  half a dozen learning objectives
\item
  a summary of prerequisites
\end{itemize}

Doing this earlier often wastes effort, since material is usually added,
cut, or moved around in earlier steps.

Deliverable: course description, learning objectives, and prerequisites.

Note: see the appendix for a discussion of how to write good learning
objectives.

\section*{Reminder}

As noted at the start, this process is described as a sequence, but in
practice you will loop back repeatedly as each stage informs you of
something you overlooked.
