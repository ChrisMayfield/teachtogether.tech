\chapter{Individual Learning}\label{s:individual}

\begin{objectives}

\item Explain what metacognition is and why it is important to
  learning.

\item Explain what near and far transfer are, and correctly identify
  which one occurs most often.

\item Name and explain six strategies learners can use to accelerate
  their learning.

\item Explain why working long hours reduces productivity.

\item Define calibrated peer review and explain its benefits for
  learning.

\item List common myths about computing education.

\end{objectives}

The previous three chapters have looked at what instructors can do to
help their learners.  This chapter looks at what learners can do for
themselves by changing their study strategies and getting enough rest.

The key to getting more out of learning is
\glossref{g:metacognition}{metacognition}, or thinking about one's own
thinking processes. Just as good musicians listen to their own
playing, and good teachers reflect on their teaching
(\chapref{s:performance}), learners will learn better and faster if
they make plans, set goals, and monitor their progress.  It's
difficult for learners to master these skills in the abstract---for
example, just telling them to make plans doesn't have any effect---but
lessons can be designed to encourage certain study practices, and
drawing attention to these practices in class helps them realize that
learning is a skill that can be improved like any other
\cite{McGu2015,Miya2018}.

The big prize is transfer of learning, which occurs when one thing we
have learned helps us learn something else more quickly.  Researchers
distinguish between \glossref{g:near-transfer}{near transfer}, which
occurs between similar or related areas like fractions and decimals,
or loops in different programming languages, and
\glossref{g:far-transfer}{far transfer}, which occurs between
dissimilar domains---the idea that learning to play chess will help
mathematical reasoning or vice versa.

Near transfer undoubtedly occurs---no kind of learning beyond simple
memorization could occur if it didn't---and instructors leverage it
all the time by giving learners exercises that are close in form or
content to what has just been presented in a lesson.  However,
\cite{Sala2017} recently analyzed many studies of far transfer and
concluded that while we might \emph{want} to believe in it:

\begin{quote}

  {\ldots}the results show small to moderate effects. However, the
  effect sizes are inversely related to the quality of the
  experimental design{\ldots} We conclude that far transfer of
  learning rarely occurs.

\end{quote}

When far transfer \emph{does} occur, it seems to happen only once a
subject has been mastered \cite{Gick1987}.  In practice, this means
that learning to program won't help you play chess and vice versa.

\section{Six Strategies}\label{s:individual-strategies}

Psychologists study learning in a wide variety of ways, but have
reached similar conclusions about what actually works \cite{Mark2018}.
The \href{http://www.learningscientists.org/}{Learning Scientists}
have catalogued six of these strategies and summarized them in
\href{http://www.learningscientists.org/downloadable-materials}{a set
  of downloadable posters}.  Teaching these strategies to students,
and mentioning them by name when you use them in class, can help them
learn how to learn faster and better \cite{Wein2018}.

\subsection*{Spaced Practice}

Ten hours of study spread out over five days is more effective than
two five-hour days, and far better than one ten-hour day.  You should
therefore create a study schedule that spreads study activities over
time: block off at least half an hour to study each topic each day
rather than trying to cram everything in the night before an exam
\cite{Kang2016}.

You should also review material after each class (but not immediately
after---take at least a half-hour break).  When reviewing, be sure to
include at least a little bit of older material: for example, spend 20
minutes looking over notes from that day's class, and then 5 minutes
each looking over material from the previous day and from a week
before.  (Doing this also helps you catch any gaps or mistakes in
previous sets of notes while there's still time to correct them or ask
questions: it's painful to realize the night before the exam that you
have no idea why you underlined ``Demodulate!!'' three times.)

When reviewing, make notes about things that you had forgotten: for
example, make a flash card for each fact that you couldn't remember,
or that you remembered incorrectly.  This will help you focus the next
round of study on things that most need attention.

\begin{callout}{The Value of Lectures}

  According to \cite{Mill2016a}, ``The lectures that predominate in
  face-to-face courses are relatively ineffective ways to teach, but
  they probably contribute to spacing material over time, because they
  unfold in a set schedule over time. In contrast, depending on how
  the courses are set up, online students can sometimes avoid exposure
  to material altogether until an assignment is nigh.''

\end{callout}

\subsection*{Retrieval Practice}

Researchers now believe that the limiting factor for long-term memory
is not retention (what is stored), but recall (what can be accessed).
Recall of specific information improves with practice, so outcomes in
real situations can be improved by taking practice tests or
summarizing the details of a topic from memory and then checking what
was and wasn't remembered.  For example, \cite{Karp2008} found that
repeated testing improved recall of word lists from 35\% to 80\%.

Research also shows that recall is better when practice uses
activities similar to those used in testing; for example, writing
personal journal entries helps with multiple-choice quizzes, but less
than doing multiple-choice quizzes \cite{Mill2016a}.  This is called
\glossref{g:transfer-appropriate-processing}{transfer-appropriate
  processing}.

One way to exercise retrieval skills is to solve problems twice.  The
first time, do it entirely from memory without notes or discussion
with peers.  After grading your own work against a rubric supplied by
the instructor, solve the problem again using whatever resources you
want.  The difference between the two shows you how well you were able
to retrieve and apply knowledge.

Another method (mentioned above) is to create flash cards.  In
physical form, a question or other prompt is written on one side, and
the answer is written on the other; in digital form, these are ideal
for deployment on mobile devices like phones.  If you are studying as
part of a group, you can exchange flash cards with a partner; this
also helps you discover important ideas that you may have missed or
misunderstood.

A quicker version of this is
\glossref{g:read-cover-retrieve}{read-cover-retrieve}: as you read
something, cover up key terms or sections with small sticky notes.
When you are done, go through it a second time and see how well you
can guess what's under each of those stickies.

Whatever method you use, don't just practice recalling facts and
definitions: make sure you also check your understanding of big ideas
and the connections between them.  Sketching a concept map and then
comparing it to your notes or to a previously-drawn concept map is a
quick way to do this.

\begin{callout}{Hypercorrection}

  One powerful finding in learning research is the
  \glossref{g:hypercorrection}{hypercorrection effect}
  \cite{Metc2016}.  Most people don't like to be told they're wrong,
  so it's reasonable to assume that the more confident someone is that
  the answer they've given in a test is correct, the harder it is to
  change their mind if they were actually wrong.  However, it turns
  out that the opposite is true: the more confident someone is that
  they were right, the more likely they are not to repeat the error if
  they are corrected.

\end{callout}

\subsection*{Interleaving}

One way you can space your practice is to interleave study of
different topics: instead of mastering one subject, then the next,
then a third, shuffle study sessions.  Even better, switch up the
order: A-B-C-B-A-C is better than A-B-C-A-B-C, which in turn is better
than A-A-B-B-C-C \cite{Rohrer2015}.  This is effective because
interleaving fosters creation of more links between different topics,
which in turn increases retention and recall.

How long you should spend on each item depends on the subject and how
well you know it, but somewhere between 10 and 30 minutes is long
enough for you to get into a state of flow
(\secref{s:individual-time}) but not for your mind to wander.
Interleaving study will initially feel harder than focusing on one
topic at a time, but that's a sign that it's working.  If you are
making flash cards for yourself, or doing practice tests, you should
see improvement after only a couple of days.

\subsection*{Elaboration}

Explaining things to yourself as you go through them helps you
understand and remember them.  One way to do this is to follow up each
answer on a practice quiz with an explanation of why that answer is
correct, or conversely with an explanation of why some other plausible
answer isn't.  Another is to tell yourself how a new idea is similar
to or different from one that you have seen previously.

Talking to yourself may seem like an odd way to study, but
\cite{Biel1995} explicitly trained people in self-explanation, and
yes, they outperformed those who hadn't been trained.  An exercise
that builds on this is to go through code line by line with a group,
having a different person to explain each line in turn and say why it
is there and what it accomplishes.

\cite{Chi1989} found that some learners simply halt when they hit an
unexplained step (or a step whose explanation they don't understand)
when doing mechanics problems in a physics class.  Others who pause
their ``execution'' of the example to generate an explanation of
what's going on learn faster.  Instructors should therefore
demonstrate the latter strategy to learners.

Explaining things to others even works on exams, though the extent of
the benefits are still being studied.  \cite{Cao2017a,Cao2017b} looked
at two-stage exams, i.e., a normal (individual) exam which is then
immediately followed by a second exam in which students work in small
groups to solve a set of problems.  They found significant short-term
gains for students doing exams collaboratively, but not long-term
gains, i.e., the benefits visible a couple of weeks after the mid-term
had faded by the final.  They also found that students in the middle
of the class benefited strongly, and that homogeneous-ability groups
benefited, while heterogeneous groups did not.

\subsection*{Concrete Examples}

One specific form of elaboration is so useful that it deserves its own
heading, and that is the use of concrete examples.  Whenever you have
a statement of a general principle, try to provide one or more
examples of its use, or conversely take each particular problem and
list the general principles it embodies.  \cite{Raws2014} found that
interleaving examples and definitions made it more likely that
learners would remember the latter correctly.

One structured way to do this is the
\href{https://betterexplained.com/articles/adept-method/}{ADEPT
  method}: give an \textbf{A}nalogy, draw a \textbf{D}iagram, present
an \textbf{E}xample, describe the idea in \textbf{P}lain language, and
then give the \textbf{T}echnical details.  Again, if you are studying
with a partner or in a group, you can swap and check work: see if you
agree that other people's examples actually embody the principle being
discussed, or which principles are used in an example that they
haven't listed.

Another useful technique is to teach by contrast, i.e., to show
learners what a solution is \emph{not}, or what kind of problem a
technique \emph{won't} solve.  For example, when showing children how
to simplify fractions, it's important to give them a few like 5/7 that
can't be simplified so that they don't become frustrated looking for
answers that don't exist.

\subsection*{Dual Coding}

The last of the six core strategies that the
\href{http://www.learningscientists.org/}{Learning Scientists}
describe is to present words and images together.  As discussed in
\secref{s:load-split-attention}, different subsystems in our brains
handle and store linguistic and visual information, and if
complementary information is presented through both channels, then
they can reinforce one another.  However, learning is more effective
when the same information is \emph{not} presented simultaneously in
two different channels \cite{Maye2003}, because then the brain has to
expend effort to check the channels against each other.

One way to take advantage of dual coding is to draw or label
timelines, maps, family trees, or whatever else seems appropriate to
the material.  (I am personally fond of pictures showing which
functions call which other functions in a program.)  Drawing a diagram
\emph{without} labels, then coming back later to label it, is
excellent retrieval practice.

\section{Time Management}\label{s:individual-time}

I used to brag about the hours I was working.  Not in so many words,
of course---I had \emph{some} social skills---but I would show up for
class around noon, unshaven and yawning, and casually mention how I'd
been up working 'til 6:00 a.m.

Looking back, I can't remember who I was trying to impress.  Instead,
what I remember is how much of the work I id in those all-nighters I
threw away once I'd had some sleep, and how much damage the stuff I
didn't throw away did to my grades.

My mistake was to confuse ``working'' with ``being productive''.  You
can't produce software (or anything else) without doing some work, but
you can easily do lots of work without producing anything of value.
Convincing people of this can be hard, especially when they're in
their teens or twenties, but pays tremendous dividends.

Scientific study of overwork and sleep deprivation goes back to at
least the 1890s (see \cite{Robi2005} for a short, readable summary).
The most important results for learners are:

\begin{enumerate}

\item Working more than eight hours a day for an extended period of
  time lowers your total productivity, not just your hourly
  productivity---i.e., you get less done in total (not just per hour)
  when you're in crunch mode than you do when you work regular hours.

\item Working over 21 hours in a stretch increases the odds of you
  making a catastrophic error just as much as being legally drunk.

\item Productivity varies over the course of the workday, with the
  greatest productivity occurring in the first four to six hours.
  After enough hours, productivity approaches zero; eventually it
  becomes negative.

\end{enumerate}

These facts have been reproduced and verified for over a century, and
the data behind them is as solid as the data linking smoking to lung
cancer.  The catch is that \emph{people usually don't notice their
  abilities declining}.  Just like drunks who think they're still able
to drive, people who are deprived of sleep don't realize that they're
not finishing their sentences (or thoughts).  Five eight-hour days per
week has been proven to maximize long-term total output in every
industry that has ever been studied; studying or programming are no
different.

But what about short bursts now and then, like pulling an all-nighter
to meet a deadline?  That has been studied too, and the results aren't
pleasant.  Your ability to think drops by 25\% for each 24 hours
you're awake.  Put it another way, the average person's IQ is only 75
after one all-nighter, which puts them in the bottom 5\% of the
population.  Two all nighters in a row, and their effective IQ is 50,
which is the level at which people are usually judged incapable of
independent living.

\begin{callout}{When You Just Can't Say No}

  Research has shown that our ability to exert willpower runs out,
  just like our ability to use muscles: if we have to resist eating
  the last donut on the tray when we're hungry, we are less likely to
  fold laundry and vice versa.  This is called
  \glossref{g:ego-depletion}{ego depletion} \cite{Mill2016a}, and an
  effective counter is to build up habits so that doing the right
  thing is automatic.

\end{callout}

``But---but---we have so many assignments to do!'', your learners say.
``And they're all due at once!  We \emph{have} to work extra hours to
get them all done!''  No: in order to be productive, they have to
prioritize and focus, and in order to do that, they have to be taught
how.  One widely-used technique is to make a list of things that need
to be done, sort them by priority, and then switch off email and other
interruptions for 30-60 minutes and complete one of those tasks.  If
any task on a to-do list is more than an hour long, break it down into
smaller pieces and prioritize those separately.

The most important part of this is switching off interruptions.
Despite what many people want to believe, people are not good at
multi-tasking.  What we can become good at is
\glossref{g:automaticity}{automaticity}, which is the ability to do
something routine in the background while doing something else
\cite{Mill2016a}.  Most of us can talk while chopping onions, or drink
coffee while reading; with practice, we can also take notes while
listening, but we can't study effectively, program, or do other
mentally challenging tasks while paying attention to something else.

The point of all this organization and preparation is to get into the
most productive mental state possible.  Psychologists call it
\glossref{g:flow}{flow} \cite{Csik2008}; athletes call it ``being in
the zone'', while musicians talk about losing themselves in what
they're playing.  Whatever name you use, people produce much more per
unit of time in this state than normal.

That's the good news.  The bad news is that it takes roughly ten
minutes to get back into a state of flow after an interruption, no
matter how short the interruption was.  This means that if you are
interrupted half a dozen times per hour, you are \emph{never} at your
productive peak.

\section{Peer Assessment}\label{s:individual-peer}

Asking people on a team to rate their peers is a common practice in
industry.  \cite{Sond2012} surveyed the literature on student peer
assessment, distinguishing between grading and reviewing.  The
benefits they found included increasing the amount, diversity, and
timeliness of feedback, helping students exercise higher-level
thinking, encouraging reflective practice, and supporting development
of social skills.  The concerns were predictable: validity and
reliability, motivation and procrastination, trolls, collusion, and
plagiarism.  However, while these concerns are legitimate, the
evidence shows that they aren't significant in class.  For example,
\cite{Kauf2000} compared confidential peer ratings and grades on
several axes for two undergraduate engineering courses and found that
self-rating and peer ratings statistically agreed, that collusion
(i.e., everyone giving their peers the same grades) wasn't
significant, that students didn't inflate their self-ratings, and
crucially, that ratings were not biased by gender or race.

One important variation on peer assessment and review is
\glossref{g:contributing-student}{contributing student pedagogy}, in
which students produce artifacts to contribute to other students'
learning.  This can be developing a short lesson and sharing it with
the class, adding to a question bank, or writing up notes from a
particular lecture for in-class publication.  For example,
\cite{Fran2018} found that students who made short videos to teach
concepts to their peers had a significant increase in their own
learning compared to those who only studied the material or viewed the
videos.

Another is \glossref{g:calibrated-peer-review}{calibrated peer
  review}, in which a student reviews one or more examples using a
rubric and compares their evaluation against the instructor's review
of the same work \cite{Kulk2013}.  Only once student's evaluations are
close enough to the instructor's are they allowed to start evaluating
peers' actual work.

As long as evaluation is based on observables, rather than personality
traits, peer assessment can actually be as accurate as assessment by
TAs and other outsiders.  ``Observables'' means that instead of
asking, ``Is the person outgoing,'' or ``Does the person have a
positive attitude,'' assessments should ask, ``Does the person listen
attentively during meetings,'' or, ``Does the person attempt to solve
problems before asking for help.''  The evaluation form in
\appref{s:peereval} shows a sample to get you started.  To use it,
rank yourself and each of your teammates, then calculate and compare
scores.

\section{Exercises}\label{s:individual-exercises}

\exercise{Learning Strategies}{individual}{20}

\begin{enumerate}

\item
  Which of the six learning strategies do you regularly use? Which
  ones do you not?

\item
  Write down three general concepts that you want your learners to
  master, and then give two specific examples of each. (This uses the
  ``concrete examples'' practice).

\item
  For each of those concepts, work backward from one of your examples
  to explain how the concept explains it. (This uses the
  ``elaboration'' practice).

\end{enumerate}

\exercise{Connecting Ideas}{pairs}{5}


This is an exercise is an example of using elaboration to improve
retention.  Pick a partner, and have each person independently choose
an idea, then announce your ideas and try to find a four-link chain
that leads from one to the other.  For example, if the two ideas are
``Saskatchewan'' and ``statistics'', the links might be:

\begin{itemize}
\item
  Saskatchewan is a province of Canada;
\item
  Canada is a country;
\item
  countries have governments;
\item
  governments are elected; and
\item
  people try to predict election results using statistics
\end{itemize}

\exercise{Convergent Evolution}{pairs}{15}

One practice that wasn't covered above is
\glossref{g:guided-notes}{guided notes}, which are instructor-prepared
notes that cue students to respond to key information in a lecture or
discussion.  The cues can be blank spaces where students add
information, asterisks next to terms students should define, etc.

Create 2--4 guided note cards for a lesson you have recently taught or
are going to teach.  Swap cards with your partner: how easy is it to
understand what is being asked for?  How long would it take to fill in
the prompts?

\exercise{Changing Minds}{pairs}{10}

\cite{Kirs2013} argues that myths about digital natives, learning
styles, and self-educators are all reflections of the mistaken belief
that learners know what is best for them, and cautions that we may be
in a downward spiral in which every attempt by education researchers
to rebut these myths confirms their opponents' belief that learning
science is pseudo-science.  Pick one thing you have learned about
learning so far in this book that surprised you or contradicted
something you previously believed, and practice explaining it to a
partner in 1--2 minutes.  How convincing are you?

\exercise{Flash Cards}{individual}{15}

Use sticky notes or anything else you have at hand to make up a dozen
flash cards for a topic you have recently taught or learned, trade
with a partner, and see how long it takes each of you to achieve 100\%
perfect recall.  When you are done, set the cards aside, then come
back after an hour and see what your recall rate is.

\exercise{Using ADEPT}{whole class}{15}

Pick something you have recently taught or been taught and outline a
short lesson that uses the five-step ADEPT method to introduce it.

\exercise{The Cost of Multi-Tasking}{pairs}{10}

\href{http://www.learningscientists.org/blog/2017/7/28-1}{The Learning
  Scientists} blog describes a simple experiment you can do without
preparation or equipment other than a stopwatch to demonstrate the
mental cost of multi-tasking.  Working in pairs, measure how long it
takes each person to do each of these three tasks:

\begin{itemize}

\item
  Count from 1 to 26.

\item
  Recite the alphabet from A to Z.

\item
  Interleave the numbers and letters, i.e., say, ``1, A, 2, B,
  {\ldots}'' and so on.

\end{itemize}

Have each pair report their numbers: you will probably find that the
third (in which you are multi-tasking) takes significantly longer than
either of the component tasks.

\exercise{Myths in Computing Education}{small groups}{20}

Working in groups of 3--4, vote on which of the following statements
are true or false.  When you are done, check your answers against
\cite{Guzd2015b} (which you can
\href{https://cacm.acm.org/blogs/blog-cacm/189498-top-10-myths-about-teaching-computer-science/fulltext}{read
  online}).

\begin{enumerate}

\item The lack of women in Computer Science is just like all the other
  STEM fields.

\item To get more women in CS, we need more female CS faculty.

\item Student evaluations are the best way to evaluate teaching.

\item Good teachers personalize education for students' learning
  styles.

\item A good CS teacher should model good software development
  practice, because their job is to produce excellent software
  engineers.

\item Some people are just naturally better programmers than others.

\end{enumerate}

\exercise{Calibrated Peer Review}{pairs}{20}

\begin{enumerate}

\item
  Create a 5--10 point rubric for grading programs of the kind you
  would like your learners to write that has entries like ``good
  variable names'', ``no redundant code'', and ``properly-nested
  control flow''.

\item
  Choose or create a small program that contains 3--4 violations of
  these entries.

\item
  Grade the program according to your rubric.

\item
  Have your partner grade the same program with the same rubric.  What
  do they accept that you did not?  What do they critique that you did
  not?

\end{enumerate}

\exercise{Top Ten Myths}{whole class}{15}

\cite{Guzd2015b} presents a list of the top 10 mistaken beliefs about
computing education.  His list of things that many people believe, but
which aren't true, is:

\begin{enumerate}

\item
  The lack of women in Computer Science is just like all the other
  fields of science, technology, engineering, and medicine.

\item
  To get more women in CS, we need more female CS faculty.

\item
  A good CS teacher is a good lecturer.

\item
  Clickers and the like are an add-on for a good teacher.

\item
  Student evaluations are the best way to evaluate teaching.

\item
  Good teachers personalize education for students' learning styles.

\item
  High schools just can't teach CS well, so they shouldn't do it at
  all.

\item
  The real problem is to get more CS curriculum into the hands of
  teachers.

\item
  All I need to do to be a good CS teacher is model good software
  development practice, because my job is to produce excellent
  software engineers.

\item
  Some people are just born to program.

\end{enumerate}

\noindent
Have everyone vote +1 (agree), -1 (disagree), or 0 (not sure) for each
point, then
\href{https://cacm.acm.org/blogs/blog-cacm/189498-top-10-myths-about-teaching-computer-science/fulltext}{read
  the full explanations in the original article}) and vote again.
Which ones did people change their minds on?  Which ones do they still
believe are true, and why?
