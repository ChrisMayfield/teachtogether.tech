\chapter{Meetings, Meetings, Meetings}\label{s:meetings}

Most people are really bad at meetings: they don't have an agenda
going in, they don't take minutes, they waffle on or wander off into
irrelevancies, they repeat what others have said or recite banalities
simply so that they'll have said something, and they hold side
conversations (which pretty much guarantees that the meeting will be a
waste of time).  Knowing how to run a meeting efficiently is a core
skill for anyone who wants to get things done. (Knowing how to take
part in someone else's meeting is just as important, but gets far less
attention---as a colleague once said, everyone offers leadership
training, nobody offers followership training.) The most important
rules for making meetings efficient are not secret, but are rarely
followed:

\begin{description}

\item[Decide if there actually needs to be a meeting.] If the only
  purpose is to share information, have everyone send a brief email
  instead. Remember, you can read faster than anyone can speak: if
  someone has facts for the rest of the team to absorb, the most
  polite way to communicate them is to type them in.

\item[Write an agenda.]  If nobody cares enough about the meeting to
  write a point-form list of what's supposed to be discussed, the
  meeting itself probably doesn't need to happen.

\item[Include timings in the agenda.]  Agendas can also help you
  prevent early items stealing time from later ones if you include the
  time to be spent on each item in the agenda.  Your first estimates
  with any new group will be wildly optimistic, so revise them upward
  for subsequent meetings.  However, you shouldn't plan a second or
  third meeting because the first one ran over-time: instead, try to
  figure out why you're running over and fix the underlying problem.

\item[Prioritize.] Every meeting is a micro-project, so work should be
  prioritized in the same way that it is for other projects: things
  that will have high impact but take little time should be done
  first, and things that will take lots of time but have little impact
  should be skipped.

\item[Make one person responsible for keeping things moving.] One
  person should be tasked with keeping items to time, chiding people
  who are having side conversations or checking email, and asking
  people who are talking too much to get to the point.  This person
  should \emph{not} do all the talking; in fact, whoever is in charge
  will talk less in a well-run meeting than most other participants.

\item[Require politeness.] No one gets to be rude, no one gets to
  ramble, and if someone goes off topic, it's the chair's job to say,
  ``Let's discuss that elsewhere.''

\item[No technology] (unless it's required for accessibility reasons).
  Insist that everyone put their phones, tablets, and laptops into
  politeness mode (i.e., closes them).  If this is too stressful, let
  participants hang on to their electronic pacifiers but turn off the
  network so that they really \emph{are} using them just to take notes
  or check the agenda.

\item[No interruptions.] Participants should raise a finger, put up a
  sticky note, or make one of the other gestures people make at
  high-priced auctions instead if they want to speak next.  If the
  speaker doesn't notice you, the person in charge ought to.

\item[Record minutes.] Someone other than the chair should take
  point-form notes about the most important pieces of information that
  were shared, and about every decision that was made or every task
  that was assigned to someone.

\item[Take notes.]  While other people are talking, participants
  should take notes of questions they want to ask or points they want
  to make.  (You'll be surprised how smart it makes you look when it's
  your turn to speak.)

\item[End early.] If your meeting is scheduled for 10:00-11:00, you
  should aim to end at 10:55 to give people time to get where they
  need to go next.

\end{description}

\noindent
As soon as the meeting is over, the minutes should be circulated
(e.g., emailed to everyone or posted to a wiki):

\begin{description}

\item[People who weren't at the meeting can keep track of what's going
  on.]  You and your fellow students all have to juggle assignments
  from several other courses while doing this project, which means
  that sometimes you won't be able to make it to team meetings.  A
  wiki page, email message, or blog entry is a much more efficient way
  to catch up after a missed meeting or two than asking a team mate,
  ``Hey, what did I miss?''

\item[Everyone can check what was actually said or promised.]  More
  than once, I've looked over the minutes of a meeting I was in and
  thought, ``Did I say that?'' or, ``Wait a minute, I didn't promise
  to have it ready then!''  Accidentally or not, people will often
  remember things differently; writing it down gives team members a
  chance to correct mistaken or malicious interpretations, which can
  save a lot of anguish later on.

\item[People can be held accountable at subsequent meetings.]  There's
  no point making lists of questions and action items if you don't
  follow up on them later.  If you're using a ticketing system, the
  best thing to do is to create a ticket for each new question or task
  right after the meeting, and update those that are being carried
  forward.  That way, your agenda for the next meeting can start by
  rattling through a list of tickets.

\end{description}

\cite{Brow2007} and \cite{Broo2016} have lots of good advice on
running meetings, and if you want to ``learn, then do'', an hour of
training on chairing meetings is the most effective place to start.

\begin{callout}{Sticky Notes and Interruption Bingo}

  Some people are so used to the sound of their own voice that they
  will insist on talking half the time no matter how many other people
  are in the room. One way to combat this is to give everyone three
  sticky notes at the start of the meeting. Every time they speak,
  they have to take down one sticky note. When they're out of notes,
  they aren't allowed to speak until everyone has used at least one,
  at which point everyone gets all of their sticky notes back. This
  ensures that nobody talks more than three times as often as the
  quietest person in the meeting, and completely changes the dynamics
  of most groups: people who have given up trying to be heard because
  they always get trampled suddenly have space to contribute, and the
  overly-frequent speakers quickly realize just how unfair they have
  been.

  Another useful technique is called interruption bingo. Draw a grid,
  and label the rows and columns with the participants' names. Each
  time someone interrupts someone else, add a tally mark to the
  appropriate cell. Halfway through the meeting, take a moment to look
  at the results.  In most cases, you will see that one or two people
  are doing all of the interrupting, often without being aware of
  it. After that, saying, ``All right, I'm adding another tally to the
  bingo card,'' is often enough to get them to throttle back.  (Note
  that this technique is intended to manage interruptions, not
  speaking time.  It may be completely appropriate for people with
  more knowledge of a subject to speak about it more often in a
  meeting, but it is almost never appropriate to repeatedly cut people
  off.)

\end{callout}

\section*{Online Meetings}

Chelsea Troy's discussion of
\href{https://chelseatroy.com/2018/03/29/why-do-remote-meetings-suck-so-much/}{why
  online meetings are often frustrating and unproductive} makes an
important point: in most online meetings, the first person to speak
during a pause gets the floor.  The result?  ``If you have something
you want to say, you have to stop listening to the person currently
speaking and instead focus on when they’re gonna pause or finish so
you can leap into that nanosecond of silence and be the first to utter
something.  The format{\ldots}encourages participants who want to
contribute to say more and listen less.''

The solution is to run a text chat beside the video conference where
people can signal that they want to speak, and have the moderator
select people from the waiting list.  If the meeting is large or
argumentative, this can be reinforced by having everyone mute
themselves, and only allowing the moderator to unmute people.

\section*{The Post Mortem}

The most valuable part of a project isn't the software you write or
the grade you're given: it's the post mortem in which you reflect on
what you just accomplished and what you could o better next time.  The
aim of a post mortem is \emph{not} to point the finger of shame at
individuals, although if that has to happen, the post mortem is the
best place for it.

A post mortem is run like any other meeting, but with a few additional
guidelines \cite{Derb2006}:

\begin{description}

\item[Get a moderator who wasn't part of the project] and doesn't have
  a stake in it.  Otherwise, the meeting will either go in circles, or
  focus on only a subset of important topics.  In the case of student
  projects, this moderator might be the course instructor, or a TA.

\item[Set aside an hour, and only an hour.] In my experience, nothing
  useful is said in the first ten minutes of anyone's first post
  mortem, since people are naturally a bit shy about praising or
  damning their own work.  Equally, nothing useful is said after the
  first hour: if you're still talking, it's probably because one or
  two people have a \emph{lot} they want to get off their chests.

\item[Require attendance.] Everyone who was part of the project ought
  to be in the room for the post mortem.  This is more important than
  you might think: the people who have the most to learn from the post
  mortem are often least likely to show up if the meeting is optional.

\item[Make two lists.] When I'm moderating, I put the headings ``Do
  Again'' and ``Do Differently'' on the board, then do a lap around
  the room and ask every person to give me one item (that hasn't
  already been mentioned) for each list.

\item[Comment on actions, rather than individuals.] By the time the
  project is done, some people simply won't be able to stand one
  another.  Don't let this sidetrack the meeting: if someone has a
  specific complaint about another member of the team, require him to
  criticize a particular event or decision.  ``He had a bad attitude''
  does \emph{not} help anyone improve their game.

\end{description}

Once everyone's thoughts are out in the open, organize them somehow so
that you can make specific recommendations about what to do next time.
This list is one of the two major goals of the post mortem (the other
being to give people a chance to be heard).
