\chapter{License}\label{s:license}

\noindent
\emph{This is a human-readable summary of (and not a substitute for)
  the license.  Please see
  \url{https://creativecommons.org/licenses/by/4.0/legalcode} for the
  full legal text.}

This work is licensed under the Creative Commons Attribution 4.0
International license (CC-BY-4.0).

\begin{center}
  \textbf{You are free to:}
\end{center}

\begin{itemize}
\item
  \textbf{Share}---copy and redistribute the material in any medium or
  format
\item
  \textbf{Remix}---remix, transform, and build upon the material for
  any purpose, even commercially.
\end{itemize}

\noindent
The licensor cannot revoke these freedoms as long as you follow the
license terms.

\begin{center}
  \textbf{Under the following terms:}
\end{center}

\begin{itemize}
\item
  \textbf{Attribution}---You must give appropriate credit, provide a
  link to the license, and indicate if changes were made. You may do
  so in any reasonable manner, but not in any way that suggests the
  licensor endorses you or your use.
\item
  \textbf{No additional restrictions}---You may not apply legal terms
  or technological measures that legally restrict others from doing
  anything the license permits.
\end{itemize}

\begin{center}
  \textbf{Notices:}
\end{center}

You do not have to comply with the license for elements of the
material in the public domain or where your use is permitted by an
applicable exception or limitation.

No warranties are given. The license may not give you all of the
permissions necessary for your intended use. For example, other rights
such as publicity, privacy, or moral rights may limit how you use the
material.
