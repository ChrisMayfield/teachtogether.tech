\chapter{Checklists for Events}\label{s:events}

\cite{Gawa2007} popularized the idea that using checklists can save
lives (and make many other things better too). The results of recent
studies have been more nuanced \cite{Avel2013,Urba2014}, but we
still find them useful, particularly when bringing new instructors
onto a team.

The checklists below are used before, during, and after instructor
training events, and can easily be adapted for end-learner workshops as
well. We recommend that every group build and maintain its own
checklists customized for its instructors' and learners' needs.

\section*{Scheduling the Event}

\begin{enumerate}
\item
  Decide if it will be in person, online for one site, or online for
  several sites.
\item
  Talk through expectations with the host(s) and make sure that everyone
  agrees on who is covering travel costs.
\item
  Determine who is allowed to take part: is the event open to all
  comers, restricted to members of one organization, or something in
  between?
\item
  Arrange instructors.
\item
  Arrange space, including breakout rooms if needed.
\item
  Choose dates. If it is in person, book travel.
\item
  Get names and email addresses of attendees from host(s).
\item
  Make sure they are added to the registration system.
\end{enumerate}

\section*{Setting Up}

\begin{enumerate}
\item
  Set up a web page with details on the workshop, including date,
  location, and a list of what participants need to bring.
\item
  Check whether any attendees have special needs.
\item
  If the workshop is online, test the video conferencing link.
\item
  Make sure attendees will all have network access.
\item
  Create an Etherpad or Google Doc for shared notes.
\item
  Email attendees a welcome message that includes a link to the workshop
  home page, background readings, and a description of any prerequisite
  tasks.
\end{enumerate}

\section*{At the Start of the Event}

\begin{enumerate}
\item
  Remind everyone of the code of conduct.
\item
  Collect attendance.
\item
  Distribute sticky notes.
\item
  Collect any relevant online account IDs.
\end{enumerate}

\section*{At the End of the Event}

\begin{enumerate}
\item
  Update attendance records. Be sure to also record who participated as
  an instructor or helper.
\item
  Administer a post-workshop survey.
\item
  Update the course notes and/or checklists.
\end{enumerate}

\section*{Travel Kit}

Here are a few things instructors take with them when they travel to
teach:

\begin{itemize}
\item
  sticky notes
\item
  cough drops
\item
  comfortable shoes
\item
  a small notepad
\item
  a spare power adapter
\item
  a spare shirt
\item
  deodorant
\item
  a variety of video adapters
\item
  laptop stickers
\item
  a toothbrush or some mouthwash
\item
  a granola bar or some other emergency snack
\item
  Eno or some other antacid (because road food)
\item
  business cards
\item
  a printed copy of the notes, or a tablet or other device
\item
  an insulated cup for tea/coffee
\item
  spare glasses/contacts
\item
  a notebook and pen
\item
  a portable WiFi hub (in case the room's network isn't working)
\item
  extra whiteboard markers
\item
  a laser pointer
\item
  a packet of wet wipes (because spills happen)
\item
  USB drives with installers for various operating systems
\item
  running shoes, a bathing suit, a yoga mat, or whatever else you
  exercise in or with
\end{itemize}
