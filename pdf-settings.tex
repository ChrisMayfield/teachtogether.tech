\usepackage{fix-cm}
\usepackage[english]{babel}
\usepackage{amssymb}
\usepackage{amsmath}
\usepackage{graphicx}
\usepackage{subfigure}
\usepackage{makeidx}
\usepackage{multicol}
\usepackage{inputenc}
\usepackage{epigrafica}
\usepackage[LGR,OT1]{fontenc}
\usepackage{mathptmx}
\usepackage{changepage}
\usepackage[center]{caption}
\usepackage{float}

\hfuzz=30pt % Don't bother to report over-full boxes < 10pt
\vfuzz=30pt % Don't bother to report over-full boxes < 10pt


% Use numbers for nested lists all the way down.
\usepackage{enumitem}
\setlist[enumerate,1]{label=\arabic*., ref=\arabic*}
\setlist[enumerate,2]{label=\arabic*., ref=\arabic*}
\setlist[enumerate,3]{label=\arabic*., ref=\arabic*}

% Allow multi-page tables.
\usepackage{longtable}

% Make description items cross-referenceable
\def\namedlabel#1#2{\begingroup
    #2%
    \def\@currentlabel{#2}%
    \phantomsection\label{#1}\endgroup
}
\makeatother

% Change hanging indentation for description environment.
\makeatletter
\renewenvironment{description}%
  {\list{}{\leftmargin=1em\itemsep3pt % <------- Adjust this length
    \labelwidth\z@ \itemindent-\leftmargin
    \let\makelabel\descriptionlabel}}%
  {\endlist}
\makeatother

% Use smaller font for verbatim environments.
\usepackage{fancyvrb}

% Syntax highlighting with minted <http://mirror.its.dal.ca/ctan/macros/latex/contrib/minted/minted.pdf>
\usepackage{minted}%

\frenchspacing
\tolerance=5000

%----------------------------------------

% Ignore subtitle in PDF version.
\newcommand{\subtitle}[1]{}

% Asides
\newenvironment{aside}[1]%
{\addvspace{6pt}\begin{adjustwidth}{2em}{2em}%
\centerline{\itshape\textbf{#1}}\par\itshape\noindent\ignorespaces}%
{\end{adjustwidth}\addvspace{6pt}}


% Remove indentation from footnotes
\usepackage[hang,flushmargin]{footmisc}

%----------------------------------------

% Render hyperlinks with footnotes as well.
\newcommand{\hreffoot}[2]{{#2}\footnote{\href{#1}{#1}}}

% Mark recommendations.
\newcommand{\recommendation}[1]{\emph{#1}}

\usepackage[T1]{fontenc}

% Bibliography.
\usepackage[backend=biber,style=alphabetic,sorting=nyt,maxbibnames=99]{biblatex}
\addbibresource{book.bib}

% https://tex.stackexchange.com/questions/8428/use-bibtex-key-as-the-cite-key
\DeclareFieldFormat{labelalpha}{\thefield{entrykey}}
\DeclareFieldFormat{extraalpha}{}

\setlength{\biblabelsep}{\labelsep}% <-- adjust this to your liking, the standard is 2\labelsep
\defbibenvironment{bibliography}
  {\addcontentsline{toc}{chapter}{\bfseries References}\list
     {\printtext[labelalphawidth]{%
        \printfield{prefixnumber}%
        \printfield{labelalpha}%
        \printfield{extraalpha}}}
     {\setlength{\labelsep}{\biblabelsep}%
      \setlength{\leftmargin}{24pt}%\labelsep
      \setlength{\itemsep}{\bibitemsep}%
      \setlength{\parsep}{\bibparsep}}%
      \renewcommand*{\makelabel}[1]{\bf##1\hss}}
  {\endlist}
  {\item}
\bibparsep3pt

\usepackage{csquotes}
%----------------------------------------

\def\UrlFont{\rmfamily}

% Chapters, sections, and appendices with labels.
\newcommand{\chaplbl}[2]{\chapter{#1}\label{#2}}
\newcommand{\seclbl}[2]{\section{#1}\label{#2}}
\newcommand{\appref}[1]{Appendix~\ref{#1}}
\newcommand{\chapref}[1]{Chapter~\ref{#1}}
\newcommand{\figref}[1]{Figure~\ref{#1}}
\newcommand{\secref}[1]{Section~\ref{#1}}

% Image figures.
\newcommand{\figimg}[3]{\begin{figure}%
\centering%
\includegraphics[width=\textwidth]{#1}%
\caption{#2}%
\label{#3}%
\end{figure}}

% PDF figures.
\newcommand{\figpdf}[3]{\begin{figure}%
\centering%
\includegraphics[scale=0.6]{#1}%
\caption{#2}%
\label{#3}%
\end{figure}}

% PDF figures forcing location.
\newcommand{\figpdfhere}[3]{\begin{figure}[H]%
\centering%
\includegraphics[scale=0.6]{#1}%
\caption{#2}%
\label{#3}%
\end{figure}}

% Exercise headings.
\newcommand{\exercise}[3]{\subsection*{#1 (#2/#3)}}

\setlength{\paperheight}{11in}

%----------------------------------------

% Create an index.
\makeindex

%----------------------------------------
 
% Always load hyperref last
% https://tex.stackexchange.com/questions/16268/warning-with-footnotes-namehfootnote-xx-has-been-referenced-but-does-not-exi
\usepackage{hyperref}

% Glossary references and items.
\newcommand{\gref}[2]{\hyperlink{#1}{\textbf{#2}}\index{#2}}
\newcommand{\grefdex}[3]{\hyperlink{#1}{\textbf{#2}}\index{#3}}
\newcommand{\grefcross}[2]{\hyperlink{#1}{\textbf{#2}}}
\newcommand{\gitem}[2]{\item[\hypertarget{#1}{#2}]}
